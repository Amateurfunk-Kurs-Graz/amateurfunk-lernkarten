\documentclass[avery5371,grid,frame,a4paper]{flashcards}
\geometry{a4paper,portrait,top=\topoffset,left=46pt,right=46pt,twoside=true,bottom=0.0in,noheadfoot}
\usepackage[utf8]{inputenc}
\usepackage[T1]{fontenc}
\usepackage[ngerman]{babel}
\usepackage{amsmath}
\usepackage[inline]{enumitem}
\usepackage{eurosym}
\usepackage{siunitx}
\usepackage{multirow}
%\newcommand{\cardpaper}{a4paper}
%\newcommand{\cardpapermode}{portrait}
\renewcommand{\cardrows}{4}
\renewcommand{\cardcolumns}{2}
\cardfrontstyle[\large\slshape]{headings}
\cardbackstyle{empty}
%\cardbackstyle{plain}

\DeclareSIUnit{\pers}{pers}
\DeclareSIUnit{\EUR}{\text{\euro}}

\sisetup{
  range-phrase = --,
  detect-weight = true,
  detect-family = true
}

\setitemize{itemsep=0pt}

\newcommand\question[2]{
  \begin{flashcard}[{\chap} -- #1]{#2}\end{flashcard}
}
\newcommand{\card}[3]{
  \begin{flashcard}[{\chap} -- #1]{#2}#3\end{flashcard}
}
\newcommand\class[1]{{\footnotesize [Klassen: #1]}}

\begin{document}
% Stand 15. November 2014
% http://www.bmvit.gv.at/bmvit/telekommunikation/funk/funkdienste/downloads/amateur_fragen.pdf

% Rechtliches

\def\chap{Rechtliches \class{1,3,4}}

\card{01}{Welche gesetzlichen Bestimmungen sind für den Amateurfunk maßgeblich?}{
  \begin{itemize}
    \item Internationaler Fernmeldevertrag,
    \item Vollzugsordnung für Funkdienst (VO-Funk),
    \item Telekommunikationsgesetz,
    \item Amateurfunk-Gesetz,
    \item Amateurfunk-Verordnung,
    \item Amateurfunkgebühren-Verordnung,
    \item Kundmachung der Staaten, die Einwände gegen Amateurfunk erhoben haben.
  \end{itemize}
}

\card{02}{Was ist die ,,ITU``?}{
  \begin{itemize}
    \item Internationale Fernmeldeunion
    \item völkerrechtlicher Verein
    \item anerkennt Hoheitsrechte
    \item fördert Beziehungen und Zusammenarbeit der Länder durch guten Fernmeldedienst
  \end{itemize}
}

\card{03}{Welche Zwecke verfolgt der internationale Fernmeldevertrag?}{
  \begin{itemize}
    \item Aufrechterhaltung, Ausbau der Zusammenarbeit zur Verbesserung
    \item Verwendung der Fernmeldeeinrichtungen
    \item technische Entwicklung
    \item Leistungserhöhung der Dienste
    \item Steigerung der Inanspruchnahme (öffentlich)
    \item Verbilligung
  \end{itemize}
}

\card{04}{Welche Aufgaben hat das Radiocommunication Bureau?}{
  \begin{itemize}
    \item Registrierung der Frequenzen,
    \item Anerkennung der Frequenzen,
    \item Beratung, auch im Hinblick gestörter Frequenzen
 \end{itemize}
}
\card{05}{Was ist die CEPT und welche Bedeutung hat sie?}{
  \begin{itemize}
    \item Konferenz der europäischen Post und Fernmeldeverwaltungen
    \item 43 europäische Staaten
    \item Australien, USA erkennt sie an
    \item Zweck:
      \begin{itemize}
        \item Beziehungen vertiefen
        \item Zusammenarbeit fördern
        \item Markt für TK schaffen
      \end{itemize}
  \end{itemize}
}

\card{06}{Was ist die VO Funk (Radio Regulations) und was regelt sie?}{
  \begin{itemize}
    \item Vollzugsordnung für den Funkdienst
    \item Bestandteil des Internationalen Fernmeldevertrags
    \item Bestimmungen über die Praxis
    \item Für Amateurfunker wichtig, weil alle Bestimmungen auch für AF gelten
    \item Frequenz muss stabil und frei von Nebenaussendungen sein (state-of-the-art)
  \end{itemize}
}

\card{07}{Definieren Sie den Begriff ,,Funkanlage`` im Sinne des TKG.}{
  \begin{itemize}
    \item Sende/Empfangseinrichtung
    \item beabsichtigte Informationsübertragung
    \item ohne Verbindungsleitungen
    \item mittels elektromagnetischer Wellen
  \end{itemize}
}

\card{08}{Erläutern Sie den Unterschied zwischen einem Telekommunikationsdienst und dem Amateurfunkdienst?}{
  \begin{description}
    \item[Telekommunikationsdienst]
      gewerblich, Signalübertragung über Kommunikationsnetze einschließlich Telekommunikation
      (alles außer Rundfunk)- und Übertragungsdienste in Rundfunknetze
    \item[Amateurfunk]
      \begin{itemize}
        \item technisch/experimentell
        \item Erd/Weltraumfunkstellen
        \item eigene Ausbildung, Verkehr mit anderen, Not/Katastrophendienst, technische Studien
      \end{itemize}
  \end{description}
}

\card{09}{Wann erlischt eine Bewilligung? Was kann passieren, wenn Sie ohne oder ohne entsprechende Amateurfunkbewilligung Amateurfunk betreiben?}{
  \begin{itemize}
    \item Tod
    \item Ablauf der Zeit
    \item Verzicht
    \item Widerruf (Verstoß gegen Bestimmungen)
  \end{itemize}
  Urkunde ist innerhalb 2~Monaten ans Fernmeldebüro zurückzusenden
}

\card{10}{Was passiert, wenn man ohne Bewilligung funkt?}{
  Verwaltungsübertretung / Verwaltungsstrafe
  \SI{3633}{\EUR}
}

\card{11}{Welche Funkanlagen sind bewilligungspflichtig, welche Art der Bewilligungen gibt es?}{
  \small
  \begin{itemize}
    \item Funkanlagen grundsätzlich bewilligungspflichtig
    \item BMVIT kann für Gerätearten/type generell Errichtung und Betrieb bewilligen
    \item BMVIT kann Einfuhr, Vertrieb und Besitz generell für bewilligungspflichtig erklären (öff. Sicherheit, Behörden)
    \item AF-Bewilligung berechtigt zum Besitz von AF-Sendeanlagen, zu Änderung und Selbstbau, zur Einfuhr, zum vorübergehenden Besitz von Funkanlagen, die keine AF sind (3~Monate), zwecks Umbau zur AF für Eigenbedarf
  \end{itemize}
}

\card{12}{Sie ändern den Standort Ihrer Funkanlage – was haben Sie zu tun?}{
  \footnotesize
  Bedarf einer Bewilligung:
    \begin{itemize}
      \item Standortänderung
      \item Verwendung außerhalb des bewilligten Einsatzgebietes
      \item technische Änderung
    \end{itemize}
  Behörde kann Bewilligungen ändern:
    \begin{itemize}
      \item zur Sicherheit des TK-Verkehrs
      \item aus technischen/betrieblichen Belangen
      \item aus internationalen Gründen (Fernmeldevertragsrecht, geänderte Frequenznutzung)
      \item Schonung wirtschaftl./betrieblicher Interessen; man muss auf eigene Kosten nachkommen (ang. Frist)
    \end{itemize}
}

\card{13}{Was versteht man unter dem Aufsichtsrecht der Fernmeldebehörden über Telekommunikationsanlagen?}{
  \small
  \begin{itemize}
    \item
      TKG Kommunikationsdienste unterliegen der Aufsicht der Regulierungsbehörde
      (Organe der Fernmeldebehörden, des Büros für Funkanlagen und TK-Endeinrichtungen)
    \item
      Die Organe haben der Reg.behörde Hilfe insbesondere bei fernmeldetechnischen Fragen zu leisten
    \item
      TK-Anlagen unterliegen der Aufsicht der Fernmeldebehörden.
      TK-Anlagen sind Anlagen/Geräte zur Abwicklung von Kommunikation, Kabelrundfunknetze, Funkanlage, TK-Endeinrichtungen
  \end{itemize}
}

\card{14}{Ein Organ der Fernmeldebehörde will ihre Funkanlage überprüfen, was haben Sie zu tun?}{
  \begin{itemize}
    \item Organen (Ausweis!) der FMB sind berechtigt, TK-Anlagen (Funkanlagen, Endgeräte) bzw. Teile auf Einhaltung der Gesetze und Verordnungen zu prüfen
    \item Der Zugang ist ihnen zu gestatten
    \item Auskünfte, Unterlagen, \dots
    \item ,,Vorführung`` der Anlagen, auf eigene Kosten
  \end{itemize}
}

\card{14}{Welche Geheimhaltungspflichten treffen Sie als Funkamateur?}{
  Werden mittels Anlage Nachrichten empfangen, die nicht für die Anlage, das Endgerät, den Benutzer bestimmt sind:

  \begin{itemize}
    \item Inhalt der Nachricht bzw. Tatsache des Empfangs darf nicht aufgezeichnet bzw. anderen mitgeteilt bzw. verwertet werden.
    \item Aufgezeichnete Nachrichten sind zu löschen.
  \end{itemize}
}

\card{16}{Was kann die Fernmeldebehörde machen, falls Sie einen anderen Funkdienst stören?}{
  \small
  \begin{itemize}
    \item Bei Störungen einer TK-Anlage durch eine andere können zweckmäßige Maßnahmen angeordnet und vollzogen werden, die zum Schutz der gestörten Anlagen notwendig sind. Vermeidung überflüssiger Kosten.
    \item Unbefugt errichtete / betriebene TK-Anlagen können ohne Androhung außer Betrieb gesetzt werden.
    \item Für sonstige entgegen den Bestimmungen errichtete / betriebene TK-Anlagen gilt das nur zur Sicherung / Wiederherstellung ungestörter Kommunikation.
  \end{itemize}
}

\card{17}{Welche Gebühren müssen als Funkamateur entrichtet werden?}{
  \begin{center}
    \begin{tabular}{ccc}
      A & \SI{100}{\watt} & \SI{1,45}{\EUR} \\
      B & \SI{200}{\watt} & \SI{2,91}{\EUR} \\
      C & \SI{400}{\watt} & \SI{4,36}{\EUR} \\
      D & \SI{1000}{\watt} & \SI{6,54}{\EUR}
    \end{tabular}
  \end{center}
  \begin{itemize}
    \item Klubfunkstelle: \SI{6,54}{\EUR}
    \item Klubfunkstelle (Vereinsräume, Räume Organisationen im öffentlichen Interesse) zu Unterrichtszwecken ohne strahlender Antenne / Fernwirkung: \SI{1,45}{\EUR}
  \end{itemize}
}

\card{18}{Definieren Sie den Begriff ,,Amateurfunkdienst``?}{
  \begin{itemize}
    \item technisch / experimentell
    \item Erd / Weltraumfunkstellen
    \item von Funkamateuren für:
      \begin{itemize}
        \item Ausbildung
        \item Verkehr untereinander
        \item Not / Katastrophenfunk
        \item technische Studien
      \end{itemize}
  \end{itemize}
}

\card{19}{Definieren Sie den Begriff ,,Funkamateure``?}{
  Das ist eine Person
  \begin{itemize}
    \item der die Amateurfunkbewilligung erteilt wurde
    \item die sich mit Funktechnik/Betrieb beschäftigt
    \item persönliche Neigung bzw. Organisation im öffentlichen Interesse
    \item jedoch nicht kommerziell / politisch engagiert
  \end{itemize}
}

\card{20}{Definieren Sie den Begriff ,,Amateurfunkstelle``?}{
  \begin{itemize}
    \item Einer oder mehrere, oder Gruppe von Sendern und Empfängern (Zusatzeinrichtungen)
    \item zum Betrieb des Amateurfunkdienstes an einem bestimmten Ort
    \item Erfassen von in Österreich dem AFU-Dienst zugewiesene Frequenzbereiche, auch wenn der Sende/Empfangsbereich über diese Frequenzbereiche hinausgeht
  \end{itemize}
}

\card{21}{Definieren Sie den Begriff ,,Stationsverantwortlicher``?}{
  Natürliche Person, namhaft gemacht
  \begin{itemize}
    \item von Amateurfunkverein / von einer Organisation im öffentlichen Interesse
    \item verantwortlich für die Einhaltungen der Bestimmungen / Verordnungen des AFG
  \end{itemize}
}

\card{22}{Definieren Sie den Begriff ,,Klubfunkstelle``?}{
  Amateurfunkstelle eines Amateurfunkvereins oder einer im öffentlichen Interessen tätigen Organisation
}

\card{23}{Definieren Sie den Begriff ,,Bakensender``?}{
  automatische Amateurfunksendeanlage
  \begin{itemize}
    \item fester Standort
    \item sendet ständig technische und betriebliche Merkmale
    \item Zweck: Frequenzmessung / Erforschung der Funkausbreitungsbedingungen
  \end{itemize}
}

\card{24}{Definieren Sie den Begriff ,,Relaisfunkstelle``?}{
  automatische Amateurfunksendeanlage: \\
  Amateurfunkstelle, die der automatischen Informationsübertragung dient
}

\card{25}{Darf Amateurfunk von Nichtamateuren abgehört werden?}{Ja, jeder darf abhören.}

\card{26}{Voraussetzungen zur Erlangung einer Amateurfunkbewilligung?}{
  \small
  \begin{itemize}
    \item Errichtung/Betrieb AF-Stelle nur mit Bewilligung
    \item Ausnahmen: Mitbenutzung, Funkempfangsanlage, die nur AF-Frequenzbereiche abdeckt.
    \item Bewilligung ist Personen auf Antrag zu erteilen, wenn: 14. Lebensjahr vollendet, Amateurfunkprüfung abgelegt, befreit oder §~25. Nichtvollhandlungsfähige: Haftung einer vollhandlungsfähigen Person bez. Gebührenforderung.
    \item Bewilligung für AF-Verein/Organisation: Stationsverantwortlicher mit Hauptwohnsitz im Inland (handlungsfähig, Prüfung abgelegt, befreit oder §~25)
  \end{itemize}
}

\card{27}{Wie und wo ist ein Antrag auf Erteilung einer Amateurfunkbewilligung zu stellen?}{
  \small
  Schriftlich, Daten des Antragstellers/des Stationsverantwortlichen:
  \begin{itemize}
    \item
      Vor- und Zuname, Geburtsdatum, Hauptwohnsitz, Standort und Gebiet der AF-Stelle,
      Leistungsstufe, Bewilligungsklasse, technisch Merkmale
    \item
      Beizulegen: Amateurfunkprüfungszeugnis, Bescheid über Befreiung, §~25-Zeugnis,
      Vorschlag Rufzeichen, kein Anspruch.
    \item Entscheidung über Antrag: zuständig. Fernmeldebüro (für Ausländer: FMB für W/NÖ/B)
  \end{itemize}
}

%\card{28}{Welche Angaben stehen in einer Amateurfunkbewilligung?}{\begin{itemize} \item Vor- / Zuname \item Geburtsdatum \item Hauptwohnsitz \item Standort und Gebiet der AF-Stelle \item Leistungsstufe \item Bewilligungsklasse \item Rufzeichen \item technisch Merkmale\end{itemize}}

\card{28}{Rufzeichen und Sonderrufzeichen?}{
  In der Amateurfunkbewilligung ist ein Rufzeichen zuzuweisen.
  Auf Antrag kann BMVIT zu besonderen Anlässen Sonderrufzeichen befristet zuweisen.
  BMVIT kann FMB ermächtigen Sonderrufzeichen zuzuweisen.
  Rufzeichen aussenden: zu Beginn, während Übertragung wiederholt, am Ende.
  Bei Klubfunkstelle: Klubfunkstellenrufzeichen mit Zustimmung des Stationsverantwortlichen auch eigenes Rufzeichen (nur Berechtigungsumfang!)
}

\card{29}{Wozu berechtigt eine Amateurfunkbewilligung?}{
  \small
  Berechtigt zur Errichtung, zum Betrieb
  \begin{itemize}
    \item 1+ fester AF-Stellen (angegebene Standorte)
    \item 1+ beweglicher AF-Stellen (gesamtes Bundesgebiet)
    \item vorübergehend (3 Monate) feste AF-Stelle an einem anderen Ort im Bundesgebiet als angegeben
  \end{itemize}
  Berechtigt zum Besitz von AF-Sendeanlagen und
  \begin{itemize}
    \item Änderung / Selbstbau
    \item Einfuhr für den Eigenbedarf
    \item Besitz von Nicht-AF-Anlagen zum Zweck des Umbaus (vorübergehend, 3 Monate)
  \end{itemize}
}

\card{30}{Unter welchen Voraussetzungen dürfen Aussendungen durchgeführt werden?}{
  \small
  Aussendungen mit einer AF-Stelle nur
  \begin{itemize}[leftmargin=10pt]
    \item in den zugewiesenen Frequenzen (Bewilligungsklasse)
    \item in der festgesetzten Sendeart (BWK)
    \item mit der erlaubten Sendeleistung (abhängig von Leistungsstufe des Frequenzbereichs und AF-Bewilligung)
    \item mit der erlaubten Bandbreite
    \item bei persönlicher Anwesenheit (außer Relais/Baken)
    \item AF-Stellen nicht mit TK-Netzen verbinden!
    \item BMVIT kann Ausnahmen vorsehen (Technikerprobung: Bandbreite, Ausbildung: Sendeleistung)
  \end{itemize}
}

\card{31}{Wie ist der Amateurfunkverkehr abzuwickeln?}{
  Offene Sprache, nicht verschlüsselt.
  Inhalt:
  \begin{itemize}
    \item Übertragungsversuche
    \item technische/betriebliche Mitteilungen
    \item Bemerkung persönlicher Natur, bildliche Darstellungen, bei denen wegen Belanglosigkeit eine Inanspruchnahme von TK-Diensten nicht verlangt werden kann
    \item Verkehr nur unmittelbar zw. bewilligten AF-Stellen ohne Benutzung anderer TK-Anlagen
  \end{itemize}
}

\card{32}{Definieren Sie den Begriff Not- und Katastrophenfunkverkehr?}{
  \begin{itemize}
    \item Notfunkverkehr: Nachrichtenübermittlung zwischen Funkstelle in Not/beteiligt/Zeuge und einer/mehreren hilfeleistenden Funkstellen.
    \item Notfall: menschliches Leben in Gefahr
    \item Katastrophenfunkverkehr: Nachrichtenübermittlung (nat./int. Hilfeleistung betreffend) zwischen Funkstelle im Katastrophengebiet (geogr. Gebiet, für die Dauer) und Hilfe leistenden Organisationen.
  \end{itemize}
}

\card{33}{Wo können Sie erfahren, unter welchen technischen Parametern (Sendeart, Leistungsstufe, Einschränkungen, etc.) Sie mit Ihrer Lizenzklasse in welchem Frequenzband Amateurfunk betreiben dürfen?}{
  In der \emph{Anlage 2} der \emph{Amateurfunkverordnung} werden die dem Amateurfunk zugewiesenen Frequenzbereiche, der Status, die zulässige Bewilligungsklasse und Leistungsstufe sowie eventuelle Bemerkungen bzw. Einschränkungen definiert.
}

\card{34}{Was ist ein und wozu gibt es ein Funktagebuch?}{
  \begin{itemize} 
    \item Zur Klärung frequenztechnischer Fragen wenn von der FMB verlangt
    \item Auch mit Hilfe von EDV
    \item Bei Notfunkverkehr komplette Nachricht aufzeichnen
    \item 1 Jahr aufbewahren, den Organen des FMB unmittelbar lesbar vorweisen
  \end{itemize}
}

\card{35}{In welchem Umfang ist Mitbenutzung einer Amateurfunkstelle möglich?}{
  \small
  Inhaber der AF-Bewilligung/Stationsverantwortliche (bleibt für Einhaltung der Bestimmungen verantwortlich, muss überwachen) können Personen, die die AF-Prüfung bestanden haben, die Mitbenutzung gestatten. Mitbenützer darf das nur im Umfang
  \begin{itemize}
    \item der Prüfungskategorie des AF-Prüfungszeugnisses
    \item der Bewilligungsklasse / Leistungsstufe der AF-Bewilligung des AF-Stellen Inhabers
    \item Der BMVIT kann zum Zweck der Ausbildung Ausnahmen vorsehen
  \end{itemize}
}

\card{36}{Wer ist für Amtshandlungen nach dem Amateurfunkgesetz zuständig?}{
  \begin{itemize} 
    \item Für die Amtshandlungen zuständig ist das örtliche FMB (entspr. Hauptwohnsitz). 
    \item Bei mehreren FMBs ist einvernehmlich vorgehen. 
    \item Der BMVIT ist zuständig für die Entscheidung über Rechtsmittel gegen Bescheide des FMB, soweit nicht der UVS zuständig ist.
  \end{itemize}
}

\card{37}{Nennen Sie einige Verwaltungsstrafbestimmungen in Bezug auf den Amateurfunk?}{
\scriptsize
  \begin{minipage}{0.48\textwidth}
    \begin{itemize}[leftmargin=10pt,itemsep=0pt]
      \item Senden in AF-Frequenz, aber nicht Bewilligungsklasse
      \item Sendearten nicht in der Bewilligungsklasse
      \item höhere Sendeleistung / Bandbreite*
      \item nicht persönlich anwesend
      \item Verbindung AF-Stellen / TK-Anlagen*
    \end{itemize}
    * \quad sofern Ausnahme nicht vorliegend
  \end{minipage}
  \begin{minipage}{0.5\textwidth}
    \begin{itemize}[leftmargin=10pt,itemsep=0pt]
      \item vorsätzlich Verkehr mit nicht bewilligter Funkstelle
      \item nicht unmittelbarer Verkehr mit bewilligter Funkstelle
      \item Verkehr mit Funkstellen in Ländern, die Einwand erhoben haben
      \item Gestattung von Mitbenutzung durch Personen ohne Prüfung
      \item Mitbenutzung ohne Prüfung
      \item mangelhafte Überwachung der Mitbenutzung (Einhalten der Bestimmungen)
    \end{itemize}
  \end{minipage}
%  \begin{itemize}
%    \item senden in Frequenzbereichen, die nicht dem AF-Dienst zugewiesen sind
%    \item wenn im Verkehr mit anderen Funkstellen Ansehen/Sicherheit/Wirtschaftsinteressen gefährdet werden, gegen öffentliche Ordnung/Sittlichkeit verstoßen wird
%    \item wenn Notrufe gestört/nicht beantwortet werden
%    \item wenn ein anderes oder kein Rufzeichen gesendet wird
%  \end{itemize}
%  \begin{itemize}
%    \item Errichten oder Betreiben einer AF-Stelle ohne AF-Bewilligung
%    \item Verwendung von Daten der Rufzeichenliste für andere Zwecke als AF
%  \end{itemize}
%  Wenn Tatbestand strenger bestrafbar (Gerichte zuständig), keine Verwaltungsübertretung.
}

\card{38}{Was ist eine CEPT-Lizenz? \\ (oder CEPT-Novizen-Lizenz)}{
  \begin{itemize}
    \item Eine AF-Bewilligung oder eine Urkunde, die einen Hinweis darauf enthält, dass sie eine CEPT-Lizenz ist.
    \item Erteilung/Ausstellung: Von der Behörde eines Staates, der die CEPT-Empfehlung T/R61-01 anwendet.
    \item CEPT-Novice-Lizenz: entsprechend ERC/REC~05(06)
  \end{itemize}
}

\card{39}{Was darf ein ausländischer CEPT-Lizenz Inhaber oder CEPT-Novizen-Lizenz in Österreich ohne eigene österreichische Bewilligung?}{Inhaber einer ausländischen CEPT-Lizenz, älter als 14 Jahre, dürfen 3 Monate ab Einreisetag eine AFU-Stelle errichten und betreiben.}

\card{40}{Was bedeutet der Begriff Reziprozität und nennen Sie ein Beispiel?}{
  \begin{itemize}
    \item Begriff aus dem Völkerrecht
    \item Angehörige anderer Staaten werden in Österreich so behandelt, wie Österreicher im anderen Staat.
  \end{itemize}
  Beispiel:
  \begin{itemize}
    \item Ausländern wird Bewilligung nur erteilt, wenn Österreichern in diesem Staat auch das Errichten und Betreiben einer AFU-Stelle gestattet ist
  \end{itemize}
}

\card{41}{Nennen Sie die Bewilligungsklassen und wozu berechtigen diese?}{
  •  3 Klassen (1, 3 und 4)
  •  international Klasse 1 (CEPT AFU-Bewilligung), Klasse 4 (CEPT NOVICE-Lizenz), Klasse 3 national
  •  Klasse 1 darf alle Frequenzbereiche und Sendearten (Einschränkungen beachten) nutzen.
  •  Klasse 3 darf nur \SI{2}{\metre} und \SI{70}{\centi\metre} und bestimmte Sendearten (Einschränkungen beachten) nutzen. Keine Selbstbauanlagen, nur kommerziell gefertigte, nicht veränderte, Leistungsstufe A
  •  Klasse 4: \SI{2}{\metre} und \SI{70}{\centi\metre}, 4 KW-Bereiche, sonst wie Klasse 3
  •  Mitbenutzung von Klubfunkstellen ist gestattet.
}

\card{42}{Welche Leistungsstufen kennen Sie und nennen Sie deren Merkmale?}{
  \begin{center}
    \vspace{5pt}
    \begin{tabular}{cl}
      A & \SI{100}{\watt}att max \\
      B & \SI{200}{\watt}att max \\
      C & \SI{400}{\watt}att max \\
      D & \SI{1000}{\watt}att max
    \end{tabular}
  \end{center}
  Überschreitung der Grenzwerte um \SI{20}{\percent} tolerabel.
}

\card{43}{Unter welchen Voraussetzungen kann eine Amateurfunkbewilligung für die Leistungsstufe C erteilt werden?}{
  Wenn am genannten Standort seit mindestens 1 Jahr eine AF-Stelle mit ,,Leistungsstufe B`` störungsfrei betrieben wurde.
}

\card{44}{Unter welchen Voraussetzungen kann eine Amateurfunkbewilligung für die Leistungsstufe D erteilt werden?}{
  Bewilligung für ,,Leistungsstufe D``:
  \begin{itemize}
    \item nur AFU-Vereinen und im öffentlichen Interesse tätigen Organisationen
    \item kann von Ergebnissen eines Probebetriebs (6~Monate) abhängig gemacht werden
  \end{itemize}
}

\card{45}{Was bedeutet der Status eines Funkdienstes (Primär, Primär/Exklusiv(Pex), Sekundär, ISM)?}{
 \footnotesize
  \begin{description}
    \item[Pex] primärer Funkdienst (exklusiv für Amateurfunk)
    \item[P] primärer Funkdienst (Mitbenutzung durch andere FD)
    \item[S] sekundärer Funkdienst (primärer Funkdienst hat Vorrang),
      \begin{itemize}[leftmargin=0pt,itemsep=0pt]
        \item dürfen keine Störungen bei primären verursachen
        \item können keinen Schutz gegen Störungen von primären verlangen
        \item können Schutz gegen Störungen von sekundären verlangen
      \end{itemize}
    \item[ISM] Hochfrequenzbereich für industrielle, wissenschaftliche, medizinische Anwendung
  \end{description}
}

\card{46}{Ist die Verwendung der Betriebsart Telegraphie an eine bestimmte Voraussetzungen gebunden?}{
  \begin{itemize}
    \item Nein, Verwendung aller Betriebsarten bei Klasse 1, 4 und Klasse 3 zulässig.
    \item Einige Länder außerhalb der CEPT verlangen für die Erteilung einer Gastlizenz unter \SI{30}{\mega\Hz} eine Telegrafieprüfung.
  \end{itemize}
}

\card{47}{Wann wird eine schädliche Störung als solche behandelt?}{
  \small
  \begin{itemize}
    \item Wenn die Funkanlagen entsprechend Bewilligungen errichtet sind und die gestörte Empfangsanlage vorschriftsmäßig betrieben wird.
    \item Nicht, wenn Störung durch andere, ordnungsgemäß errichtete/betriebene AF-Stellen verursacht wird.
    \item Nicht in ISM Bändern.
    \item Bei Störung durch TK-Einrichtungen kann die FMB (wenn alle beteiligten Anlagen den Vorschriften entsprechen) unter Abwägung des wirtschaftlichen Aufwands technische/betriebliche Maßnahmen zur Behebung anordnen.
  \end{itemize}
}

\card{48}{Was gilt für einen Amateurfunkbetrieb auf Schiffen und in Flugzeugen?}{Es entscheidet der Pilot / der Kapitän, ob AFU durchgeführt werden darf.}

\card{49}{Welche Aussendungen dürfen von einer Amateurfunkstelle empfangen werden?}{Mit einer Empfangsanlage dürfen empfangen werden: \begin{itemize} \item Aussendungen anderer AF-Stellen \item Rundfunk \item Nachrichten an alle, sofern diese für den
Gebrauch durch die Öffentlichkeit bestimmt \item Not/Katastrophenverkehr\end{itemize}}

\card{50}{Was darf der Nachrichteninhalt einer Amateurfunkaussendung sein?}{\small{Offene Sprache (Abkürzungen, Zeichen, Esperanto, Latein), Nachricht muss verständlich bleiben, nur normierte Übertragungsverfahren:
•  Morsealphabet, Telegraphiealphabet Nr. 2, AMTOR/PACTOR, ITU-R-Empf. M476/M625, HELL-System, (Fernsehen AM), im ITU-R-Report 624 beschriebene, (Packet Radio) AX-25
Protokoll (alle Übertragungsgeschwindigkeiten), DVBT (EN300744), DVBS (EN300421)
•  Verwendung anderer Verfahren: Rufzeichen in offener Sprache/normiert, Inhalt 3 Wochen reproduzierbar dokumentiert
•  Aussendung von reinem Träger nur zu Mess/Testzwecken}}

\card{51}{Gibt es eine Möglichkeit, dass ein Funkamateur, der die Prüfungskategorie 3 erfolgreich abgelegt hat, auf anderen Frequenzen als dem 2m / \SI{70}{\centi\metre}-Band Funkverkehr haben darf?}{\begin{itemize} \item Klubfunkstelle mit Bewilligungsklasse 1 \item darf auf allen, dem AF zugewiesenen Frequenzen \item von Personen mit Klasse 3 und 4 \item zum Zweck der Ausbildung \item unter Überwachung eines Inhabers (Klasse 1) \item mitbenutzt werden\end{itemize}}

\card{52}{Wer darf eine Relaisfunkstelle errichten / betreiben / benutzen und wie ist deren Rufzeichen auszusenden?}{• Bewilligung für eine Relaisfunkstelle wird nur einem
Amateurfunkverein/einer im öffentlichen Interesse tätigen Organisation erteilt,
• wenn der Einsatz der Betriebsfrequenzen (hinsichtl. zugeteilter Frequ.) störungsfrei erfolgen kann.
• eigenes Bewilligungsverfahren
• Benutzung ist allen AF-Stellen zu gestatten
• Bei Sprachübertragungsrelais: Aussendung des Rufzeichens in Sprache oder mit 60-100 Zeichen pro Minute in Telegraphie.
• Bei anderen: Aussendung des Rufzeichens in der jeweiligen Sendeart.}

\card{53}{Was haben Sie zu tun, wenn Sie Funkverkehr mit einer nicht bewilligten Amateurfunkstelle haben und mit wem dürfen Sie keinen Amateurfunkverkehr haben?}{\begin{itemize} \item Nicht bewilligte AF-Stelle: Verkehr abbrechen. \item Alles unterlassen, was das Ansehen, die Sicherheit, die Wirtschaftsinteressen gefährdet, was gegen die öffentliche Ordnung oder Sittlichkeit verstößt. \item Unzulässiger Verkehr: Mit AFU-Stellen in Ländern, die Einwand erhoben haben \item Kundmachung durch BMVIT im Bgbl.\end{itemize}}

\card{54}{Welche besonderen Aufgaben hat die ITU in Bezug auf Funkdienste und welche Ausschüsse sind dafür zuständig?}{\small{Aufgaben:
•  Zuweisung der Frequenzen
•  Verhinderung gegenseitiger Störungen
•  Verbesserung der Ausnutzung der Bänder
•  Förderung der Zusammenarbeit der Hilfsdienste zur Erhaltung menschlichen Lebens \\
Ausschüsse:
•  Radiocommunication Bureau: zugeteilte Frequenzen (Länder) registrieren, Anerkennung sichern, Beratung bei Störungen
•  Radiocommunication Sector: Studien über technische und betriebliche Fragen, Mitglieder beraten
•  Telecommunication Sector: Beratung, Studien: Technisches, Betriebs/Gebührenfragen (so billig wie möglich, trotzdem dotiert)}}

\card{55}{Was bedeutet missbräuchliche Verwendung von Funkanlagen?}{\small{• Nachrichtenübermittlung, die öffentliche Ordnung und Sicherheit gefährdet, gegen Gesetze verstößt
• Belästigung oder Verängstigung anderer
• Verletzung der geltenden Geheimhaltungspflicht
• Nachrichtenübermittlung, die nicht dem bewilligten Zweck der FA entspricht
• Inhaber (nicht Zugangsanbieter) müssen zumutbare Maßnahmen zur Vermeidung von Missbrauch treffen
• bewilligter Zweck, Standort / im Einsatzgebiet
• bewilligte Frequenzen, Rufzeichen
• nicht zugelassene FA / TK-Einrichtungen dürfen nicht mit einem öffentl. Komm.netz verbunden/betrieben werden}}

\card{56}{Was hat der Inhaber einer Amateurfunkstelle zu tun, wenn er nicht bei dieser Stelle anwesend ist?}{Der Inhaber einer Amateurfunkstelle hat \emph{geeignete Maßnahmen} zu treffen, die Inbetriebsetzung seiner Funkstelle durch \emph{unbefugte Personen} auszuschließen. Aussendungen dürfen nur durchgeführt werden, wenn der Inhaber einer Amateurfunkbewilligung oder der Mitbenützer der Amateurfunkstelle während der gesamten Dauer der Aussendung \emph{persönlich} an der Amateurfunkstelle \emph{anwesend} ist, \emph{außer} es handelt sich um eine Relaisfunkstelle oder einen Bakensender.}

\card{57}{Welche Bestimmungen sind beim Betrieb einer Amateurfunkstelle im Ausland zu beachten?}{Die Bestimmungen des Gastlandes.}

\card{58}{Unter welchen Voraussetzungen darf der Inhaber einer Amateurfunkbewilligung der Bewilligungsklasse 3 im Ausland Amateurfunkbetrieb durchführen?}{Er muss eine Gastlizenz beantragen.}

\card{59}{Wozu berechtigt eine Amateurfunkbewilligung der Klasse 4?}{\begin{itemize}
  \item Sendebetrieb im 160, 80, 15, 10, 2m und \SI{70}{\centi\metre} Band
  \item Leistungsstufe A (max. \SI{100}{\watt})
  \item nur kommerzielle, unmodifizierte Geräte verwenden
\end{itemize}}

\card{60}{Aufgrund welcher internationalen Regelung dürfen Funkamateure aus bestimmten Ländern auch ohne individuelle Gastzulassung vorübergehend in Österreich Amateurfunk ausüben?}{\small Die Empfehlung T/R~61-01 regelt die Gültigkeit von Amateurfunkbewilligungen für die CEPT-Mitgliedsländer. Mit der Bewilligungsklasse 1 (= CEPT-Zertifikat für Funkamateure) darf in den CEPT-Mitgliedsländern auf die Dauer von 3~Monaten ohne Gastlizenz Amateurfunkbetrieb unter Beachtung nationaler Bestimmungen durchgeführt werden.

\begin{description}
  \item[T/R~61-02] Umfang und Inhalt der Amateurfunkprüfung zur Erlangung eines CEPT-Zertifikats
  \item[ERC/REC~05/06] Selbiges zur Erlangung eines CEPT-Novice-Zertifikates
\end{description}
}

\card{61}{Unter welchen Voraussetzungen ist die Verbindung von Amateurfunkstellen mittels Internettechnologie zulässig?}{Folgende Voraussetzungen müssen erfüllt sein:
\begin{itemize}
  \item zwei oder mehrere Amateurfunkstellen werden verbunden
  \item Erprobung neuer Übertragungstechnologien
  \item kein gewerblich-wirtschaftliche Zwecke
  \item kein reiner Internetzugang
\end{itemize}}


\def\chap{Betrieb und Fertigkeiten \class{1,4}}

\card{01}{Wie eröffnen Sie einen Funkverkehr in Phonie, wie in Telegraphie?}{
  \small
  \begin{enumerate}
    \item Reinhören, ob Frequenz frei ist
    \item
      Phonie: ,,is this frequency in use?{}``,
      CW: ,,QRL?{}``
    \item
      Phonie: ,,this frequency in use!{}`` $\rightarrow$ ,,sorry!{}``, \\
      CW: ,,QRL`` $\rightarrow$ ,,SRI``
    \item Wenn frei, 3~mal \\
      Phonie: ,,CQ, CQ, CQ - this is call, call`` \\
      CW: ,,CQ CQ CQ DE call``
  \end{enumerate}
  Beachte die \emph{tote Zone}.
  Contest: ,,CQ Contest, this is …`` (3~mal) ,,CQ Test de …`` (1-3~mal)
}

\card{02}{Was ist das gebräuchliche Minimum einer Amateurfunkverbindung?}{
  \begin{itemize}
    \item Rufzeichen
    \item Rapport (RS bzw. RST)
    \item Vorname
    \item Standort (QTH)
    \item (optional) Stationsbeschreibung
  \end{itemize}
}

\card{03a}{Welche Bedeutung haben die Q-Gruppen im allgemeinen?
  \begin{center}
    QRM \quad QSO \quad QSY \quad QSL \quad QRP \quad QTR
  \end{center}
}{\begin{description}
  \item[QRM] ich werde gestört (Fremdstörungen),
  \item[QSO] ich habe Verbindung mit \dots
  \item[QSY] wechseln Sie auf die Frequenz \dots{} kHz
  \item[QSL] ich werde eine Empfangsbestätigung (QSL-Karte) geben
  \item[QRP] vermindern Sie die Sendeleistung
  \item[QTR] es ist \dots Uhr GMT (UTC)
 \end{description}
}

\card{03b}{Welche Bedeutung haben die Q-Gruppen im allgemeinen?
  \begin{center}
    QRS \quad QRX \quad QRO \quad QRV \quad QSP \quad QRG
  \end{center}
}{\begin{description}
  \item[QRS] geben Sie langsamer
  \item[QRX] ich werde Sie um \dots~Uhr auf \dots~kHz wieder rufen
  \item[QRO] erhöhen Sie Ihre Sendeleistung
  \item[QRV] ich bin betriebsbereit
  \item[QSP] ich werde an \dots\ weiterübermitteln,
  \item[QRG] ihre genaue Frequenz ist \dots~kHz
\end{description}}

\card{03c}{Welche Bedeutung haben die Q-Gruppen im allgemeinen?
  \begin{center}
    QRT \quad QRU \quad QRN \quad QRB \quad QTH \quad QSB
  \end{center}
}{\begin{description}\itemsep0pt
  \item[QRT] stellen Sie die Aussendung(en) ein
  \item[QRU] ich habe nichts für Sie vorliegen
  \item[QRN] ich habe atmosphärische Störungen (1 = keine, 5 = sehr stark),
  \item[QRB] die Entfernung zwischen unseren beiden Stationen ist \dots\ km
  \item[QTH] mein Standort ist \dots
  \item[QSB] Ihre Zeichen weisen Fading auf (= die Empfangsfeldstärke schwankt).
 \end{description}
}

\card{04}{Sie wollen, dass Ihre Gegenstation die Sendeleistung vermindert. Welche Q-Gruppe verwenden Sie?}{Die Q-Gruppe QRP}
\card{05}{Was bedeuten die Hinweise \\ ,,5 UP`` bzw. ,,10 DOWN``?}{Der eigene Sender sollte um \SI{5}{\kilo\Hz} nach oben (\SI{10}{\kilo\Hz} nach unten) verstellt werden, um dort, wo die Gegenstation hört, zu senden. Der Empfänger bleibt auf der Frequenz, auf der die Gegenstation sendet. (Split-Betrieb).}
\card{06}{Sie wollen in einen bestehenden Funkverkehr einsteigen. Wie führen Sie das durch?}{
  \begin{itemize}
    \item Funkverkehr beobachten
    \item in einer Sendepause sein Rufzeichen senden (KW) oder mit ,,OE1ABC bittet um Aufnahme`` (UKW-FM) auf sich aufmerksam machen
    \item mögliche Rückmeldungen sind ,,bitte warten`` (,,please standby``) oder ,,your call please`` oder ,,no breaker please`` (sehr unhöflich – nur für Ausnahmefälle!)
  \end{itemize}
}
\card{07}{Welche betrieblichen Auswirkungen haben die besonderen Ausbreitungsbedingungen auf Kurzwelle?}{
  \small
  2 typische Ausbreitungsformen auf KW:
  \begin{itemize}
    \item Bodenwellen (Erdbodens/Meeres): die Reichweite nimmt mit steigender Frequenz rasch ab \& ist abhängig von den Bodeneigenschaften
    \item Raumwellen (Reflexionen an der Ionosphäre): es kann weltweiter Funkverkehr bei geeigneter Frequenzwahl durchgeführt werden $\Rightarrow$ Tote Zone (Diagramm siehe Skript)
  \end{itemize}

  Strahlungsdiagramme von Kurzwellenantennen: \\
  \begin{itemize*}
    \item Horizontaler Dipol
    \item Vert. Dipol
    \item 3-Elem. hor. Yagi
  \end{itemize*}
}
\card{08}{Welche betriebliche Auswirkung hat die Bodenwellen-Ausbreitung?}{
  \small
  Eine Bodenwelle ist die Ausbreitung eines elektromagnetischen Feldes entlang der Erdoberfläche.
  \begin{itemize}
    \item bei zunehmender Entfernung zur Antenne kommt es zu einer Dämpfung
    \item die Bodenleitfähigkeit (Salzgehalt) spielt eine maßgebliche Rolle
    \item Reichweite abhängig von verwendeter Sendeleistung, Art-und Wirkungsgrad der Antenne, Arbeitsfrequenz bzw. Band (bei steigender Frequenz nimmt die Reichweite ab).
  \end{itemize}
}
\card{09}{Welche betriebliche Auswirkung hat die Raumwellen-Ausbreitung, in welchem Frequenzbereich ist sie von Bedeutung?}{
  \small
  \begin{itemize}
    \item Raumwelle ist die Ausbreitung eines elektromagnetischen Feldes / von Funkwellen über eine (oder mehrere) Reflexionen an der Ionosphäre (ermöglichen im KW-Bereich den weltweiten Funkverkehr)
    \item maßgebliche Ausbreitungsform im Kurzwellenbereich (\SIrange{3}{30}{\mega\Hz})
    \item auch für die Mittel- und Grenzwellenausbreitung (\SIrange{1,5}{3,0}{\mega\Hz}) bei Nacht und unter bestimmten Voraussetzungen bis in den \SI{2}{\metre}-Band-Bereich von Bedeutung
  \end{itemize}
}
\card{10}{Welche betriebliche Bedeutung hat die kritische Frequenz?}{
  \small
  Die kritische Frequenz ist die obere Grenzfrequenz, bei der, bei sog. ,,Senkrechtlotung`` noch Reflexion an der Ionosphäre auftreten (als $f_0$ bezeichnet)
  \begin{itemize}
    \item Abhängig von Dichte der freien Elektronen in der Ionosphäre
    \item Funkwellen mit Frequenzen, die größer als die kritische Frequenz sind, werden in der Ionosphäre nicht mehr reflektiert
    \item Maximum Usable Frequency MUF $= f_0 / \sin(\alpha)$ (abhängig von der Zeit des Zielortes / Sonnenstand)
  \end{itemize}
}
\card{11}{Welche betriebliche Bedeutung haben die Begriffe ,,MUF`` und ,,LUF``?}{
  \small
  \begin{description}
    \item[MUF] ,,maximum usable frequency``: höchste noch nutzbare Frequenz auf einer vorgegebenen Übertragungsstrecke. Abhängig von: kritischer Frequenz \& Abstrahlwinkel der Antenne
    \item[LUF] ,,lowest usable frequency``: die niedrigste noch nutzbare Frequenz, bei Raumwellenausbreitung, bei der die Feldstärke am Empfangsort ausreichend stark ist
  \end{description}
  Bei Über- bzw. Unterschreitung: keine Signalreflexion (Diagramm siehe Skriptum)
}
\card{12}{Was versteht man unter Fading auf Kurzwelle, wodurch entsteht Fading und wie reagieren Sie, um den Funkverkehr aufrecht zu erhalten?}{
  Schwanken der Empfangsfeldstäre (QSB = die Empfangsfeldstärke schwankt)
  \begin{itemize}
    \item können schnell oder langsam erfolgen
    \item Ursachen: Mehrwegeausbreitung oder Drehung der Polarisationsebene
  \end{itemize}
}
\card{13}{Ausbreitung von Funkwellen -- Ausbreitungsmerkmale in den verschiedenen Amateurfunk Frequenzbereichen?}{
  \small
  Ausbreitung mit Lichtgeschwindigkeit als ,,Bodenwelle``, ,,direkte Wellen`` oder ,,Raumwellen``
  \begin{description}
    \item[unter \SI{30}{\mega\Hz}] primär Raumwellenausbreitung
    \item[unter \SI{30}{\mega\Hz}] es tritt auch die Bodenwelle auf und reicht im \SI{160}{\metre}-Band 100--200~km, nimmt aber mit zunehmender Frequenz rasch ab.
    \item[ab \SI{30}{\mega\Hz}] die Funkwellen nehmen zunehmend ,,optisches Verhalten`` an, d.h. ihre Ausbreitung erfolgt gradlinig. Es treten keine Bodenwellen mehr auf. Primär ,,direkte Wellen``
  \end{description}
}
\card{14}{Welchen Einfluß hat die Ionosphäre auf die Ausbreitung von Funkwellen über \SI{30}{\mega\Hz}?}{
  Auf Frequenzen über \SI{30}{\mega\Hz} hat die Ionosphäre im Allgemeinen nur mehr eine ablenkende Wirkung, es tritt jedoch keine Reflexion mehr auf. Zudem beobachtet man eine Polarisationsdrehung (,,Faradaydrehung``). Durch sporadische E-Schichten kann dennoch kurzzeitig bis in die 6~m-Bereiche Reflexion auftreten.
}
\card{15}{Erklären Sie die Begriffe Fresnelzone, Geländeschnitt}{
  \begin{description}
  \item[Geländeschnitt] graphische Darstellung des Profils der Erdoberfläche zw. Sende- und Empfangsstandort
  \item[Fresnelzone] ellipsenförmige Zone zwischen Empfänger und Sender. In dieser Zone sollten keine Hindernisse sein, sonst kommt es zur Streckendämpfung.
  \end{description}
}
\card{16}{Was ist die tote Zone? Was ist ein Skip?}{
  \begin{description}
    \item[tote Zone] Bereich zwischen der nutzbaren Reichweite der Bodenwellen und dem ersten Auftreten der Raumwelle.
    \item[Skip] Auftreffen der Raumwelle auf der Erde nach der Reflexion an der Ionosphäre.
    \item[Skipdistanz] Entfernung zwischen Senderstandort und dem Skip.
  \end{description}
}
\card{17}{Wovon hängt die maximal erzielbare Reichweite auf Kurzwelle ab?}{
  \small
  \begin{itemize}
    \item Maximale Reichweite (DX) wird immer durch Raumwellen erzielt.
    \item Reichweite ist somit vom Zustand der Ionosphäre und vom Abstrahlwinkel der Antenne abhängig.
    \item Verbesserte Reflexionsergebnisse an der der Ionosphäre durch Antennen mit geringem Erhebungswinkel der Strahlungskeule.
    \item Reichweite ist auch abhängig von den elektrischen Eigenschaften an den Bodenreflexionspunkten und nur wenig von der Sendeleistung abhängig.
  \end{itemize}
}
\card{18}{Was verstehen Sie unter kurzem Weg? Was unter langem Weg?}{
  \begin{itemize}
    \item Die kürzeste Entfernung zwischen 2 Punkten A und B auf der Erde ist entlang eines ,,Großkreises`` (Meridians)
    \item Es gibt 2 Möglichkeiten um das Ziel zu erreichen:  kurz oder lang
    \item je nach Ausbreitungsbedingungen und Betriebsfrequenz ist einer der beiden Wege bevorzugt oder nur auf einem ist Funkverkehr möglich.
  \end{itemize}
}
\card{19}{Was verstehen Sie unter dem Dämmerungseffekt?}{
  Der Dämmerungseffekt sind unübliche Ausbreitungsbedingungen, bei denen die Feldstärken stark ansteigen, um nach Ende der Dämmerung teilweise schlagartig zusammenzubrechen.
  \begin{itemize}
    \item tritt während dem Sonnenauf- und Sonnenuntergangs auf
    \item Ursache: mitunter die sich rasch ändernden Ionisiationsverhältnisse der E- und D-Schicht
  \end{itemize}
}
\card{20}{Was verstehen Sie unter der ,,Grey-Line``, welche Besonderheiten in der Funkausbreitung können auftreten?}{
  \small
  \item
  Unter der Grey-Line versteht man die Dämmerungszone, in der es zu unüblicher Funkausbreitung mit häufig extremen Reichweiten bei hohen Signalfeldstärken kommen kann.

  \item
  Durch die sich ändernden Dichteverhältnisse der Elektronenverteilung in der D- und E-Schicht kann es bei relativ steilem Einfall von Funkstrahlen zu sehr flachen Austrittswinkel kommen. Daher es können sehr große Entfernungen (teilweise sogar ohne Erdreflexionen) überbrückt werden. Die Empfangssignalfeldstärke ist unüblich hoch.
}
\card{21}{Beschreiben Sie den Aufbau der Ionosphäre und welche betriebliche Konsequenzen ergeben sich daraus?}{
  \footnotesize
  \begin{minipage}{0.49\textwidth}
    \begin{itemize}
      \item mehreren Schichten erhöhter Ionen-/Elektronenkonzentration
      \item für Funkausbreitung ist die Elektronenkonzentration wichtig
      \item am Tag treten 4 Schichten auf: D-, E-, F1-, F2-Schicht.
      \item Abenddämmerung: D-, E-, F1-Schicht werden weniger
    \end{itemize}
  \end{minipage}
  \begin{minipage}{0.5\textwidth}
    \begin{itemize}
      \item in der Nacht: nur eine F-Schicht
      \item Tageszeit unabhängig: sporadische E-Schichten
      \item für den KW-Bereich ist die D-Schicht zu wenig für die Reflexion ionisiert $\Rightarrow$ Dämpfung
      \item die ,,Nachtfrequenzen`` liegen deutlich tiefer als die ,,Tagfrequenzen``
    \end{itemize}
  \end{minipage}
}
\card{22}{Wie verhalten sich die Ionosphärenschichten im Tagesverlauf bzw. im Jahresverlauf?}{
  \footnotesize
  Tagesverlauf:
  \begin{itemize}
    \item Dämmerungsbeginn: zuerst bildet sich D, dann E
    \item Tag: E kann bereits reflektieren, F spaltet sich in F1 und F2  auf (für die Raumwellenausbreitung maßgeblich)
    \item bei Sonnenhöchststand (Mittag) ist das Maximum an freien Elektronen erreicht.
    \item abnehmende Einstrahlung: Elektronendichte nimmt ab
  \end{itemize}

  Jahresverlauf:
  \begin{itemize}
    \item D und E kaum beeinflusst
    \item F starke Abhängigkeit, insbesondere was die Schichthöhe und Elektronendichte (Sommer = Maximum) betrifft.
  \end{itemize}
}
\card{23}{Welchen Einfluss hat die geographische Breite auf die Kurzwellenausbreitung?}{
  Die geographische Breite hat primär einen Einfluss auf den Einfallswinkel der Sonnenstrahlung --> die Ionisierung ist im Bereich des Äquators am stärksten und im Bereich der Pole am schwächsten.
  \begin{itemize}
    \item kurze Dämmerung am Äquator
    \item Polarnacht auf den Polen
  \end{itemize}
}
\card{24}{Was versteht man unter Sonnenaktivität, unter der Sonnenfleckenrelativzahl, unter dem ,,Solar-Flux``? Welchen Einfluss hat sie auf die Kurzwellenausbreitung?}{
  \small
  Sonnenaktivität ist die Gesamtzahl der auf der Sonne stattfindenden Vorgänge.
  \begin{description}
    \item[SFRZ] Häufigkeit der Sonnenflecken
    \item[Sonnenstrahlung] Solar Flux bewirkt die Ionisation: der Materiestrom wirkt sich vorrangig auf das Erdmagnetfeld und damit nur indirekt auf den Funkverkehr aus. Es kommt fallweise zu gewaltigen Energieausbrüchen auf der Sonne, die sich in erhöhter Strahlung und erhöhtem Teilchenstrom auswirken $\Rightarrow$ abrupter Anstieg von Ionisation
  \end{description}
}
\card{25}{Welchen Zyklen unterliegen die Ausbreitungsbedingungen auf Kurzwelle?}{
  Ausbreitungsbedingungen unter dem Einfluss von der Sonne und der Eigenbewegung der Sonne.
  4 Zyklen:
  \begin{itemize}
    \item Tagesgang (24h, Ursache= Erdrotation)
    \item 27-Tagesrhytmus (mittlere Umlaufzeit Sonne)
    \item Jahresgang (Jahreszeiten, Neigung der Erdachse)
    \item Sonnenfleckenzyklus (dauert im Schnitt 11,2 Jahre = 11-Jahreszyklus)
  \end{itemize}
}
\card{26}{Beschreiben Sie das charakteristische Ausbreitungsverhalten in den dem Amateurfunkdienst zugewiesenen Frequenzbändern unter \SI{30}{\mega\Hz}?}{
  \footnotesize
  BW = Bodenwellenausbreitung, RW = Raumwellenausbreitung
  \begin{minipage}{0.5\textwidth}
    \centerline{160-m-Band}
      \begin{description}
        \item[Tag] BW
        \item[Dämmerung] Raum- und BW
        \item[Nacht] RW
      \end{description}

    \centerline{80-m-Band}
      \begin{description}
        \item[Tag] BW
        \item[Dämmerung] DX-Reichweiten möglich
        \item[Nacht] RW
      \end{description}
  \end{minipage}
  \begin{minipage}{0.48\textwidth}
    \centerline{40-m-Band}
      \begin{description}
        \item[Tag] BW (zusätzlich RW)
        \item[Dämmerung] ausgeprägter Dämmerungseffekt
        \item[Nacht] RW (Schattenzone)
      \end{description}

    \centerline{30-m-Band}
    24~h für weltweiten Funkverkehr offen (mehr siehe Skript)
  \end{minipage}
}
\card{27}{Was versteht man unter einem Mögel-Dellinger-Effekt und welche betriebliche Auswirkungen hat er?}{Durch gewaltige und plötzliche Energieausbrüche auf der Sonne kommt es zu verstärkten Strahlungsausbrüchen (von gewaltigen Materieausstößen begleitet). Nach 8min (Lichtgeschwindigkeit) erreicht die Strahlung die Erde -keine Vorwarnung möglich (Sudden Ionospheric Dusturbances). Durch die erhöhte Ionisation steigt auch in der D-Schicht die Dämpfung deutlich an. Schließlich kann sie so stark werden, dass der Funkverkehr zusammenbricht - bis zu einigen Stunden.}
\card{28}{Welche Auswirkungen haben Polarlicht-Erscheinungen auf die Kurzwellenausbreitung?}{Es kann zu ausgeprägten Reflexionserscheinungen bis in den hohen UKW-Bereich hinein kommen. Es kommt zu einem ausgeprägten, schnellen Fading und Nachhalleffekt - die Signale sind selbst bei hoher Empfangsfeldstärke fast unlesbar. Ausbreitungswege, die durch diese Zonen führen, werden stark beeinflusst.}
\card{29}{Welche Faktoren können den Funkbetrieb auf Kurzwelle beeinflussen?}{
  \small
  \begin{itemize}
    \item Signal-Rauschabstand: für Sprechfunk (SSB) ein S/N-Abstand von \SI{10}{\dB} erforderlich
    \item Signal-Störabstand bei natürlichen Störquellen: Gewitter, statische Entladung
    \item Signal-Störabstand bei nicht natürlichen Störquellen: Funken (z.B. nicht entstörte Maschinen)
    \item Splattern
    \item Anomalien in der Funkausbreitung (zB. Fading, Nachhalleffekte)
  \end{itemize}
}
\card{30}{Wie wirkt sich die Tageszeit auf die Ausbreitung in den Kurzwellenbändern bis \SI{40}{\metre} aus? (\SI{160}{\metre}/\SI{80}{\metre}-/\SI{40}{\metre}-Band)}{
  Das 160-m-, 80-m-Band, gelegentlich auch das 40-m-Band sind aufgrund der D-Schicht-Dämpfung während des Tages nur für Bodenwellenausbreitung nutzbar. Ab Beginn der Abend-Dämmerung und während der Nacht ist Raumwellenausbreitung gegeben, solange die LUP nicht unterschritten wird.
}
\card{31}{Was verstehen Sie unter ,,Sporadic E-Verbindungen``?}{
  Die Funkverbindungen, über Raumwellen, die durch Reflexionen an sporadischen E-Schichten ermöglicht werden. Sie treten selten auf Frequenzbereichen unter \SI{20}{\mega\Hz} auf und sind eine typische Erscheinungsform auf dem \SI{10}{\metre}- und dem 6m-Band.
}
\card{32}{Was verstehen Sie unter ,,Short-Skips``?}{
  Ausbreitungsbedingungen, bei denen Funkverkehr in die sonst tote Zone hinein möglich ist, ohne dass die gesamte tote Zone erreicht werde kann. Die Ursache können sporadische E-Schichten sein.
}
\card{33}{Was verstehen Sie unter einem Notverkehr, wie wird er angekündigt?}{
  \small
  Funkverkehr, der der Rettung menschlichen Lebens und/oder Güter hohen Werts dient
  \begin{itemize}
    \item jeder andere Funkbetrieb ist einzustellen
    \item angekündigt durch Notzeichen {\footnotesize (Mayday bzw. SOS$\times 3$)}
    \item in Not befindliche Station ist immer Leitfunkstelle
    \item Mayday Relais: Hinweis auf die Übermittlung eines Notrufes/Notmeldung (wie Notruf selbst zu handhaben)
    \item Silence Mayday: Aufforderung zur Betriebseinstellung an andere Funkstellen
  \end{itemize}
}
\card{34}{Sie empfangen einen Notruf – woran erkennen Sie diesen und wie haben Sie sich zu verhalten?}{
  \footnotesize
  \begin{enumerate}[itemsep=0pt]
    \item erkennt man am Notzeichen
    \item Funkverkehr sofort einstellen
    \item Frequenz beobachten
    \item Wenn keine andere Station antwortet $\rightarrow$ antworten!
    \item Notverkehr mitschreiben
    \item nach Art der benötigten Hilfe fragen
    \item Alarmierung von Rettungskräften und der nächstgelegene Dienststelle der Funküberwachung
    \item Wenn die notrufende Station nicht antwortet und den Notruf fortsetzt, dann auf anderen Frequenzen mit Mayday Relay auf den Notruf aufmerksam machen
  \end{enumerate}
}
\card{35}{Auf welchen Bändern könnten Sie einen Notruf empfangen?}{
  \begin{itemize}
    \item Ein Notruf kann auf jeder Frequenz abgesetzt werden
    \item die Wahrscheinlichkeit ist auf den sog. ,,gemischten`` Bändern (werden auch von anderen Funkdiensten genutzt) am größten
  \end{itemize}
}
\card{36}{Welche Sendearten sind im Kurzwellenbereich zulässig?}{
  Auf Kurzwelle, d.h. im Frequenzbereich zwischen \SIrange{3}{30}{\mega\Hz} sind gemäß VO-Funk alle Sendearten zulässig, die eine Bandbreite von höchstens \SI{7}{\kilo\Hz} haben.

  Für den Amateurfunkdienst: Erweiterung der Regelung auf das \SI{160}{\metre}-Band und im Bereich über \SI{29}{\mega\Hz} ist auch die Sendeart ,,Schmalband-FM`` zugelassen. In den höherfrequenten Bändern können höhere Bandbreiten verwendet werden
}
\card{37}{Müssen Sie ein Funktagebuch führen und welche Angaben muss es enthalten?}{
  Nur auf Anordnung der Fernmeldebehörde für einen festgelegten Zeitraum.
  Funktagebuch (,,Logbuch``): Aufzeichnungen der wesentlichen betrieblichen Daten eines Funkverkehrs

  Wesentliche Daten:
  \begin{itemize*}
    \item Datum
    \item Uhrzeit (Beginn/Ende)
    \item Rufzeichen der Gegenstation
    \item Betriebsart
    \item Sendefrequenz
    \item fortlaufende Nummerierung
    \item Unterschrift
    \item auch elektronisch möglich
  \end{itemize*}
}
\card{38}{Was verstehen Sie im Telegraphiebetrieb unter ,,BK-Verkehr``?}{
  eine Betriebstechnik, bei der zwischen den eigenen Aussendungen, bei Telegraphie sogar zwischen den ausgesendeten Punkten oder Strichen, empfangen werden kann.

  Der Funkverkehr kann daher mit der Betriebsabkürzung BK sofort unterbrochen werden und damit sehr flüssig abgewickelt werden. Er setzt aber die erforderliche technische Ausrüstung voraus.
}
\card{39}{Was verstehen Sie unter UTC (GMT) -- Zusammenhang zu Lokalzeit, Sommerzeit}{
  UTC (Universal Time Coordinated) ist die international koordinierte Weltzeit bezogen auf den Null-Meridian. Ist wichtig für die Vereinbarung von Funkkontakten weltweit. Während der Sommerzeit erhöht sich der Unterschied zwischen UTC und Lokalzeit um 1 Stunde. Zum Beispiel für Österreich
  \begin{itemize}
    \item 13:00 Lokalzeit = 12:00~UTC
    \item 15:00 Sommerzeit = 13:00~UTC
  \end{itemize}
}
\card{40}{Nennen Sie die konkreten Frequenzbereiche, die dem Amateurfunkdienst in den jeweiligen Frequenzbändern zugewiesen sind (5 Beispiele)}{
  \small
  Details sind in der Anlage 2 der AFV festgelegt, siehe Tabelle im Skriptum

  \vspace{10pt}
  \footnotesize
  \begin{minipage}{0.5\textwidth}
    \begin{tabular}{rlrl}
      \SIrange{1,81}{1,95}{\mega\Hz}: & \SI{160}{\metre} \\
      \SIrange{3,5}{3,8}{\mega\Hz}: & \SI{80}{\metre} \\
      \SIrange{7,0}{7,2}{\mega\Hz}: & \SI{40}{\metre} \\
      \SIrange{10,1}{10,5}{\mega\Hz}: & \SI{30}{\metre} \\
      \SIrange{14,0}{14,35}{\mega\Hz}: & \SI{20}{\metre} \\
      \SIrange{18,068}{18,168}{\mega\Hz}: & \SI{17}{\metre} \\
      \SIrange{28,0}{29,7}{\mega\Hz}: & \SI{10}{\metre} \\
      \SIrange{50}{52}{\mega\Hz}: & \SI{6}{\metre} \\
      \SIrange{144}{146}{\mega\Hz}: & \SI{2}{\metre} \\
      \SIrange{430}{440}{\mega\Hz}: & \SI{70}{\centi\metre} \\
    \end{tabular}
  \end{minipage}
}
\card{41}{Wie arbeiten Sie mit ausländischen Amateurfunkstationen zusammen, die einen anderen/erweiterten Bandbereich benutzen? (Beispiele: \SI{40}{\metre}, \SI{80}{\metre})?}{
  Man nennt den Betrieb mit unterschiedlichen Sende- und Empfangsfrequenz ,,Split-Betrieb``. Dabei bleibt der Empfänger auf der Sendefrequenz der Gegenstation und der Sender wird auf die von der Gegenstation genannte, im zulässigen Frequenzband liegende, Frequenz eingestellt.
}
\card{42}{Was bedeuten die folgenden Abkürzungen: BK, CQ, CW, DE, K?}{
  \begin{description}
    \item[BK] Unterbrechung [break]
    \item[CQ] an alle Funkstellen [seek you]
    \item[CW] Telegraphie [continuous wave]
    \item[DE] von
    \item[K] kommen
  \end{description}
}
\card{42}{Was bedeuten die folgenden Abkürzungen: PSE, RST, R, N, UR?}{
  \begin{description}
    \item[PSE] bitte [please]
    \item[RST] Rapport [readability, signal strength, tone equality]
    \item[R] verstanden [roger]
    \item[N] Nein
    \item[UR] dein (your)
  \end{description}
}
\card{42}{Was bedeuten die folgenden Abkürzungen: FB, DX, RPT, HW, CL?}{
  \begin{description}
    \item[FB] gut [faible]
    \item[DX] Weitverbindung
    \item[RPT] Wiederholdung [repeat]
    \item[HW] wie? [how?]
    \item[CL] ich schließe die Funkstelle [close]
  \end{description}
}
\card{43}{Wie wirkt sich Polarisationsfading auf den Kurzwellenbetrieb aus?}{
  \small
  Polarisationsfading sind Feldstärkenschwankungen am Empfangsort durch Drehung der Polarisationsebene.

  \item Nach einmaliger Reflexion an der Ionosphäre sind alle Funkwellen elliptisch polarisiert, daher sie enthalten dann immer einen vertikalen und horizontalen Polarisationsanteil.

  \item Auswirkung: Die Signalfeldstärke bei Verwendung einer linear polarisierten Antenne geht nie ganz auf Null zurück, das auftretende Fading kann aber den Empfang für Sprechfunk teilweise fehlerhaft oder unmöglich machen.
}
\card{44}{Was versteht man unter Schwund im Kurzwellenbereich und wie reagieren Sie, um den Funkverkehr aufrecht zu erhalten?}{
  \small
  Schwund bedeutet Fading bzw. das Schwanken der Empfangsfeldstärken. Ursache: Mehrwegeausbreitung und nachfolgender Überlagerung von Signalen mit Phasenunterschied am Empfangsort sowie Drehung der Polarisationsebene, durch Schwankungen der Elektronendichte in der Ionosphäre.

  Maßnahmen:
  \begin{itemize}
    \item langsamer sprechen, wiederholen, buchstabieren
    \item Polarisationsebenenwechsel der Empfangsantenne
    \item Frequenzwechsel
    \item Bandwechsel
  \end{itemize}
}
\card{45}{Welche Maßnahmen ergreifen Sie, wenn Sie darauf aufmerksam gemacht werden, dass Ihre Aussendung ,,splattert``?}{
  \footnotesize
  Splattern ist ein übersteuertes Sendesignal, bei dem zu große Bandbreite und Nebenaussendungen auftreten. Ursache ist die Übersteuerung der Senderendstufe oder eines Leistungsverstärkers bis in den  nichtlinearen Teil der Kennlinie. Lösung:
  \begin{enumerate}
    \item Zurücknahme der Sendeleistung und das Neuabstimmen der Sendeendstufe
    \item ggf. hilft eine Zurücknahme der Mikrophonverstärkung
    \item bleiben die ersten 2 Maßnahmen ohne Erfolg, muss die gesamte Signalaufbereitung des Senders überprüft werden.
  \end{enumerate}
}
\card{46}{Was ist ein ,,Pile-Up`` -- wie verhalten Sie sich richtig?}{
  \small
  Ein Pile-Up bezeichnet die Situation, wenn eine große Zahl von Funkstationen eine meist sehr seltene Station rufen. Durch mangelhafte Funkdisziplin entsteht ein hoher Störpegel, der einen raschen und geordneten Betrieb oft erschwert.
  \begin{itemize}
    \item zuerst hören und herausfinden wie die Betriebsabwicklung erwünscht ist (Split-Betrieb, Listen usw.)
    \item beachten der eventuell vorhandenen Regeln
    \item wenn man selbst die Ursache des Pile-Up  sind, Regeln festlegen!
  \end{itemize}
}
\card{47}{Was verstehen Sie unter den Begriffen {\footnotesize\texttt MAYDAY - SECURITEE - SILENCE MAYDAY - MAYDAY RELAY?}}{
  \small
  Verwendung dieser Begriffe bei Notruf/Notverkehr:
  \begin{description}
    \item[MAYDAY (in Telegraphie SOS)] Notruf
    \item[SCURITEE] Sicherheitsfunkverkehr
    \item[SILENCE MAYDAY] Aufforderung zur Einhaltung einer Funkstelle, für alle nicht am Notfunkverkehr teilenehmenden Funkstellen
    \item[MAYDAY RELAY] Ankündigung der Übermittlung eines Notrufes für eine in Not befindliche Station.
  \end{description}
}
\card{48}{Welche Mess- und Kontrollgeräte sind bei einer Amateurfunkstelle vorgeschrieben?}{
  \small
  \begin{description}
    \item[Frequenzmessgerät] bei selbstgebauten od. veränderten Sende- oder Empfangsanlage
    \item[Strom- und Spannungsmessgerät] Spannung $>50~V$
    \item[Leistungsmessgerät] bei Sendeanlagen, die d. Betrieb einer höheren Sendeleistung ermöglichen, als bewilligt -Leistungsstufe.
  \end{description}
  Für den Großteil der kommerziellen Amateurfunkgeräte mit eingebauter Frequenzanzeige und definierter Sendeleistung sind daher keine Mess- und Kontrollgeräte verbindlich vorgeschrieben.
}
\card{49}{Was ist bei der Abstimmung des Leistungsverstärkers einer Amateurfunkstelle zu beachten?}{
  \begin{itemize}
    \item Der Leistungsverstärker eines Senders ist immer abstrahlungsfrei abzustimmen
    \item wird durch Verwendung einer geeigneten ,,Kunstantenne`` (,,Dummy-Load``) sichergestellt, die so aufgebaut ist, dass von ihr keine störende Abstrahlung erfolgt.
  \end{itemize}
}
\card{50}{Wie wird ein Funkrufzeichen allgemein bzw. ein Amateurfunkrufzeichen aufgebaut – nach welcher Vorschrift?}{
  \begin{itemize}
    \item Geregelt in der Vollzugsordnung für den Funkdienst – VO-Funk (in Österreich durch AFG und AFV umgesetzt)
    \item Jedes Funkrufzeichen beginnt mit dem Landeskenner, den Ziffern und/oder Buchstaben (oder eine Kombination)
    \item Amateurfunkrufzeichen beginnen mit dem Landeskenner (ja nach Größe des Landes auch mehrere), einer Ziffer und 1-3 Buchstaben.
  \end{itemize}
}
\card{51}{Buchstabieren Sie folgende Worte bzw. den folgenden Text nach dem internationalen Buchstabieralphabet: \dots}{
  Alfa – Bravo – Charlie – Delta – Echo – Foxtrott – Golf – Hotel – India – Juliett – Kilo – Lima – Kike – November – Oslar – Papa – Quebec – Romeo – Sierra – Tango – Uniform –Viktor – Whiskey – X-Ray – Yankee - Zulu
}
\card{52}{Was ist beim Betrieb an den Bandgrenzen zu beachten?}{
  Es ist zu beachten, dass die Aussendung im gesamten Umfang die Bandgrenze nicht überschreitet. Dabei ist die Toleranz der verfügbaren Messmöglichkeiten/Messgeräte und die verwendete Sendeart zu beachten. Messen kann man die Bandbreite der Aussendung mit z.B. einem Spektrum-Analysator.

  Beispiel: bei Verwendung von USB-Modulation darf nicht näher als \SI{3,5}{\kilo\Hz} an die obere Bandgrenze herangegangen werden.
}
\card{53}{Nennen Sie Beispiele österreichischer Amateurfunkrufzeichen mit Zusätzen (zB: am, mm, /1).}{
  \footnotesize
  \begin{description}
    \item[/am] für den Betrieb an Bord eines im Flug befindlichen Luftfahrzeuges
    \item[/mm] für Betrieb an Bord eines Schiffes in internationalen Gewässern
    \item[/m] für mobile (im Auto, Zug…)
    \item[/p] für portable (zu Fuss..)  Ziffern /1 -/9 für vorübergehenden Betrieb an einem anderen Standort
  \end{description}
  Zusätze können von den Fernmeldebehörden (besonderer Anlass) genehmigt werden.
  Zusätze werden vom Rufzeichen mit einem Schrägstrich (Slash) getrennt.
}
\card{54}{Nennen Sie die Landeskenner von fünf Nachbarländern und von fünf weiteren Ländern.}{
  \begin{minipage}{0.48\textwidth}
    \begin{description}
      \item[I] Italien
      \item[DL] Deutschland
      \item[OK] Tschechien
      \item[OM] Slowakei
      \item[HA] Ungarn
    \end{description}
  \end{minipage}
  \begin{minipage}{0.5\textwidth}
    \begin{description}
      \item[F] Frankreich
      \item[G] England
      \item[UA] Russland
      \item[SM] Schweden
      \item[SV] Griechenland
    \end{description}
  \end{minipage}
}
\card{55}{Was bedeuten die Ziffern im österreichischen Amateurfunkrufzeichen, welche Rufzeichenzusätze sind zulässig?}{
  Sie geben das Bundesland des Standortes der Amateurfunkstelle an. Für alle zulässigen Zusätze siehe B~53.

  \begin{tabular}{llll}
    1 & Wien               & 6 & Steiermark \\
    2 & Salzburg           & 7 & Tirol \\
    3 & Niederösterreich   & 8 & Kärnten \\
    4 & Burgenland         & 9 & Vorarlberg \\
    5 & Oberösterreich     &   & \\
    0 & \multicolumn{3}{l}{für genehmigte Amateurfunkstellen auf} \\
      & \multicolumn{3}{l}{ausrüstungspflichtigen Schiffen und} \\
      & \multicolumn{3}{l}{für internationales Gebiet} \\
  \end{tabular}
}
\card{56}{Welche Bestimmungen sind beim Betrieb im \SI{160}{\metre}-Band zu beachten?}{
  Laut Anlage 2 der AFV:

  \footnotesize
  \vspace{10pt}
  \begin{tabular}{clll}
    Frequenz             & Klasse & Leistungsstufe & Sendearten \\
  \hline
    \SIrange{1810}{1830}{\kilo\Hz} & 1,4  & A      & M \& F \\
    \SIrange{1830}{1840}{\kilo\Hz} & 1,4* & A \& B & M \& F \\
    \SIrange{1840}{1850}{\kilo\Hz} & 1,4  & A \& B & M \& F \& S \\
    \SIrange{1850}{1950}{\kilo\Hz} & 1,4* & A      & M \& F \& S \\
  \end{tabular}
  \vspace{10pt}

  M = Morsetelegraphie, F = Fernschreibetelegraphie, S = Fernsprechen

  * \quad mit Status S
}
\card{57}{Welche Betriebsverfahren werden bei Scatter-Verbindungen verwendet?}{
  \small
  Scatter-Verbindungen sind Funkverbindungen, die auf Streueffekten während der Funkausbreitung beruhen
  \begin{itemize}
    \item man unterscheidet je nach Streurichtung die Vorwärts-, Rückwärts-, und Seitenstreuung
    \item es werden Richtantennen mit hohem Gewinn und relativ hohen Sendeleistungen benötigt
    \item bevorzugte Verwendung von Telegraphie oder digitaler Verfahren, wegen der rasch ändernden Eigenschaften des Streuvolumens
    \item Sendedurchgänge möglichst kurz halten
  \end{itemize}
}
\card{58}{Welche Betriebsverfahren werden bei Meteorscatter-Verbindungen angewendet?}{
  \begin{itemize}
    \item werden durch Reflexionen an lokalen Elektronenwolken ermöglicht
    \item bevorzugte Verwendung von Hochgeschwindigkeitstelegraphie bzw. digitaler Übertragungsverfahren wegen der Kurzlebigkeit der Ionenwolke
    \item die Verbindungen dauern meist nur wenige Sekunden
  \end{itemize}
}
\card{59}{Erklären Sie die Betriebsabwicklung bei Relaisbetrieb.}{
  \begin{itemize}
    \item Relaisbetrieb dient zur Erhöhung der Reichweite, zur Unterstützung des Mobilbetriebes mit kurzen Antennen - der Relaisbetrieb wird über ein Frequenzpaar abgewickelt (Frequenzablage ist genormt)

    \item Satellitenverkehr - der Satellit arbeitet auch wie ein Relais - aufgrund der hohen orbitalen Geschwindigkeit ändert sich die Uplink- und Downlinkfrequenz für die Bodenstation während eine Überflugs (siehe Doppler-Effekt)
  \end{itemize}
}
\card{60}{Was versteht man unter ,,EME-Verbindungen``? Welches Betriebsverfahren wird angewendet?}{
  \small
  ,,EME-Verbindungen`` sind Reflexionsverbindungen, wobei der Mond als Reflektor verwendet wird
  \begin{itemize}
    \item wegen der meist nur sehr geringen Empfangsstärken werden Telegraphie, digitale Verfahren oder andere Schmalbandbetriebsarten verwendet (Sprechfunk sind eher selten)
    \item hohe Stationsaufwand notwendig (drehbare und nachführbare Richtantenne mit hohem Gewinn, sehr rauscharme, hochempfindliche Vorverstärker und Mindestsendeleistung)
  \end{itemize}
}
\card{61}{Was verstehen Sie unter Packet Radio? Welches Betriebsverfahren wird angewendet?}{
  \small
  \begin{itemize}
    \item zählt zu den Maschinenbetriebsarten - ein PC und ein Modem sind erforderlich
    \item Informationen werden in Daten-Pakete (Software) zerlegt und mit Adresse und zusätzlichen Informationen zur Sicherung der Übertragung versehen
    \item mehrere Stationen können gleichzeitig denselben Übertragungskanal benutzen
    \item zur Abwicklung des PR-Verkehres ist ein bestimmtes Protokoll (AX-25) vorgeschrieben
  \end{itemize}
}
\card{62}{Was verstehen Sie unter den Begriffen Mailbox, Digipeater, Netzknoten und welche betriebliche Besonderheiten sind zu beachten?}{
  \small
  \begin{description}
    \item[Mailbox] elektronischer Briefkasten
    \item[Digipeater] Relaisfunkstelle für digitale Betriebsarten
    \item[Netzknoten] Vernetzung von Digipeatern untereinander
  \end{description}
  \begin{itemize}
    \item Wenn man eine bestimmte Funkstrecke überbrücken will, wird praktisch automatisch über Netzknoten ,,durchverbunden``, wenn der Zielpunkt bekannt ist.
    \item Über Anwenderprogramme werden verschiedenste Betriebsarten und Funktionen im DSP (Digital Signal Processor) der Soundcard kostengünstig realisiert
  \end{itemize}
}
\card{63}{Erklären Sie die Begriffe Relaisfunkstelle, Transponder, Bakensender und welche betrieblichen Besonderheiten sind zu beachten?}{
  \small
  \begin{description}
    \item[Relaisfunkstelle] unbemannt, zur Reichweitenerhöhung
      Über die Eingabefrequenz wird angesprochen und über eine Ausgangsfrequenz sofort automatisch abgesetzt, Frequenzen müssen richtig eingestellt werden (Shift), kurze Durchgänge
    \item[Transponder] unbemannt, zur Reichweitenerhöhung, verwendet 2 Amateurfunkbänder
    \item[Bakensender] unbemannter Sender, sendet neben dem Rufzeichen weitere Informationen automatisch, zur Überwachung der Ausbreitungsbedingungen
  \end{description}
}
\card{64}{Erklären Sie die Betriebsabwicklung bei ATV-Betrieb.}{
  AVT-Betrieb ist die Amateurfunk-Fernsehübertragung (Amateur Television)
  \begin{itemize}
    \item zusätzlich zur Standartausrüstung wird eine Videokamera und ein ATV-Konverter benötigt
    \item Übertragung ist analog und digital möglich
    \item die Betriebsabwicklung erfolgt auf vereinbarten oder vorgeschriebenen Frequenzen (\SI{70}{\metre}-Band oder höher)
  \end{itemize}
}
\card{65}{Was ist bei Überreichweitenbedingungen zu beachten?}{
  \small
  \begin{itemize}
    \item Überreichweiten ist die Funkausbreitung, bei der Reichweiten deutlich über die normal zu erwartende Entfernung einer Funkverbindung hinaus auftreten
    \item diese Bedingungen sind meist kurzlebig
    \item Aussendung sollte relativ kurz gehalten werden
    \item bei einer nicht ,,ausgewogenen`` Stationsausrüstung, können Überreichweiten andere Stationen stören
  \end{itemize}
}
\card{66}{Welchen Einfluss hat die Wahl des Standortes für UKW-Ausbreitung?}{
  \begin{itemize}
    \item Ausbreitung auf Frequenzen über \SI{100}{\mega\Hz} erfolgt ,,quasi-optisch``
    \item dieses Verhalten nimmt mit steigender Frequenz weiter zu
    \item je höher der Sendestandort desto größer die Reichweite
    \item Durch Reflexionen an größeren Hindernissen, kann es zur Funkschatten kommen (machen Funkverbindung unmöglich)
  \end{itemize}
}
\card{67}{Erklären Sie das Betriebsverfahren SSTV.}{
  \footnotesize
  SSTV ist die Übertragung nicht bewegter Bilder (Standbilder, Slow Scan Television)
  \begin{itemize}
    \item der Bildinhalt wird abgetastet und schmalbandig übertragen (Übertragungsbandbreite \SIrange{2}{3}{\kilo\Hz})
    \item eignet sich auch für KW
    \item neben Videokamera benötigt man einen SSTV-Konverter oder einen PC mit entsprechender Software
    \item Übertragung erfolgt analog in der Betriebsart SSB (SSB-tauglicher Transceiver ist notwendig)
    \item Verwendung von vereinbarten Frequenzen und Übertragungsgeschwindigkeiten
  \end{itemize}
}
\card{68}{Nennen Sie Einflüsse, die die Lesbarkeit einer Funkverbindung verschlechtern.}{
  \small
  Natürliche Einflüsse
  \begin{itemize}
    \item Ausbreitungsbedingungen
    \item sehr starke Niederschläge
    \item Fadingerscheinungen
  \end{itemize}

  Fremdstörungen:
  \begin{itemize}
    \item zu geringer Frequenzabstand zu anderen Stationen
    \item Splattern
  \end{itemize}
}
\card{69}{Wie beurteilen Sie die Aussendung Ihrer Gegenstelle und wie wird diese Beurteilung der Gegenstelle mitgeteilt?}{
  Beurteilung mit der RS(T)-Beurteilung:

  \begin{tabular}{cl}
    R & Lesbarkeit \\
    S & Lautsärke \\
    T & Tonqualität (nur bei Telegraphieaussendung) \\

    R1 & nicht lesbar \\
    R5 & ausgezeichnet lesbar S1] kaum hörbar \\
    S9 & sehr stark hörbar \\
    T1 & äußest roher Ton \\
    T9 & reiner Ton \\
  \end{tabular}
}
\card{70}{Wie teilen Sie der Gegenstation Ihren Standort mit?}{
  Angabe von
  \begin{itemize}
    \item Ortsnamen oder
    \item geographischen Koordinaten oder
    \item QRA-Locator (GPS-Maidenhead-Locator)
  \end{itemize}
}
\card{71}{Was ist ein ,,Contest``? Wie verhalten Sie sich richtig?}{
  \small
  Ein Funkfeuerwettbewerb, bei dem möglichst viele Stationen in einer bestimmten Zeit erarbeitet werden sollen.
  \begin{itemize}
    \item der jeweilige Veranstalter gibt ,,Contest Regeln`` heraus, die man durch zuhören, im Internet oder durch Zeitschriften erfährt
    \item möchte man nicht teilnehmen: anderes Frequenzband oder Frequenzsegment aufsuchen!
    \item erkennbar durch den Anruf ,,CQ Contest``
  \end{itemize}
}
\card{72}{Wie gehen Sie bei der Planung einer Amateurfunkverbindung zu einem bestimmten Ort vor?}{
  \small
  \begin{itemize}
    \item Ausgangspunkt ist die verfügbare technische Ausrüstung
    \item liegt die Gegenstation innerhalb des Radiohorizontes= direkt
    \item außerhalb des Radiohorizontes, ist zu prüfen, ob mittels natürlicher Hilfen oder durch Verwendung von Relaisfunkstellen oder über Raumwellenausbreitung die Gegenstation erreicht werden kann
    \item ist dies auch nicht möglich, kann mittels Nutzung eines SF-Satelliten die Verbindung geplant werden.
  \end{itemize}
}
\card{73}{Was ist hinsichtlich der Herstellung oder Veränderung von Amateurfunkgeräten zu beachten?}{
  \small
  \begin{itemize}
    \item lizensierte FA mit Lizenzklasse 1 sind berechtigt Sendeanlagen selbst zu errichten
    \item zu beachten: die Eigenbaugeräte oder modifizierten Geräte müssen den technischen Bestimmungen betreffend Art und Bandbreite der Aussendungen, Neben- und Oberwellenfreiheit, sowie der zulässigen maximal abgegebenen Sendeleistung entsprechen.  Wenn nicht vorhanden benötigt man:
    \item Frequenzmessgerät
    \item Strom- und Spannungsmessgerät
  \end{itemize}
}
\card{74}{Beschreiben Sie das typische Ausbreitungsverhalten in den Frequenzbändern 6m--2m und \SI{70}{\centi\metre}.}{
  \begin{itemize}
    \item Standardausbreitung: Lichtähnliche Ausbreitung, Abschattung, Reflexion, etc.
    \item Erhöhte Reichweite durch Inversionsschichten in der Tropsphäre (\SI{70}{\metre}=sehr gut; 2m=gut; 6m=weniger gute Reaktion)
    \item In den Sommermonaten erhöhte Reichweiten durch sporadische E-Schichten (\SI{70}{\metre}=schlechte; 2m=gut; 6m=sehr gute Reaktion)
  \end{itemize}
}

\def\chap{Betrieb und Fertigkeiten \class{3}}

\card{01}{Frequenzbereich des \SI{70}{\centi\metre}-Amateurfunkbandes / \SI{2}{\metre}-Bandes?}{
  \SI{70}{\centi\metre}-Amateurfunkband: \SIrange{430}{440}{\mega\Hz},
  439,\SIrange{1}{440}{\mega\Hz} nur Empfang,
  max. Bandbreite \SI{1}{\mega\Hz}

  Frequenzbereich 2m-Band: \SI{144}{\mega\Hz}-\SI{-146}{\mega\Hz}, \SI{40}{\kilo\Hz} Bandbreite
}
\card{02}{Wie eröffnen Sie einen Sprechfunkverkehr?}{
  \small
  \begin{enumerate}
    \item Reinhören, ob die Frequenz frei ist (tote Zohne berücksichtigen)
    \item ,,is this frequency in use?{}``
    \item Antwort ,,QRL`` oder ,,this frequency is in use`` $\rightarrow$ ,,sorry`` oder ,,SRI`` $\rightarrow$ andere Frequenz suchen
    \item Ist die Frequenz frei, allgemeiner Anruf mit: ,,CQ, CQ, CQ, this is \dots 2$\times$, call, call``
          oder gezielt nach einer Station rufen. Bei einem Funkwettbewerb ,,CQ CONTEST 3$\times$ this is \dots``
  \end{enumerate}
}
\card{03}{Wie sind Amateurfunkrufzeichen aufgebaut?}{
  Durch Vollzugsordnung-Funk geregelt:
  Landeskenner, Ziffern (Präfix) und/oder Buchstaben (Suffix)

  Große Länder verfügen über mehrere Landeskenner. In Österreich: OE Landeskenner, eine Ziffer, die das Bundesland angibt und drei Buchstaben, durch die FMB zugeteilt, plus Zusatz
}
\card{04}{Welche Zusätze zu einem Amateurfunkrufzeichen sind zulässig?}{
  \begin{tabular}{rl}
    \texttt{/6  } & vorübergehend ,,6`` als anderes Bundesland \\
    \texttt{/m  } & mobile \\
    \texttt{/am } & air mobile \\
    \texttt{/mm } & maritim mobile \\
    \texttt{/p  } & portable \\
    \texttt{/500} & bewilligte Sonderzusätze
  \end{tabular}
}
\card{05}{Nennen Sie mindestens 5 Landeskenner der umliegenden Länder.}{
  \begin{tabular}{llcll}
    \texttt{HA} & Schweiz     & & \texttt{S5} & Slowenien \\
    \texttt{HB} & Ungarn      & & \texttt{9A} & Kroatien \\
    \texttt{DL} & Deutschland & & \texttt{G}  & Großbritannien \\
    \texttt{I}  & Italien     & & \texttt{F}  & Frankreich
  \end{tabular}
}
\card{06}{Wie beurteilen Sie das Signal Ihrer Gegenstation?}{
  \small
  \begin{tabular}{cl}
    R1 bis R5 & Lesbarkeit (readability) \\
    S1 bis S9 & Lautstärke (signal strength) \\
    T1 bis T9 & Tonqualität (tone quality)
  \end{tabular}
  \vspace{10pt}

  wobei:
  \begin{description}
    \item[Lesbarkeit] R1 = nicht lesbar, R5 = ausgezeichnet lesbar
    \item[Lautstärke] S1 = kaum hörbar, S9 = sehr stark hörbar
    \item[Tonqualität] T1 = äußerst rauer Ton, T9 = reiner Ton
  \end{description}
}
\card{07}{Was versteht man unter ,,S-Stufe(n)``?}{
  S-Stufen beurteilen die Lautstärke (signal strength) einer Gegenstelle. (S-Meter)

  \vspace{20pt}
  \begin{tabular}{clclcl}
    S1 & kaum hörbar    & S6 & gut  \\
    S2 & sehr schwach   & S7 & mäßig stark \\
    S3 & schwach        & S8 & stark \\
    S4 & mittelmäßig    & S9 & sehr stark hörbar \\
    S5 & ziemlich gut   &    & \\
  \end{tabular}
}
\card{08}{Was versteht man unter Not- und Katastrophenfunkverkehr, wie wird er gekennzeichnet?}{
  §~14 AFG \emph{Notfunkverkehr} ist die Übermittlung von Nachrichten zwischen einer Funkstelle, die selbst in Not ist, beteiligt ist oder Zeuge eines Notfalls ist mit Hilfe leistenden Funkstellen. Kennzeichnung ,,MAYDAY 3$\times$`` gesprochen, ,,SOS 3$\times$`` in Telegraphie, Abschluss mit ,,OVER``. Mitschreiben und Funküberwachung verständigen. Weiterleitung mit ,,MAYDAY RELAIS``. \emph{Sicherheitsfunkverkehr} wird durch das Sicherheitszeichen ,,SECURITEE`` gekennzeichnet.
}
\card{09}{Wie nahe dürfen Sie beim Sendebetrieb an die Bandgrenze herangehen?}{
  Der Frequenzabstand ist abhängig vom Modulationsverfahren, die Aussendung im gesamten Umfang darf die Bandgrenzen nicht überschreiten (zB. Schmalbandmodulation LSB (lower side band): untere  Bandgrenze \SI{+3,5}{\kilo\Hz}, USB (upper side band): obere Bandgrenze \SI{-3,5}{\kilo\Hz}). Messung der Aussendung mit Spektrum Analysator. Toleranz der Messgeräte berücksichtigen!
}
\card{10}{Welche Sendearten sind mit der Bewilligungsklasse~3 zulässig und mit welcher maximalen Sendeleistung?}{
  §~8.~(3) AFV in Anlage~2 bezeichnete Frequenzbereiche, \SI{144}{\mega\Hz} bis \SI{146}{\mega\Hz} und \SI{430}{\mega\Hz} bis \SI{440}{\mega\Hz} (Sendebetrieb bis \SI{439,1}{\mega\Hz}) Beachtung der Verhaltensvorschriften und in der AFV enthaltenen Einschränkungen, keine Eigenbaugeräte erlaubt, Leistungsklasse A (bis \SI{100}{\watt} PEP), FM, 2~m: max \SI{40}{\kilo\Hz} Bandbreite.
}
\card{11}{Was versteht man unter einem Amateurfunkrelais, wozu dient es?}{
  Relais = Sender und Empfänger auf zwei unterschiedlichen Frequenzen, Empfangssignal moduliert den Sender, meist mit gemeinsamer Antenne an einem hochgelegenen Standort.

  Ermöglicht große Reichweiten im UKW Band.
  Senderauftastung durch Squelch oder Pilotton.
  Versatz 2~m: \SI{0,6}{\mega\Hz}
}
\card{12}{Wie wickeln Sie einen Betrieb über ein Amateurfunkrelais ab?}{
  Relaisfrequenz sowie Versatzfrequenz am Funkgerät einstellen, eventuell auch Pilotton zur Auftastung  erforderlich. Versatzfrequenz auf 2~m: minus \SI{0,6}{\mega\Hz}, \SI{70}{\centi\metre}: \SI{7,6}{\mega\Hz}, beobachten, Pausen einhalten, bei Überreichweiten (Mehrfachöffnungen möglich) kurze Aussendungen
}
\card{13}{Buchstabieren Sie Ihren Vor- und Zunamen nach dem internationalen Buchstabieralphabet.}{
  Alfa – Bravo – Charlie – Delta – Echo – Foxtrott – Golf – Hotel – India – Juliett – Kilo – Lima – Kike – November – Oslar – Papa – Quebec – Romeo – Sierra – Tango – Uniform –Viktor – Whiskey – X-Ray – Yankee - Zulu
}
\card{14}{Wie verhalten Sie sich beim Empfang von Signalen mit ,,Doppler - Shift``?}{
  Tritt vor allem bei Satellitenverkehr auf. Wenn sich Gegenstation nähert, auf tieferer Frequenz senden, bei Entfernung Sendefrequenz erhöhen. Die Bodenstation muss laufend die Sende und Empfangsfrequenz      nachführen. Kann $\pm$\SI{12}{\kilo\Hz} betragen.
}
\card{15}{Was versteht man unter ,,Frequenzablage`` bei Relaisbetrieb?}{
  Frequenzablage bei Relaisbetrieb bezeichnet die Differenz (shift) zwischen Empfangs und Sendefrequenz, im 2~m Band: minus \SI{0,6}{\mega\Hz}, \SI{70}{\centi\metre}: \SI{7,6}{\mega\Hz} genormt. Die Ablage dient zur Trennung von Sendesignal und Empfangssignal im Relais.
}
\card{16}{Nennen Sie drei anormale Ausbreitungsmöglichkeiten im \SI{70}{\centi\metre}-Band oder 2~m Band.}{
  Überreichweiten: Reflexion an Sporadic E-Schicht, Troposphärisches Ducting durch Inversionsschichte, Scatter Verbindungen = Reflexion an Ionisationskanal durch Meteoriten oder Regen, Erde Mond Erde Verbindungen, Satellitenverbindung
}
\card{17}{Welche Betriebsverfahren werden im Satellitenfunkverkehr angewendet?}{
  Satellitenbetrieb entspricht Relaisbetrieb mit Kanal oder Bandumsetzer. 2~m, \SI{70}{\centi\metre}, \SI{23}{\centi\metre} Band. Nachführbare Jagi Antennen, Berücksichtigung der Dopplerverschiebung. Sende und Empfangsfrequenz müssen getrennt nachgeführt werden, digitale Betriebsarten von Vorteil. Transponder setzt zwischen zwei unterschiedlichen Bändern um.
}
\card{18}{Was verstehen Sie unter ,,Scatter-Verbindung``?}{
  Scatterverbindungen sind Funkverbindungen, die auf Streueffekten beruhen. Man unterscheidet im Bezug auf die Ausbreitungsrichtung: Vorwärtsstreuung, Rückwärtsstreuung, Seitenstreuung. Richtantennen und hohe Sendeleistung werden benötigt. Geringes S/N, daher bevorzugt digitale Verfahren. Niederschlagsstreuung ist auch für Sprechfunk geeignet. Durchgänge wegen Änderung der Ausbreitungsbedingungen kurz halten.
}
\card{19}{Was verstehen Sie unter ,,EME-Verbindung``?}{
  Erde-Mond-Erde Verbindungen sind Reflexionsverbindungen, wobei der Mond als Reflektor verwendet wird. Es sind nachführbare Richtantennen, rauscharme Vorverstärker und Mindestsendeleistung notwendig. Selten Sprechfunk, meist digitale Schmalbandverfahren.
}
\card{20}{Was verstehen Sie unter ,,Meteor-Scatter``?}{
  Meteorscatterverbindungen weden durch Reflexionen an lokalen Elektronenwolken ermöglicht, die beim Verglühen von Meteoren an der oberen Erdatmosphäre kurzzeitig auftreten. Wegen Kurzlebigkeit bevorzugt digitale Hochgeschwindigkeitstelegrafie, Dauer wenige Sekunden bis Minuten, dann auch Sprechfunk möglich
}
\card{21}{Was verstehen Sie unter ,,Tropo-Scatter``?}{
  Verbindung durch Rückstreuung am oberen Rand der Troposphäre. Richtfunkstrecke bis 200~km Reichweite möglich. Wegen geringer Rückstrahlung hohe Sendeleistung und Richtantennen nötig. Je nach Streurichtung unterscheidet man Vorwärtsstreuung, Rückwärtsstreuung und Seitenstreuung. Meist digital genutzt, Niederschlagsstreuung ermöglicht auch (kurze) Sprechfunkverbindungen.
}
\card{22}{Was verstehen Sie unter Überreichweiten, was unter dem Funkhorizont?}{
  \small
  Durch Anomalien in der Atmosphäre kann es zu Überreichweiten kommen, sporadic E-Verbindungen (,,Es``) , oder Ducting durch Inversionsschichten. Überreichweiten können rasch wechseln, daher sind die Aussendungen kurz zu halten. Störung anderer Stationen ist möglich, wenn Stationsausrüstung nicht ausgewogen.

  Über \SI{30}{\mega\Hz} tritt mit steigender Frequenz bei der Funkausbreitung quasioptisches Verhalten auf. Dämpfung, Brechung, Streuung, Reflexion und Beugung. Der Radiohorizont ist dabei ca. $\frac13$ größer als der optische Horizont, wird als Standardausbreitung bezeichnet.
}
\card{23}{Wodurch werden starke Überreichweiten im \SI{70}{\centi\metre}-Band verursacht?}{
  Überreichweiten treten bevorzugt bei großflächigen Temperaturinversionen auf wobei Reichweiten über 1000~km keine Seltenheit sind. Ausbildung von Ducts (,,atmosphärischer Wellenleiter``)  wobei das im \SI{70}{\centi\metre} Band bevorzugt gegenüber dem 2~m Band auftritt.
}
\card{24}{Wie verhalten Sie sich bei Überreichweitenbedingungen, wenn Sie im Relaisbetrieb arbeiten?}{
  Bei Überreichweitenbedingungen sind Mehrfachauftastungen möglich, Funkverkehr kann gestört werden. Durch die Ablage wird der eigene Sendekanal nicht mitgehört, man bemerkt die Störung einer Verbindung nicht und kann auch nicht auf ein Break reagieren. Die Verbindungen sind daher kurz zu halten.
}
\card{25}{Wie können Sie sich über die herrschenden Ausbreitungsbedingungen informieren?}{
  Für die UKW-Ausbreitung sind Wettervorhersagen zweckmäßig, weiters die Sonnenaktivität, die durch Sonnenfleckenrelativzahl und den solar flux bei 2,8 GHz angegeben wird .  Die Messung der Ausbreitungsbedingungen ist mit dem Empfang von Bakensendern möglich. Aktuelle Ausbreitungsbedingungen werden im Internet publiziert.
}
\card{26}{Welche Faktoren beeinflussen die erzielbare Reichweite im 2m-Band?}{
  \SI{2}{\metre} Band vorwiegend auf optischen Horizont beschränkt (\SI{+33}{\percent} darüber hinaus). Bei Hindernissen, die gut reflektieren (zB Hochhäuser) Ausbreitung durch Mehrfachreflexion. Selten gibt es sporadische Es-Schichten, die Raumwellen reflektieren. Ausbildung von Ducts (atmosphärische Wellenleiter) mit Reichweiten über \SI{1000}{\kilo\metre} (durch starke Sonneneinstrahlung oder Temperaturinversion). Dämpfende Faktoren: Nebel, Regen, Wälder. 
}
\card{27}{Erklären Sie die Bedeutung der auch im Sprechfunk verwendeten Q-Gruppen: QSO - QSY - QRL.}{
  \begin{description}
    \item[QSO] ich habe Verbindung mit \dots
    \item[QSY] wechseln sie auf die Frequenz \dots{} kHz
    \item[QRL]
      entspricht in CW der Frage: ,,Is this frequency in use?{}``,
      als Antwort: ,,bin beschäftigt, bitte nicht stören``
  \end{description}
}
\card{28}{Erklären Sie die Bedeutung der auch im Sprechfunk verwendeten Q-Gruppen: QRM - QRB - QSB.}{
  \begin{description}
    \item[QRM] Ich werde gestört (Fremdstörungen)
    \item[QRB] Entfernung zwischen unseren Funkstellen
    \item[QSB] Ihre Signalfeldstärke schwankt
  \end{description}
}
\card{29}{Erklären Sie die Bedeutung der auch im Sprechfunk verwendeten Q-Gruppen: QRT - QSL.}{
  \begin{description}
    \item[QRT] Stellen Sie die Aussendungen ein! \textit{oder} ICH stelle den Betrieb ein.
    \item[QSL] ich bestätige den Empfang, ich habe verstanden
  \end{description}
  QSL-Karte = Funkbestätigungskarte
}
\card{30}{Erklären Sie die Bedeutung der im Sprechfunk verwendeten Abkürzungen \\ 73 - 55 - 88 - CL.}{
  \begin{description}
    \item[73] beste Grüße
    \item[55] viel Erfolg
    \item[88] Liebe und Küsse
    \item[CL] ,,Close`` bzw. Schließen der Station
  \end{description}
}
\card{31}{Was versteht man unter der Betriebsart ,,Packet-Radio``, welche Betriebsverfahren werden dabei angewendet?}{
  Packet Radio ist eine \emph{Maschinenbetriebsart} (digital) d.h. es ist zusätzlich zB. ein PC erforderlich. Die Information wird in Datenpakete zerlegt und mit einer Adresse und Prüfsumme versehen. Mehrere Stationen können nacheinander denselben Kanal benutzen. Die Stationen sind europaweit vernetzt, damit Vergrößerung des Radiohorizonts. Protokoll AX~25, Funkverkehr kann mitgelesen werden, CQ an eine mitgelesene Station oder eigener Anruf wird gestartet.
}
\card{32}{Welche Faktoren beeinflussen die erzielbare Reichweite im \SI{70}{\centi\metre}-Band?}{
  Höhe des Standorts hat großen Einfluß, daraus ergeben sich Funkhorizont und Funkschatten.

  \SI{70}{\centi\metre} Band: \SIrange{430}{440}{\mega\Hz} quasioptisches Verhalten mit Dämpfung, Brechung, Streuung, Reflexion und Beugung. Ducting bei Inversionsschichten, Reflexion an Hindernissen (Felswand, Eisenbeton), Streuung durch Regenwolken.
}
\card{33}{Was verstehen Sie unter ,,Split-Betrieb``?}{
  Das ist der Betrieb mit unterschiedlicher Sende und Empfangsfrequenz. Der Empfänger bleibt auf der Sendefrequenz der Gegenstation, der Sender wird auf die von der Gegenstation genannte im zulässigen Frequenzband liegende Frequenz eingestellt. Sendefrequenzbereich darf nicht überschritten werden! Bezeichnung ,,QSX \dots{} kHz`` bei Telegrafie. ,,5 UP`` oder ,,10 DOWN`` im VOX-Betrieb.
}
\card{34}{Welche Verfahren werden bei ATV-Betrieb im \SI{70}{\centi\metre}-Band angewendet und welche Besonderheiten sind dabei zu beachten?}{
  ATV-Betrieb = Amateurfunk-Fernsehübertragung (Amateur Television). Neben Standardfunkausrüstung wird senderseitig eine Videokamera und ein ATV-Konverter benötigt.

  Auf der Empfangseite ist ein Bildschirm erforderlich. Die Übertragung kann analog aber auch digital erfolgen. Abwicklung im \SI{70}{\centi\metre} Band oder bei höheren Frequenzen (\SI{23}{\centi\metre}) mit SAT Receiver und Pre-Amplifier auf 1,2~GHz. Zu beachten ist die hohe Bandbreite des Videosignals $\Rightarrow$ Bandbreite max. 1~MHz.
}
\card{35}{Wie gehen Sie bei der Planung einer Amateurfunkverbindung zu einem bestimmten Ort vor?}{
  Ausgangspunkt ist die verfügbare technische Ausrüstung (Frequenzband, Sendeleistung, Betriebsarten, Antennen).

  \begin{description}
    \item[Aus der Entfernung] Festlegung ob innerhalb des Radiohorizonts, direkte Erreichbarkeit oder durch Relaisfunkstellen, Digipeater oder über Raumwellenausbreitung.
    \item[Sonst] Satellit oder warten auf Überreichweiten aus Wettervorhersage. (Inversion)
  \end{description}
}
\card{36}{Wie teilen Sie der Gegenstation den Standort ihrer Amateurfunkstelle mit?}{
  QTH (Standort) wird entweder mit geografischen Koordinaten (Längen und Breitengrad) übermittelt. Weiters mittels Ortsnamen oder QRA-Locator (Unterteilung der Erde in Groß, Mittel und Kleinfelder) Graz JN77RB (= Maidenhead Locator)
}
\card{37}{Was ist hinsichtlich der Herstellung oder Veränderung von Geräten für den Amateurfunkverkehr im 2~m oder \SI{70}{\centi\metre}-Band zu beachten?}{
  Amateurfunker der Bewilligungsklasse~3 dürfen nur Geräte aus kommerzieller Fertigung im 2~m und \SI{70}{\centi\metre} Band betreiben. Herstellung und Veränderung sind der Klasse~1 vorbehalten. Die Sendeleistung (PEP) darf maximal \SI{100}{\watt} betragen. Es wird kein CE Kennzeichen benötigt. Für NICHT modifizierte Geräte sind keine  Kontrollinstrumente für Frequenz, Spannung und Leistung erforderlich.
}
\card{38}{Sie haben einen abstimmbaren Leistungsverstärker - wie stimmen Sie ihn ab?}{
  Der Leistungsverstärker ist immer abstrahlungsfrei abzustimmen. Dies wird durch Verwendung einer Dummy Load erreicht (Kunstantenne = absorbierender Widerstand). Die Anpassung an die Betriebsantenne ist auf ein Minimum zu beschränken, wenn auf der gewählten Frequenz kein erkennbarer Funkverkehr stattfindet.
}

\def\chap{Technische Grundlagen \class{1}}

\card{01}{Ohmsches und Kirchhoff’sches Gesetz}{\small
Ohmsches Gesetz  gibt den Zusammenhang zwischen einem Widerstand  (R) der anliegenden 
Spannung (U) und dem durch den Widerst. fließenden Strom (I) wieder.
\[ U = I \cdot R  \qquad  I = U / R \qquad R = U / I \]
\begin{description}
  \item[1. Kirchhoffsches Gesetz] Parallelschaltung von Widerst., Gesamtstrom = Summe der Teilströme.
  \item[2. Kirchhoffsches Gesetz] Widerst. in Reihe geschaltet, Gesamtspannung = Summe der Teilspannungen.
\end{description}}

\card{02}{Begriff Leiter, Halbleiter, Nichtleiter}{
  \small
  \begin{description}
    \item[Leiter]
      Materialien, die den elektr. Strom sehr gut leiten. Alle Metalle, Kohle und Säuren.
      Beste Leitfähigkeit: Silber, Kupfer, Aluminium, Gold, Messing.
    \item[Halbleiter] Materialien, die Leitfähigkeit
      aufgrund physikalischer oder elektrischer Einflüsse ändern
      (Silizium, Germanium).
    \item[Nichtleiter] Isolatoren leiten schlecht bis gar nicht.
      Keramik, Kunststoff, trockenes Holz.
      Gute Isol.: Glas, Keramik, Teflon, Glasfaser Harz, Gummi.
  \end{description}
}


\card{03}{Kondensator, Begriff Kapazität, Einheiten - Verhalten bei Gleich- und Wechselspannung}{
  \small
  \begin{description}
    \item[Kondensator]
      Ladungsspeicher; besteht aus zwei elektr. leitenden Materialien, durch Isolator getrennt.
      Bei \emph{Gleichspannung} lädt er sich auf und kann später die Ladung an einen Verbraucher abgeben.
      Bei \emph{Wechselspannung} durch die laufende Umladung wird er zu einem Stromfluss im Leitungskreis,
      der mit steigender Frequenz zunimmt.
    \item[Einheit] Farad~(F) für Kapazität  \hspace{20pt} \textbf{Kürzel} C
    \item[Kleinere Einheiten] Milli- ($10^3$) bis Picofarad ($10^{12}$)
  \end{description}
}

\card{04}{Spule, Begriff Induktivität, Einheiten - Verhalten bei Gleich- und Wechselspannung}{
  \footnotesize
  \begin{description}
    \item[Spule] eine oder mehrere Windungen eines Leiters auf einen magnetischen Kern (Induktivität)
    \item[Gleichspannung] baut in der Spule ein Magnetfeld auf
    \item[Wechselspannung] durch den Richtungswechseln des Stromes kommt es zu Richtungswechseln des Magnetfeldes (Selbstinduktion) der dem verursachenden Strom entgegen wirkt.
    Mit steigender Frequenz nimmt Widerstand zu; als induktiver Blindwiderstand~(XL) bezeichnet.
    \item[Einheit] Henry~(H)   \hspace{50pt} \textbf{Formel} (L)
    \item[Kleinere Einheiten] Millihenry, Mikrohenry, PicoH 0,001~H = 1~mH = 1000 microH
  \end{description}
}

\card{05}{Wärmeverhalten von elektrischen Bauelementen}{
  Alle Metalle und die meisten guten Leiter erhöhen mit steigender Temperatur ihren Widerstand.
  \centerline{PTC $\Rightarrow$ positive temperatur coefficient}
  Die meisten Halbleiter verringern mit steigender Temperatur ihren Widerstand.
  \centerline{NTC $\Rightarrow$ negative temperatur coefficient}
  \small
  \begin{description}
    \item[Kenngrößen] gibt an um wie viel Ohm sich der Widerstand ändert, wenn die Temperatur um 1 Grad erhöht wird
    \item[Einheit] Ohm/Grad
  \end{description}
}

\card{06}{Stromquellen (Kenngrössen)}{
  \footnotesize
  \begin{description}
    \item[Gleichstrom Primärbatterien] Durch chemischen Prozess wird elektrische Spannung zwischen zwei Polen erzeugt. Strom kann entnommen werden (Entladung).
    \item[Sekundärbatterien] Akkus vorher aufladen, dann Strom entnehmen.
    \item[Beispiele]
      Bleiakku, Nickel-Cadmium-Akku, Nickel-Metallhybrid-Akku, Lithium-Ionen-Akku,
      Solarzelle, Piezo-Elemente
    \item[Kenngröße] Spannung, Strombelastbarkeit, Kapazität (Fassungsvermögen) in Ah
  \end{description}
  Die 220~V Steckdose liefert Wechselstrom mit 50~Hz.
}

\card{07}{Sinus- und nicht-sinusförmige Signale}{
  \footnotesize
  \begin{description}
    \item[Sinusförmige Signale] haben zeitlichen Verlauf der exakt einer mathemat. Sinusfunktion entspricht und sind frei von Oberwellen (zB Spannung des Wechselstromnetzes).
    \item[Nicht sinusförmige Signale]
      Wechselspannungen mit beliebigem Kurvenverlauf.
      Dreieck-, Rechteck-, Trapez-, Sägezahn-, Rauschsignale: Kombination aus aus mehreren Sinussignalen.
    \item[Kenngrößen]
      \begin{description}
        \item[bei Gleichspannung] Spannung (Amplitude)
        \item[bei Wechselspannung] 3 Kenngrößen: Kurvenform, Scheitelspannung (V), Frequenz (Hz) / Polaritätswechsel/sec
      \end{description}
  \end{description}
}

\card{08}{Was verstehen Sie unter dem Begriff Skin-Effekt?}{
  \small
  Bei zunehmenden Frequenzen wird Stromfluss im Leiter immer mehr zum Rand gedrängt.
  Strom fließt praktisch nur an der Außenhaut.
  Dadurch steigt der Widerstand an, was zu Leistungsverlust führt, nicht bei Gleichstrom.
  Dicke HF Leiter auch als Rohre ausgeführt.
  \begin{description}
    \item[Abhilfe] viele dünne Adern vergrößern die Oberfläche. Dickere Drähte und Versilbern der Leiter
    \item[Größenordnung] Eindringtiefe des Stroms
      9,\SI{38}{\milli\metre} bei 50~Hz, 70~$\mu$m bei 1~MHz, 7~$\mu$m bei \SI{100}{\mega\Hz}
  \end{description}
}

\card{09}{Gleich- und Wechselspannung - Kenngrößen}{
  \scriptsize
  \begin{description}
    \item[Gleichspannung] Spannung ist konstant, die Polarität verändert sich nicht. \textbf{Kürzel} DC (direct current) \textbf{Kenngrößen} Spannung, Strombelastbarkeit der Quelle, Kapazität in Ah
    \item[Wechselspannung]
      Spannung und Polarität ändern sich laufend ($\rightarrow$ Frequenz); der zeitliche Verlauf kann als Kurve dargestellt werden. 
      \begin{description}
        \item[Kürzel] AC (alternating current)
        \item[Kenngröße] Spannung, Amplitude, Frequenz, Kurvenform, Strombelastbarkeit der Quelle
        \item[Formelzeichen] $f = \frac1T$
        \item[Einheit] Hertz (Hz, kHz, MHz)
      \end{description}
  \end{description}
}


\card{10}{Was verstehen Sie unter dem Begriff Permeabilität?}{
  Wird ein Material in eine Spule eingebracht, erhöht dies die Induktivität der Spule.
  Permeabilität ist jene Materialkonstante, die angibt um wie viel höher die Induktivität
  gegenüber Vakuum ist.
  \begin{description}
    \item[Formelzeichen] $\mu$
    \item[Beispiele]
      \begin{tabular}{rlrl}
        Luft       & 1      & Eisen      & 5000 \\
        Aluminimum & 250    & Mu Metall  & 100 000 \\
        Nickel     & 600    &            & \\
      \end{tabular}
  \end{description}
}

\card{11}{Serien- und Parallelschaltung von $R$, $L$, $C$}{
  \begin{minipage}{0.5\textwidth}
    \centering
    Serienschaltung \\ von $R$ und $L$
    \begin{center}
      $ R_{\text{ges}} = R_1 + R_2 $ \\
      $ L_{\text{ges}} = L_1 + L_2 $
    \end{center}
    Parallelschaltung \\ von $R$ und $L$
    \begin{center}
      $ {R_{\text{ges}}} = \frac{R_1 \cdot R_2}{R_1 + R_2} $ \\
      $ {L_{\text{ges}}} = \frac{L_1 \cdot L_2}{L_1 + L_2} $
    \end{center}
  \end{minipage}
  \begin{minipage}{0.49\textwidth}
    \centering
    Parallelschaltung von $C$
    \begin{center}
      $ C_{\text{ges}} = C_1 + C_2 $
    \end{center}
    Serienschaltung von $C$
    \begin{center}
      $ {C_{\text{ges}}} = \frac{C_1 \cdot C_2}{C_1 + C_2} $
    \end{center}
  \end{minipage}
}


\card{12}{Was verstehen Sie unter dem Begriff Dielektrikum?}{
  Isolierende Schicht zwischen den Platten eines Kondensators. z.B. Keramik, Kunststoff; Teflon
  {\small
    \begin{description}
      \item[Kenngößen]
        Dielektritätskonstante, Materialkonstante die angibt um wie viel höher die Kapazität gegenüber Vakuum ist, wenn dieses Material zwischen den Kondensatorplatten angeordnet wird.
      \item[Beispiele] Luft 1, Papier 1--4, Teflon 2, Wasser 80, destilliertes Wasser isoliert
      \item[Eigenschaften] Hohe Dielektritätskonstante, hohe Spannungsfestigkeit, geringe Dicke
    \end{description}
  }
}

\card{13}{Wirk-, Blind- und Scheinleistung bei Wechselstrom.}{
  \begin{description}
    \item[Wirkleistung] nur ohmsche Widerstand (Verbraucher) vorhanden.
    \item[Blindleistung] nur kapazitive oder induktive Verbraucher vorhanden.
    \item[Scheinleistung] ohmsche und (kapazitive oder induktive) Verbraucher vorhanden.
  \end{description}
  {\small
    \textbf{Achtung!}
      Wirk- und Blindleistung können nicht addiert werden,
      da Wirk- und Blindströme nicht gleichphasig sind.
  }
}


\card{14}{Begriff elektrischer Widerstand (Schein-, Wirk- und Blindwiderstand), Leitwert}{
  \small
  \begin{description}
    \item[Ohmscher Widerstand] bei Gleichstrom nur Ohmscher Widerstand,
      keine Phasenverschiebung (,,Wirkwiderstand``),
      Leitwert ist Kehrwert des Ohmschen Widerstands: $G = \frac1{R}$. Einheit Siemens ($S$).
    \item[Blindwiderstand]
      Phasenverschiebung von Strom ($+90^\circ$) und Spannung ($-90^\circ$) bei $C$ und $L$.
      ,,Reaktanz``. Einheit Ohm.
    \item[Scheinwiderstand]
      Phasenverschiebung von 0--$90^\circ$. RC- und RL-Kombinationen. ,,Impedanz``. Einheit Ohm.
  \end{description}
}

\card{15}{Berechnen Sie den induktiven Blindwiderstand einer Spule mit $30~\mu H$ bei 7~MHz (Werte sind variabel)}{
  \centering
  siehe Skriptum, Seite 39, Frage~T15
}

\card{16}{Berechnen Sie den kapazitiven Blindwiderstand eines Kondensators von 500~pF bei \SI{10}{\mega\Hz} (Werte sind variabel)}{
  \centering
  siehe Skriptum, Seite 38, Frage~T16
}

\card{17}{Der Transformator - Prinzip und Anwendung}{
  \footnotesize
  Gemeinsamer Eisenkern mit 2 Wicklungen (Spulen) fließt Wechselströme in Spule (Primärspeicher).
  Dabei induziert das erzeugte wechselnde Magnetfeld in der 2. Spule (Sekundärspule)
  eine Wechselspannung. Die Wechselspannungen sind proportional zu den Windungszahlen
  = Übersetzungsverhältnis.
  \begin{description}
    \item[Anwendung] Stromversorgungs-, NF- und HF-Technik
    \item[Übertrager] anderes Wort für Transformator
    \item[Kenndaten]
      Primär- / Sekundärspannung, Windungszahlen, Übersetzungsverhältnis,
      maximal übertragbare Leistung, Impedanz
  \end{description}
}

\card{18}{Der Resonanzschwingkreis: Kenngrößen}{
  \footnotesize
  Zusammenschaltung von Kondensator und Spule. Sie weist einen frequenzabhängigen Scheinwiderstand $Z$ (Impedanz) auf. Jedes Element hat einen frequenzabhängigen Blindwiderstand $X_C$ bzw. $X_L$. $X_C$ nimmt mit Frequenz ab, $X_L$ nimmt zu.

  \begin{description}
    \item[Parallelschwingkreis] bei Resonanz ist $Z$ am Maximum
    \item[Serienschwingkreis] bei Resonanz ist $Z$ am Minimum
    \item[Kenngrößen] Resonanzfrequenz, Bandbreite, Güte
  \end{description}

  Die Resonanzfrequenz ist jene Frequenz $f_r$, für die die Blindwiderstände von $C$ und $L$ gleich sind und die Impenz $Z$ ohmsch ist (dabei ist auch kein Blindwiderstand mehr feststellbar).
}
\card{19}{Der Resonanzschwingkreis: Anwendungen in der Funktechnik}{
  \small
  Als \emph{Selektionsmittel} (Filter) eingesetzt, um Signale einer Frequenz hervorzuheben oder zu unterdrücken.

  Anwendung:
  \begin{itemize*}
    \item Eingangsschaltung von Empfängern
    \item HF-Verstärker
    \item Oszillatoren
  \end{itemize*}

  \begin{description}
    \item[Parallelschwingkreis] nutzt hohe Impedanz im Resonanzfall; nur erwünschte Signale gelangen in Empfänger
    \item[Serienschwingkreis] nutzt niedere Impedanz im Resonanzfall; unerwünschte Signale der Frequenz werden ,,kurzgeschlossen``, anderen gelangen in Empfänger.
  \end{description}
}
\card{20}{Berechnen Sie die Resonanzfrequenz eines Schwingkreises mit folgenden Werten: L = \SI{15}{\henry}, C = \SI{30}{\pico\farad} (Werte sind variabel)}{

  {\begin{align*}
    f &= \frac{159}{\sqrt{L\cdot C}} \\
      &= \frac{159}{\sqrt{15 \cdot 30}} \\
      &= \frac{159}{21.213} \\
      &= \SI{7.49}{\mega\Hz} \\
  \end{align*}}
}
\card{21}{Filter – Arten, Aufbau, Verwendung und Wirkungsweise}{
  \footnotesize
  \textbf{Arten:} Hochpassfilter, Tiefpassfilter, Bandpassfilter.
  \textbf{Verw.:}

  \begin{description}
    \item[Bandpass] Eingang von Empfängern
    \item[Oberwellenfilter] am Ausgang von Sendeverstärkern
    \item[Kenngrößen]
  \end{description}
  {
    \scriptsize
      \begin{minipage}{0.5\textwidth}
        \begin{itemize}
          \item Grenzfrequenz (u.G b. Hochp. o. G b. Tiefp.)  % TODO: whaaaat?
          \item Bandbreite (b. Bandpass)
          \item Durchlassdämpfung, Flankensteilheit (Anzahl Filterstufen)
          \item Welligkeit (Durchlass nicht alle Frequenz gleich)
        \end{itemize}
      \end{minipage}
      \begin{minipage}{0.47\textwidth}
        \begin{itemize}
          \item Quarzfilter: extr. Hohe Güte aufber. Sign. Empfänger u. Sender
          \item Aktive Filter: Im NF-Bereich Operationsverstärker für Audio Signale (Analogfilter).
            DSP (Digitale Filter).
        \end{itemize}
      \end{minipage}
  }
}
\card{22}{Was sind Halbleiter?}{
  \item
  Materialien, deren Leitfähigkeit durch physikalische Einflüsse gesteuert werden kann.
  Basismaterialien: Silizium, Germanium (bzw. deren Dotierungen [= Verunreinigung]).

  \item
  Je nach Dotierung entsteht P-Leiter (positive Ladungsträger) oder N-Leiter (negative Ladungsträger).
  Wichtige Eigenschaften kommen erst zustande  wenn P- und N-Leiter zusammengebracht werden.
}
\card{23}{Die Diode - Aufbau, Wirkungsweise und Anwendung}{
  \small
  Ein Halbleiter-Bauelement mit P-N-Übergang.
  P-Schicht ist Anode. N-Schicht ist Kathode.
  \begin{description}
    \item[Anwendungen] Gleichrichter (da nur 1 Flussrichtung)
    \item[Durchlass] +Pol an Anode (mind. \SI{0,7}{\volt} bei Silizium)
    \item[Sperre] +Pol an Kathode (gekennzeichnet durch Ring)
    \item[Kenngröße] Max. Sperrspannung, Strombelastbarkeit
    \item[Bauform] Schraubbef. (Kühlung), Kunstoffgehäuse, Glasgehäuse, Mehrfachdioden in einem Gehäuse
  \end{description}
}
\card{24}{Der Transistor - Aufbau, Wirkungsweise und Anwendung}{
  \small
  Ein Halbleiter-Bauelement.
  Besteht aus 2 N-Leitern zwischen denen eine dünne Schicht eines P-Leiters liegt (NPN-Typ; auch PNP-Typ möglich).
  Mittlere Schicht = \emph{Basis}.
  Äußere Schichten = \emph{Emitter} und \emph{Kollektor}. Jede Schicht hat Anschluss.
  In digitalen Schaltkreisen: Vielzahl von Transistoren auf gemeinsamer Unterlage (= \emph{Substrat}).

  \begin{description}
    \item[Kenndaten] Typ NPN oder PNP. Stromverstärkung, max. Kollektorspannung, Strom, Grenzfrequenz
    \item[Anwendungen] Verstärker NF-HF Oszillatoren, Signalerzeugung, Schalter, Regelkreise
  \end{description}
}
\card{25}{Die Elektronenröhre - Aufbau, Wirkungsweise und Anwendung}{
  \small
  \begin{description}
    \item[Diodenaufbau]
      In luftleerem Glaskolben mehrere Elektroden.
      Die Kathode wird zum Glühen gebracht, emittiert Elektronen, Anode fängt Elektronen auf (Stromfluss nur in dieser Richtung möglich).
    \item[Triodenaufbau]
      Gitterförmige Elektrode zwischen Anode und Kathode, Gittervorspannung bei Elektrode, große Anodenstromänderung durch kleine Spannungsänderung
    \item[Anwendung] als HF-Leistungsverstärker (PA = Power Amplifier)
  \end{description}
}
\card{26}{Arten von Gleichrichterschaltungen - Wirkungsweise}{
  \small
  \begin{description}
    \item[Einweg-Gleichrichter]
      Trafo (\SIrange{230}{12}{\volt})-Diode-Kondensator:
      Es wird nur positive Halbwelle verwendet, hohe Restwelligkeit, \SI{50}{\Hz}
    \item[Doppelweg-Gleichrichter]
      Trafo (Mittelanzapfung beider Halbwellen),
      2 Dioden,
      Verbindung beider Dioden zum Kondensator,
      beide Halbwellen verwendet, geringe Restwelligkeit \SI{100}{\Hz}
    \item[Vollweg/Brückengleichrichter]
      Trafo (1 Wicklung nötig)-4-Dioden-Kondensator,
      beide Halbwellen verwendet, geringe Restwelligkeit, \SI{100}{\Hz}
  \end{description}
}
\card{27}{Stabilisatorschaltungen}{
  In Stromversorgungs-, Funk-, Verstärker- und Messgeräten.
  Einfachste Form: Zenerdiode und Vorwiderstand Spannungsstabilisierung durch Zenerdiode und Längstransistor.

  Spannungen über $x$ Volt werden durch Zenerdiode und Transistor ,,vernichtet``.
  Nach dem Transistor liegen immer max. $x$ Volt an.
  Festspannungsregler: komplette Stabilitätsschaltung in einem IC.
  Festspannungsregler sind als integrierte Schaltkreise fertig erhältlich
}
\card{28}{Hochspannungsnetzteil - Aufbau, Dimensionierung und Schutzmaßnahmen}{
  ,,Hochspannung``: ab \SI{500}{\volt}.
  Primär als Leistungsverstärker in Elektronenröhren.
  Sorgfältig dimensionierte (spannungsfeste) Bauteile (Trafos, Gleichrichter, Kondensatoren, Stecker).

  \begin{itemize}
    \item Gleichspannungen im Hochspannungsbereich absolut lebensgefährlich!
    \item Schutzmaßnahmen ab \SI{50}{\volt} notwendig.
    \item Vor Eingriff Stecker ziehen und einige Minuten warten
  \end{itemize}
}
\card{29}{Welche Arten von digitalen Bauteilen kennen Sie? - Wirkungsweise}{
  \footnotesize
  Dienen der Erzeugung und Verarbeitung von digitalen Signalen.
  Nur zwei Spannungszustände (notiert als 0 und 1, ,,nichtlinear``).

  \begin{description}
    \item[Vorteile] Rechenoperationen im binären Zahlenraum
    \item[Gatter] sind Bauelemente zur logischen Verknüpfung
    \item[Kippstufen] Bauelemente um zwischen Zuständen zu wechseln
    \item[Puffer] Bauelemente, zum Speichern und Ausgeben von binären Signalfolgen
    \item[Zähler] Bauelemente, zum Ermitteln der Zahl von Impulsen pro Zeit
    \item[Anzeigen] Bauelemente zur grafischen Darstellung von Buchstaben und Symbolen
  \end{description}
}
\card{30}{Was sind elektronische Gatter? - Wirkungsweise}{
  \item
  Sind die einfachste Form digitaler Bauelemente.
  Sie verknüpfen zwei oder mehrere digitale Eingangssignale mit digitalen Ausgangssignal.

  \item
  Gatter kennen nur 2 Zustände: low und high, aktiv und passiv oder 0 und 1.
}
\card{31}{Messung von Spannung und Strom am Beispiel eines vorgegebenen Stromkreises}{
  \begin{description}
    \item[Voltmeter (Spannungsmessung)]
      Innenwiderstand soll möglichst hoch sein,
      Parallel zum Schaltungsteil gemessen
    \item[Amperemeter (Strommessung)]
      Innenwiderstand möglichst gering,
      Durch Auftrennen des Stromkreises in Reihe gemessen
  \end{description}
}
\card{32}{Erklären Sie die prinzipielle Wirkungsweise eines Griddipmeters, Anwendung und Funktion}{
  \begin{description}
    \item[Anwendung] Zur Bestimmung der Resonanzfrequenz eines Schwingkreises.
    \item[Wirkungsweise, Funktion]
      Besteht aus einem einstellbaren Oszillator.
      Wenn beide Frequenzen übereinstimmen, wird dem Oszillator Energie entzogen (wird am Messinstrument angezeigt). So kann die Frequenz festgestellt werden.
  \end{description}
}
\card{33}{Erklären Sie die Funktionsweise eines HF-Wattmeters}{
  Das hochfrequente Signal wird direkt oder über Richtkoppler (richtungsabhängiges Abzweigen von elektromagnetischen Wellen aus einer Leitung) einem Diodengleichrichter zugeführt. 

  Bei konstantem Abschlusswiderstand wird Skala des Messwerks direkt in Watt kalibriert.
  Messen des SWR (Stehwellenverhältnis $1:\infty$).
  Kontrolle der Impedanz.
  Leistungsmesser für fwd. (hinlaufende) und refl. (rücklaufende) Welle auf der Basis von Diodengleichrichtern.
}
\card{34}{Erklären Sie die Funktionsweise eines Oszillografen (Oszilloskop)}{
  \small
  Mit Oszillografen kann der zeitliche Verlauf sinusförmiger oder nichtsinusförmiger Signale dargestellt und gemessen werden.

  \begin{itemize}
    \item Achsen: x = Zeit, y = Spannung
    \item In einer Kathodenstrahlröhre treffen gebündelte Elektronen auf einen Bildschirm und bringen ihn am Auftreffpunkt zum Leuchten
    \item Die Ablenkfrequenz kann eingestellt und an die Frequenz des darzustellenden Signals angepasst werden.
  \end{itemize}
}
\card{35}{Erklären Sie die Funktionsweise eines Spektrumanalysators}{
  \small
  Können mehrere Signale mit verschiedenen Frequenzen gleichzeitig in wählbaren Frequenzbereich dargestellt werden.

  \begin{itemize}
    \item Über eine Kathodenstrahlröhre erfolgt optische Darstellung der in einem Signal enthaltenen Frequenzen.
    \item Achsen: x = Frequenzen, y = Amplitude
    \item Damit lassen sich ein Frequenzbereich, das Nutzsignal und evtl. unerwünschte Aussendungen sowie deren Stärke messtechnisch erfassen.
  \end{itemize}
}
\card{36}{Begriff Demodulation}{
  \footnotesize
  Bei Demodulation wird das NF-Signal (Sprache, Daten) aus dem modulierten HF-Signal zurückgewonnen.
  Je nach Modulationsart ist Demodulator unterschiedlich aufgebaut und trägt verschiedene Bezeichnungen.

  \begin{description}
    \item[Frequenzmodulation (FM)]
      LC-Schwingkreis:
      Ratiodetektor, Quadraturdemodulator
    \item[Amplitudenmodulation (AM)]
      eine Diode ein RC-Glied für die Rückgewinnung des Nutzsignals ausreichend:
      Diodendemodulator, Synchrondetektor
    \item[Einseitenband Modulation (SSB)]
      wie AM Demodulation: zusätzlich Dazumischen des Trägers durch einen beat frequency oscillator (BFO):
      Produktdetektor
  \end{description}
}
\card{37}{Zeichnen Sie das Blockschaltbild eines Überlagerungsempfängers}{
  \footnotesize
  Komponenten:
  \begin{itemize}
    \item Antenne
    \item Bandpassfilter
    \item HF-Verstärker
    \item Mischer (Empfangsfrequenz mit VFO Frequenz)
    \item ZF Filter (Quarz, Bandpassfilter)
    \item ZF Verstärker
    \item Produktdetektor (SSB Demodulation, BFO-Frequenz liefert Träger)
    \item NF Verstärker (liefert AGC an ZF und HF Verstärker)
    \item NF Endstufe
  \end{itemize}
}
\card{38}{Was verstehen Sie unter Spiegelfrequenz und Zwischenfrequenz?}{
  \begin{description}
    \item[Spiegelfrequenz]
      ist die zweite unerwünschte Empfangsfrequenz eines Überlagerungsempfängers,
      da bei jeder Mischung Summen und Differenzfrequenzen entstehen.
      Unterdrückung durch Bandpassfilter im Eingang (lässt nur Empfangsfrequenz durch).
    \item[Zwischenfrequenz]
      ist die Frequenz,
      auf die das Empfangssignal in Überlagerungsempfänger mit Hilfe eines Lokaloszillators heruntergemischt wird.
  \end{description}
}
\card{39}{Erklären Sie die Kenngrößen eines Empfängers - Empfindlichkeit, intermodulationsfreier Bereich, Eigenrauschen}{
  \small
  \begin{description}
    \item[Empfindlichkeit]
      kleinste Signalpegel der noch empfangen werden kann (MDS = minimal detectable signal).
      definiert als das Signal, das mit einem SN Wert von \SI{3}{\dB} feststellbar ist.
      Meist Signalpegel von ca. \SI{0,2}{\micro\volt}.
    \item[Intermodulationsfreier Bereich]
      Abstand zweier gleich starker Signale, die ein Empfänger verkraften kann, ohne zu übersteuern
      (gute Werte $>$ \SI{90}{\dB})
    \item[Eigenrauschen (noise level)]
      Maß des Rauschsignals eigener Quellen, wenn kein Eingangssignal vorhanden ist
  \end{description}
}
\card{40}{Erklären Sie den Begriff des Rauschens - Auswirkungen auf den Empfang.}{
  \small
  Unregelmäßige therm. Elektronenbewegungen erzeugen in jedem Bauteil unregelmäßige Stromschwankungen, die als Rauschen (Noise) bezeichnet wird. Geringe Bandbreite = niedriger Rauschpegel.

  \begin{itemize} 
    \item Auf Gerätebauteile zurückzuführende Rauschquellen ergeben \emph{Eigenrauschen} (Abhilfe durch rauscharme Bauteile und Kühlung).
    \item \emph{Äußeres Rauschen} durch atmosphärisch-galaktisches Rauschen und dem ,,man made noise`` (technische Rauschquellen). Äußeres Rauschen ist frequenz- \& standort-abhängig.
  \end{itemize}
}
\card{41}{Mischer in Empfängern - Funktionsweise und mögliche technische Probleme}{
  Bauteil/Schaltung zur Mischung zweier Signale mit unterschiedlichen Frequenzen (Amplituden beeinflussen einander). Mischung der Empfangsfrequenz erfolgt mittels Oszillator zur Zwischenfrequenz.

  Es entstehen Summe und Differenz der beiden Frequenzen.
  Spiegelfrequenz muss schon am Eingang ausgefiltert werden (Bandpass),
  sonst Gefahr des Spiegelfrequenzempfangs.
}
\card{42}{Nichtlineare Verzerrungen - Ursachen und Auswirkungen}{
  \small
  \item
  Nichtlineare Verzerrungen = Intermodulation / Kreuzmodulation.

  \item
  Entstehen durch Aussteuerung einer Stufe in den nichtlinearen Kennlinienteil durch starke Signale im Empfangszweig. Vorstufe des Empfängers mischt unerwünschte Signale in den Empfangsbereich hinein (\emph{Geistersignale}).

  \begin{description}
    \item[Abhilfe] ,,Abschwächer`` vor dem Empfänger
    \item[In Sendern]
      Nichtlineare Verzerrungen als häufigste Ursache von unerwünschten Nebenaussendungen.
      Übersteuerung durch unsachgemäße Bedienung
  \end{description}
}
\card{43}{Empfängerstörstrahlung - Ursachen und Auswirkungen}{
  \begin{description}
    \item[Ursache]
      Jeder Oszillator ist Sender kleiner Leistung.
    \item[Auswirkung]
      Kleine Leistung kann störend strahlen
  \end{description}

  Oszillator muss vom Antenneneingang (Überlagerungsempfänger) gut entkoppelt werden.
  Entkoppelung erfolgt durch HF Vorverstärker, aktive Mischer, Bandfilter (nur Empfangssignal).
  Messung mit Spektrumanalysator am Antenneneingang bzw. Antenne am Spektrumanalysator zur Lokalisierung der Abstrahlung.
}
\card{44}{Mikrofonarten - Wirkungsweise}{
  \scriptsize
  \begin{description}[itemsep=-3pt,topsep=0pt]
    \item[Kohlemikrofon*]
      externe Stromversorgnung,
      Membran presst Kohlkörnchenschicht zusammen,
      Druck ändert elektrischen Widerstand
    \item[Kondensatormikrofon*]
      2 Platten; Abstand variiert mit Sprache
    \item[Elektretmikrofon*]
      Kunstharzmasse bildet Elektret, ändert beim Verformen die Ladung der Kapazität,
      Ausgangssignal hochohmig, daher Einbau eines Vorverstärkers
    \item[Dynamisches Mikrofon$^\dagger$]
      Membran mit beweglicher Spule verbunden,
      taucht in Magnetfeld eines Dauermagneten ein,
      induziert Wechselspannung 
    \item[Kristallmikrofon$^\dagger$]
      Kristalle aus Seignetsalz und Keramiken geben bei mechanischer Belastung elektrische Spannung ab,
      Piezoeffekt, Membran
  \end{description}%
  * \quad externe Stromversorgung \qquad
  $\dagger$ \quad interne Stromversorgung
}
\card{45}{Prinzip, Arten und Kenngrößen der Einseitenbandmodulation}{
  \footnotesize
  ausgehend von AM Signal,
  Unterdrückung von Träger und einem Seitenband (Filter- oder Phasenmethode)

  \begin{description}
    \item[Filtermethode] Quarz lässt nur Seitenband durch
    \item[Phasenmethode] SSB Signalerzeugung über Phasenschiebernetzwerk
  \end{description}

  Trägersignal max. \SI{50}{\percent}, je Seitenband max. \SI{25}{\percent} Leistung

  \begin{description}
    \item[Kenngrößen] Trägerunterdrückung, unterdrückt unerwünschtes Seitenband, Spitzenausgangsleistung
    \item[Vorteil] bessere Leistungsausbeute (Reichweite), halbe Bandbreite, weniger störanfällig (Fading)
  \end{description}
}
\card{46}{Prinzip, Arten und Kenngrößen der Pulsmodulation}{
  \small
  \begin{description}
    \item[Prinzip] Einzelne Impulse werden gesendet
    \item[Arten] Anwendung bei hohen Frequenzen (Ausnahme: Morsetelegrafie)
      \begin{description}
        \item[PAM] Pulsamplitudenmodulation
        \item[PDM] Pulsdauermodulation
        \item[PFM] Pulsfrequenzmodulation
        \item[PCM] Pulscodemodulation
      \end{description}
    \item[Kenngrößen] Amplitude, Dauer, Frequenzhub, Phasenhub, Codierung
  \end{description}
}
\card{47}{Erklären Sie die wichtigsten Anwendungen der digitalen Modulationsverfahren}{
  \begin{description}
    \item[FSK]
      (Frequenzumtastung, 2 definierte Frequenzen):
      RTTY, Packet Radio
    \item[PSK]
      (Phasenumtastung, Träger wird um $45^\circ$ oder $90^\circ$ verschoben, 2 oder 4 Zustände):
      PSK~31, Datenübertragung
    \item[QAM]
      (Quadrature Amplitudenmodulation, Kombination von Amplituden- und Phasenmodulation):
      digitales Fernsehen, Datenübertragung
  \end{description}
}
\card{48}{Erklären Sie die Begriffe CRC und FEC}{
  \begin{description}
    \item[CRC]
      Cyclic Redundancy Check:
      Mitsenden einer binären Prüfsumme,
      Empfänger berechnet selbst und vergleicht,
      wenn ungleich, Anforderung Wiederholung (ARQ = automatic repeat request)
    \item[FEC]
      Forward Error Correction:
      Mitsenden redundanter Information,
      erlaubt Korrektur von Fehlern bei Decodierung
  \end{description}
}
\card{49}{Prinzip und Kenngrößen der Frequenzmodulation}{
  Modulationssignal verändert die Grundfrequenz des Sendeoszillators
  Kenngrößen:
  \begin{description}
    \item[Frequenzhub in \SI{}{\kilo\Hz}] Änderung Trägersignal, üblich 5 kHz
    \item[Modulationsindex] Verhältnis Frequenzhub / Modulationsfrequenz.
  \end{description}
  Lautstärke liegt in Frequenzauslenkung des Trägers.
  Im Amateurfunk wird FM auf \SI{2}{\metre} und \SI{70}{\centi\metre}-Band verwendet.
}
\card{50}{Prinzip und Kenngrößen der Amplitudenmodulation}{
  \begin{description}
    \item[Prinzip]
      Modulationssignal verändert die Ausgangsleistung des Senders.
    \item[Kenngröße]
      Modulationsgrad \SIrange{0}{100}{\percent} (größer \SI{100}{\percent} führt zu Verzerrungen)
        Frequenz des Modulationssignals ergibt Bandbreite der Seitenbänder.
        Lautstärke liegt in Amplitude des Trägers.
        AFU auf KW praktisch nur mehr in SSB.
  \end{description}
}
\card{51}{Erklären Sie den Begriff Modulation (analoge und digitale Verfahren)}{
  Aufprägen eines niederfrequenten Signals auf einen hochfrequenten Träger

  \begin{description}
    \item[analog] Niederfrequentes Signal kann jeden Wert zwischen Maximum und Minimum annehmen
    \item[digital] Niederfrequentes Signal kann nur 2 Zustände annehmen: 0 oder 1.
  \end{description}

  Mathematisch betrachtet, Verfahren ist Addition oder Multiplikation.
}
\card{52}{Oszillatoren - Grundprinzip, Arten}{
  \small
  Ein Oszillator erzeugt ein Wechselspannungssignal gewünschter Frequenz und Kurvenform; ist ein Verstärker bei dem ein Teil des Ausgangsignals wieder an Eingang zurückgeführt wird. Arten:
  \begin{description}
    \item[VFO]
      (variable frequency osc.)
      durch abstimmbaren Schwingkreis
    \item[X(C)O]
      (xtal crystal osc.)
      Quarzoszillator nur in geringen Umfang zu verändern - hohe Güte \& Temperaturstabilität
    \item[VCO]
      (voltage controlled osc.)
      Spannungsgesteuerter Oszillator
  \end{description}
}
\card{53}{Erklären Sie den Begriff VCO}{
  \begin{description}
    \item[VCO] (voltage controlled oscillator)
      Spannungsgesteuerter Oszillator.
    \item[Aufbau]
      Dem frequenzbestimmenden LC (Resonanzschwingkreis) wird eine Kapazitätsdiode parallel geschaltet.
      An die Diode wird eine variable Gleichspannung angeschlossen.
      Diode ändert je nach Spannung die Kapazität, dadurch auch die Ausgangsfrequenz des Oszillators.
    \item[Anwendung] PLL Schaltung, Superhet
  \end{description}
}
\card{54}{Erklären Sie den Begriff PLL}{
  \small
  \begin{description}
    \item[PLL] phase locked loop
    \item[Aufbau]
      Ausgangsfrequenz eines VCO wird über Frequenzteiler einem Phasenvergleicher zugeführt.
      Referenzfrequenz kommt vom Quarzoszillator.
      Am Ausgang: veränderliche Gleichspannung, die die Kapazitätsdiode des VCO steuert,
      dadurch wird der VCO immer auf die Sollfrequenz eingestellt.
    \item[Anwendung]
      quarzstabile Frequenzen, wesentlich höher, als mit Quarz zu erzeugen.
      Andere Frequenzen durch Änderung des Teilungsverhältnisses.
  \end{description}
}
\card{55}{Erklären Sie den Begriff DSP}{
  \small
  \begin{description}
    \item[DSP] digital signal processing
    \item[Aufgabe]
      Realisierung der Aufgaben von Sendern, Empfängern, Oszillatoren, Verstärkern, Filtern, Mischer, etc.
      durch Digitaltechnik
      \begin{description}
        \item[Sampling des Analogsignals] Umwandlung mit ADC (analog digital converter) in Bits
        \item[Signal processing] Umwandlung mit DAC (digital analog) in ein analoges Signal
        \item[Anti-aliasing filter] Verhindern von zu hohen Frequenzen am Eingang
      \end{description}
  \end{description}
}
\card{56}{Erklären Sie die Begriffe sampling, anti aliasing filter, ADC/DAC}{
  \begin{description}
    \item[Sampling] Abtasten der Amplitude eines Signals in einer bestimmten Frequenz
    \item[Anti aliasing filter] Verhinderung am Eingang, dass zu hohe Frequenzen digitalisiert werden (Tiefpass)
    \item[ADC/DAC converter] digital zu analog bzw. analog zu digital
  \end{description}
}
\card{57}{Merkmale, Komponenten, Baugruppen eines Senders}{
  \small
  \begin{description}
    \item[Merkmale]
      Arbeiten meist nach Überlagerungsprinzip
    \item[Komponenten, Baugruppen]
  \end{description}
  \scriptsize
  \begin{itemize}
    \item Mikrofon, Verstärker, Oszillator
    \item Balance Modulator (liefert Seitenbänder aus AM vom Mikrofon)
    \item Quarzfilter (LSB oder USB)
    \item Mischer mit VFO auf HF
    \item Bandpass (nur HF, Spiegelfrequenzen wegfiltern)
    \item Verstärker: Treiber, Endstufe
    \item Anpassung an der Endstufe
    \item Tiefpass (Oberwellen wegfiltern)
    \item Antenne
  \end{itemize}
}
\card{58}{Zweck von Puffer- und Vervielfacherstufen, Aufbau}{
  \begin{description}
    \item[Pufferstufe] Entkopplung des Oszillators von den nachfolgenden Stufen
    \item[Aufbau] wie ein sehr schwach gekoppelter Verstärker
    \item[Vervielfacherstufe]
      Stark übersteuerte Verstärkerstufe erzeugt viele Oberwellen.
      Am Ausgang filtert ein Resonanzkreis die gewünschte Oberwelle aus,
      unterdrückt die anderen Oberwellen und die Grundwelle
  \end{description}
}
\card{59}{Aufbau einer Senderendstufe, Leistungsauskopplung}{
  \begin{description}
    \item[Senderendstufe]
      verstärkt das Signal auf die gewünschte Sendeausgangsleistung.
      Verstärkende Elemente: Röhren, Transistoren
    \item[Leistungsauskopplung]
      Transformation des Hochfrequenzwiderstandes der verstärkenden Elemente auf den Normwiderstand der Senderschnittstelle (\SI{50}{\ohm}), dadurch optimale Leistungsabgabe.
      Tiefpassfilter dient zur Oberwellenunterdrückung
  \end{description}
}
\card{60}{Anpassung eines Senderausgangs an eine symmetrische oder asymmetrische Antennenspeiseleitung}{
  Eine optimale Leistungsübertragung liegt vor,
  wenn Senderschnittstelle und Speiseleitung bezüglich Wellenwiderstand und Symmetrieeigenschaften übereinstimmen.
  Stimmen diese Kenngrößen nicht überein, treten \emph{Mantelwellen} auf. Daher:
  \begin{itemize}
    \item Transformieren (Widerstand, Anpassung)
    \item Symmetrieren (mittels Balun)
  \end{itemize}
  Mittels symmetrischen Antennentuner, Balun oder Mantelwellensperre.
}
\card{61}{Der Antennentuner, Wirkungsweise, 2 typische Beispiele}{
  \small
  \begin{description}
    \item[Antennentuner (Anpassung)]
      dient zur Resonanzabstimmung der Antenne.
    \item[Wirkungsweise]
      Optimal an der Antennenschnittstelle meistens aber bei Schnittstelle Senderausgang-Antennenkabel.
      Sender erhält dadurch geforderten Nennwiderstand (\SI{50}{\ohm}), dadurch erhält man die geforderte Nennleistung des Senders. Fehlt diese Anpassung: Schutzschaltung regelt Sendeleistung auf wenige Watt zurück.
    \item[Beispiele]
      \begin{itemize}
        \item TRX, ATU mit Balun, Dipol
        \item TRX, Tuner, Koaxkabel, Antenne
      \end{itemize}
  \end{description}
}
\card{62}{Antennenzuleitungen - Aufbau, Kenngrößen}{
  \small
  %Aufbau:
    \begin{description}
      \item[Symmetrische Speiseleitungen]
        Bandkabel, Paralleldrahtleitung,
        2 Leiter werden durch isolierende Abstandshalter geführt
      \item[Asymmetrische Speiseleitungen]
        Koaxialkabel, Konzentrische Anordnung von Innenleiter, Dielektrikum, Außenleitergeflecht, Außenisolation
    \end{description}
  Kenngrößen:
    \begin{itemize*}
      \item Impedanz
      \item Dämpfung
      \item Verkürzungsfaktor (Kabelkennwert)
      \item Belastbarkeit (alle Werte unabhängig v. Länge und Frequenz)
      \item Durchmesser, Gewicht, Krümmungsradius, \dots
    \end{itemize*}
}
\card{63}{Erklären Sie den Begriff Balun. Aufbau, Verwendung und Wirkungsweise}{
  \small
  \begin{description}\itemsep-1pt
    \item[Balun] \emph{bal}anced to \emph{un}balanced
    \item[Verwendung]
      Anpassen einer symm. Last an eine asymm. und umgekehrt
      (zB Koaxialkabel an Dipol)
    \item[Wirkungsweise]
      eines der beiden gegenphasigen Signale wird mit Verzögerungsleitung oder einen Übertrager gleichsinnig zum anderen gedreht und zu diesem parallel geschaltet. Dadurch sinkt die Impedanz auf $\frac14$ ab und es erfolgt Anpassung der Wellenwiderstände.
  \end{description}
  Wird nicht symmetriert, treten am Koaxialkabel Mantelwellen auf, Schirmwirkung geht verloren und Kabel strahlt wie eine Antenne.
}
\card{64}{Der Dipol - Aufbau, Kenngrößen und Eigenschaften}{
  \small
  \begin{description}
    \item[Dipol] Antenne aus 2 gleich langen Leiterhälften
    \item[Halbwellendipol] elektrische Gesamtlänge $\frac\lambda2$
    \item[Mittige Anspeisung]
      Widerstand \SI{50}{\ohm},
      dadurch symm. Anspeisung durch Koax und Balun leicht möglich
    \item[Kenngrößen]
      \begin{itemize*}
        \item Strahlungsdiagramm (Form einer 8, Strahlungsminima in der Antennenebene)
        \item Wellenwiderstand
        \item Gewinn
      \end{itemize*}
  \end{description}
  Alle linearen Antennenformen lassen sich auf Dipole (bzw. Kombinationen) zurückführen.
  Verwendete Formen: gestreckte Drahtdipole, abgewinkelte Dipole (inverted V)
}
\card{65}{Die Vertikalantenne - Aufbau, Kenngrößen und Eigenschaften}{
  \footnotesize
  Senkrecht zur Erdoberfläche angeordnete Antennen, deren Strahlung vertikal polarisiert ist.

  \begin{itemize}
    \item Viertelwellenstrahler stark verbreitet (fehlende Strahlerhälfte aus Erdnetz / Radials)
    \item Fußpunktwiderstand im Resonanzfall ca. \SI{30}{\ohm}
    \item Horizontale Charakteristik ergibt Rundstrahler
    \item Vertikale Charakteristik abhängig von Bodeneigenschaften
  \end{itemize}
  \begin{description}
    \item[Verwendung] Mobilantennen (Fahrzeug als Gegengewicht)
    \item[Kenngrößen] Wirkungsgrad, vertikaler Abstrahlwinkel, Bandbreite
  \end{description}
}
\card{66}{Gekoppelte Antennen - Aufbau, Kenngrößen und Eigenschaften}{
  \small
  \begin{description}
    \item[Aufbau]
      Verbindung mehrere Dipole über Koppelleitungen (Gruppenantenne)
    \item[Eigenschaften]
      Alle Dipole haben die gleiche Abstrahlphase, ausgeprägte Richtwirkung.
      Gewinnverdoppelung (\SI{3}{\dB}) bei jeder Dipolzahlverdoppelung.
      Reflektor hinter Gruppenantenne erhöhen Gewinn.
    \item[Kenngrößen]
      Frequenz(bereich), Impedanz, Gewinn, Öffnungswinkel (horizontal \& vertikal), Rück- und Seitendämpfung, Nebenkeulen, \dots
  \end{description}
}
\card{67}{Strahlungsdiagramm einer Antenne}{
  \small
  Zeigt die räumliche Verteilung des abgestrahlten Feldes (Energiedichte Verteilung).
  Zeigt den Unterschied einer Rundstrahl- und Richtantenne.

  \begin{description}
    \item[Bezugsfläche] Erdoberfläche
    \item[Charaktisierung] meist erkennbar an Horizontaldiagramm und Vertikaldiagramm
    \item[Kenngrößen]
      horizontaler Öffnungswinkel (\SI{3}{\dB} Abfall-Winkel),
      vertikaler Erhebungs-/Abstrahlwinkel,
      Öffnungswinkel,
      Hauptkeule/Nebenkeulen,
      Vor-/Rückwärtsverhältnis
  \end{description}
}
\card{68}{Die Yagi-Antenne - Aufbau, Kenngrößen und Eigenschaften}{
  \small
  \begin{description}
    \item[Aufbau]
      Ergänzung eines aktiv angespeisten resonanten Halbwellendipol durch 2 oder mehrere Halbwellenstrahler: Yagi (Uda) Antenne (einseitige Richtwirkung). 0 bis beliebig viele Reflektoren (längeres Element) und 1 Direktor (kürzeres Element).
    \item[Eigenschaften]
      Einseitige Richtwirkung, mehr Direktoren bedeutet mehr Richtwirkung, aber nicht unbegrenzt steigerbar (max. \SI{18}{\dB})
  \item[Kenngrößen]
    Frequenz(bereich), Impedanz, Strahlungsdiagramm, Gewinn, Vor/Rückverhältnis
  \end{description}
}
\card{69}{Breitbandantennen - Aufbau, Kenngrößen und Eigenschaften}{
  \small
  \begin{description}
    \item[Eigenschaften]
      Innerhalb eines definierten Frequenzbereichs,
      Antenneneigenschaften ändern sich nicht;
      insbesondere Fußpunktimpedanz (Schnittstellenwiderstand)
      (Bandbreiten von 1:2 bis 1:10 erzielbar)
    \item[Aufbau]
      Dicke Antennenelem. in Rohr- und Reusenform, Wider. zur Dämpfung, ausgeklügelte Kopplung
    \item[Kenngrößen]
      Bandbreite, Wirkungsgrad, \dots {\tiny (siehe Antenne)}
  \end{description}
  Mech. Grenzen durch Bauform,
  Belastung der Antenne, um linearen Stromfluss herbeizuführen (Verluste von bis zu \SI{50}{\percent}),
  aufwendige geometrische Bauform (LogPer)
}
\card{70}{Die Parabolantenne - Aufbau, Kenngrößen und Eigenschaften}{
  \begin{description}
    \item[Aufbau]
      parabolförmige Reflektorwand hinter Strahler (liegt im Brennpunkt),
    \item[Eigenschaften]
      ausgeprägte Richtwirkung und Rückwärtsdämpfung,
      Gewinn deutlich über \SI{30}{\percent}
    \item[Kenngrößen]
      Frequenz(bereich), Impedanz, Gewinn, Öffnungswinkel der Hauptkeule,
      Rückdämpfung, Nebenkeulen, Flächenwirkungsgrad, \dots
  \end{description}
}
\card{71}{Erklären Sie den Begriff Wellenwiderstand}{
  \small
  Kenngröße, die angibt mit welchem Widerstand eine Leitung abgeschlossen werden muss.

  \begin{itemize}
    \item Charakteristisch für hochfrequente Leitungen
    \item von L und C Belag abhängig
    \item HF-Speiseleitung ist fortgesetzte Kombination von Parallelkapazitäten und Reiheninduktivitäten
    \item Impedanz vom Durchmesserverhältnis zwischen Innen- und Außenleiter bleibt auf allen Längen konstant
  \end{itemize}
}
\card{72}{Stehwellen und Wanderwellen, Ursachen und Auswirkungen}{
  \small
  \begin{description}
    \item[Ursache Stehwellen]
      Bei Fehlanpassung wird Leistung ständig an beiden Enden reflektiert,
      Stehlwelle mit Spannungs-/Strommaximum in Abständen von $\frac\lambda2$
    \item[Ursache Wanderwelle]
      HF-Speiseleitung beidseitig impedanzrichtig abgeschlossen,
      nur Wanderwellen, Leistungstransport nur in eine Richtung
    \item[Auswirkung]
      Bei Fehlanpassung, Überlastung der Endstufe und zusätzlicher Leistungsverlust
  \end{description}
}
\card{73}{Strahlungsfeld einer Antenne, Gefahren}{
  \small
  \begin{itemize} 
    \item Beachten der einschlägigen Vorschriften der EU und nationale Normen und Rechtsvorschriften
    \item ÖNORM S~1120 (zukünftig ÖVE/ÖNORM E~8850) \\
          $\Rightarrow$ Grenzwerte für Exposition der Bevölkerung durch Elektromagnetische Felder (EMF)
    \item Techn. Maßnahmen zur Minderung der Gefahren:
  \end{itemize}
  \begin{itemize}[label=*] 
    \item Vergrößerung des Abstandes zur Antenne %(Montagehöhe)
    \item Absenkung/Vermeidung der Emission (QRP, Abschalten, Anordnung der Antenne)
    \item Beschränkung des Aufenthalts-/Expositionsdauer
  \end{itemize}
}
\card{74}{Aufbau und Kenngrößen eines Koaxialkabels}{
  \begin{description}
    \item[Aufbau]
      \begin{itemize*}
        \item zentraler Innenleiter aus Kupfer oder versilbert,
        \item Dielektrikum aus Kunststoff, Teflon, \dots,
        \item Außenleiter aus Kupfergeflecht, Folie, Festmantel,
        \item Kunststoffisolation
      \end{itemize*}
    \item[Kenngrößen]
      Leitungswellenwiderstand,
      Dämpfung in \SI{}{\dB}/\SI{100}{\metre} (frequenzabhängig),
      Schirmungsfaktor, Verkürzungsfaktor,
      kleinster Biegungsradius, Zugfestigkeit, \dots
  \end{description}
}
\card{75}{Erklären Sie den Begriff Dezibel am Beispiel der Anwendung in der Antennentechnik}{
  \small
  Dezibel in der Antennentechnik beschreibt das Verhältnis zweier Leistungen (oder Spannungen)
  und dient zum Vergleichen in Antennentechnik. Antenne mit \SI{6}{\dB} Gewinn über Dipol strahlt in Hauptstrahlrichtung die 4-fache Leistung des Dipols aus. 

  \item
  Äquivalent: \SI{13}{\dB} $\rightarrow$ 20-fache Leistung, \SI{3}{\dB} $\rightarrow$ 2-fach, \SI{10}{\dB} $\rightarrow$ 10-fach.

  \item
  Für das Spannungsverhältnis gilt: \SI{6}{\dB} $\rightarrow$ 2-fache Spannung, \SI{12}{\dB} $\rightarrow$ 4-fache, \SI{20}{\dB} $\rightarrow$ 10-fache Spannung
}
\card{76}{Was versteht man unter Richtantennen, Anwendungsmöglichkeiten}{
  \small
  Eine oder mehrere Vorzugsrichtungen im Strahlungsdiagramm

  \begin{description}
    \item[Eigenschaften]
      Sendeleistung wird gezielt gebündelt (Gewinn),
      Ausblenden von Störungen
    \item[Bauformen]
      Yagi, Dipolzeilen/flächen, logarithmisch periodische Antennen, V-Antennen, Rhombic Antennen
    \item[Kenngrößen]
      Frequenz(bereich), Gewinn, Öffnungswinkel, Rück-/Seitendämpfung, Nebenkeulen, Abstrahlwinkel (je höher, desto flacher)
  \end{description}
}
\card{77}{Welche Kenngrößen von Antennen kennen Sie und wie können sie gemessen werden?}{
  \begin{center}
    Sind dem Datenblatt zu entnehmen.
  \end{center}

  \vspace{15pt}
  \scriptsize
  \begin{tabular}{ccc}
  \hline\hline
    Resonanzfrequenz             & Dipmeter                 & \SI{}{\mega\Hz} \\
  \hline
    Fußpunktimpedanz             & Impedanzmessbrücke       & \SI{}{\ohm} \\
  \hline
    Gewinn,                      & Messsender,              & \SI{}{\dB}, dBi \\
    Strahlungsdiagramm           & Pegelmessgerät,          & \\
                                 & Referenzantenne          & \\
  \hline
    Bandbreite                   & Stehwellenmessgerät      & \SI{}{\kHz} \\
  \hline
    Max. zulässige Leistung      & resultiert aus Stärke \& & \SI{}{\watt} \\
                                 & Material der Elemente \& & \\
                                 & Bauteile                 & \\
  \hline\hline
  \end{tabular}
}
\card{78}{Dimensionieren Sie einen Halbwellendipol für f = \SI{3.6}{\mega\Hz} ; V = 0.97 (Werte sind variabel)}{
  \[
    \text{Länge} 
      = \frac{\text{Verkürzungsfaktor}\cdot 300}{2 \times \text{frequency (in MHz)}}
  \] \[
      = \frac{0.97 \cdot \SI{300}{\metre\per\second}}{2 \cdot \SI{3.6}{\mega\Hz}} = \SI{40.42}{\metre}
  \]
  Im Allgemeinen $\lambda = \frac{c}{f}$, aber Antennenlänge soll $\frac\lambda2$ betragen. Weiters nutzen wir die Praktikerformel $\SI[math-rm = \mathnormal, parse-numbers = false]{\lambda}{\metre} = \frac{300}{\SI[math-rm = \mathnormal, parse-numbers = false]{f}{\mega\Hz}}$. Verkürzungsfaktor hängt von Drahtstärke (je dicker, desto kleiner) und Isoliermantel ab.
}
\card{79}{Bestimmen Sie die effektive Strahlungsleistung bei folgenden Gegebenheiten: Senderleistung: \SI{200}{\watt}; Dämpfung der Antennenleitung: 6 dB/\SI{100}{\metre}; Kabellänge : 50 m; Gewinn: \SI{10}{\dB} (Werte sind variabel)}{
  effektive Kabel-Dämpfung = 3 dB
an der Antenne kommt demnach halbe
Leistung an = \SI{100}{\watt}
Gewinn = 10 fach
Effektive Strahlungsleistung = \SI{1000}{\watt}
Strahlungsleistung für einzuhaltende 
Grenzwerte wichtig!
}
\card{80}{Bestimmen Sie die effektive Strahlungsleistung bei folgenden Gegebenheiten: Senderleistung \SI{100}{\watt}; Dämpfung der Antennenleitung \SI{12}{\dB}/\SI{100}{\metre}; Kabellänge \SI{25}{\metre}; Rundstrahlantenne mit Gesamtwirkungsgrad von \SI{50}{\percent} (Werte sind variabel)}{
  \item
  effektive Kabeldämpfung = $-6\cdot (50/100) = \SI{-3}{\dB}$ \\
  daher Leistungsfaktor = $0.5$ \\
  daher Leistung an Antenne = $\SI{200}{\watt} \cdot 0.5 = \SI{100}\watt$ \\
  Antennengewinn als Leistungsfaktor = $\SI{10}{\dB} \Rightarrow 10$ \\
  $\text{ERP} = \SI{100}{\watt} \cdot 10 = \SI{1000}{\watt}$

  \item
  Strahlungsleistung für einzuhaltende Grenzwerte wichtig!
}
\card{81}{Langdrahtantennen - Aufbau, Kenngrößen und Eigenschaften}{
  \small
  \begin{description}
    \item[Aufbau]
      linear, länger als eine Wellenlänge
    \item[Kenngröße]
      {\footnotesize Länge, resonant?, Ausrichtung, Einspeisungsart}
    \item[Eigenschaften]
      Gewinn gegenüber Halbwellendipol,
      Strahlungsdiagramm zeigt Vorzugsrichtungen,
      Annäherung der Antennenachse
  \end{description}

  Beispiel: gestreckter Dipol (beide Äste gleich lang), Zeppelinantenne (Halbwellendipol, an einem Ende frei).
  Beide Formen erfordern Tuner.
  Antenne mit Koaxspeisung, aber ohne Zusatzanpassung, darf nicht beliebig lang sein.
  Lösung: Halbwellendipol mit Mitteleinspeisung (Imped. \SI{50}{\ohm})
}
\card{82}{Zweck von Radials / Erdnetz bei Vertikalantennen - Dimensionierung}{
  \small
  Zweck ist fehlende Dipolhälfte zu ersetzen, ist notwendig damit Antenne zu geschlossenen Stromkreis wird. Als Ersatz auch gut leitender Untergrund möglich.

  \begin{itemize}
    \item Mind. 20 radial verlaufende Drähte eingraben
    \item Im Zentrum verbunden, an einen Pol der Speiseleitung angeschlossen
    \item Anderer Pol der Leitung wird an einen Viertelwellenstrahler (Monopol) angeschlossen
    \item Besonders flacher Abstrahlwinkel, Lang- \& Mittelwelle
  \end{itemize}
}
\card{83}{Blitzschutz für Antennenanlagen}{
  Standrohr und deren Ableitungen (Antennenkabel) müssen über geeignete Komponenten an den Blitzschutz angeschlossen bzw. geerdet werden

  \begin{itemize}
    \item Ringerder, Banderder (\SI{3}{\metre} lang; \SI{0,5}{\metre} tief), Staberder (>\SI{1,5}{\metre})
    \item Arbeiten nur durch konzessionierte Blitzschutzfirma ausführen lassen
  \end{itemize}
}
\card{84}{Sicherheitsabstände bei Antennen}{
  \small
  \begin{itemize}
    \item auch nur bei Empfang
    \item elektrische und mechanische Sicherheit muss gewährleistet werden
    \item Errichter ist für alle Schäden haftbar
    \item mehrere Antennen auf einem Dach dürfen sich nicht behindern
    \item Ab einer gewissen Höhe als Bauwerk eingestuft: Baupolizeiliche Genehmigung
    \item Elektromagnetische Verträglichkeit (Umwelt)
    \item Strahlungsfeld evaluieren
  \end{itemize}
}
\card{85}{Erklären Sie den Begriff ,,elektromagnetisches Feld``. Kenngrößen?}{
  \begin{itemize}
    \item Änderung elektr. Feld = erzeugt magn. Feld
    \item Änderung magn. Feld = erzeugt elektr. Feld
    \item Maxwellsche Gesetz setzen Wechselstromkreis voraus
    \item Antenne als geschlossenen Stromkreis betrachten
    \item Kenngrößen: Ausbreitungsgeschwindigkeit, Ausbreitungsrichtung, Wellenlänge, Polarisation, Feldstärke
  \end{itemize}
}
\card{86}{Begriff elektrisches und magnetisches Feld; Abschirmmaßnahmen für das elektrische bzw. das magnetische Feld?}{
  \begin{description}\itemsep0pt
    \item[elektrisches Feld] bildet sich zwischen den Platten eines Kondensators, elektr. Feldstärke (\SI{}{\volt\per\metre})
    \item[magnetisches Feld] um einen stromdurchflossenen Leiter, magnet. Flussdichte (\SI{}{\tesla})
    \item[Abschirmmaßnahmen]
      Elektrische Abschirmung durch Faradayschen Käfig,
      Magnetische Abschirmung von Gleichfeldern unvollständig durch ferromagnetische Stoffe,
      Magnetische Abschirmung bei Wechselfeldern durch leitende Materialien (Kupferblech) (Kurzschlussgefahr)
  \end{description}
}
\card{87}{Erklären Sie den Begriff ,,EMV`` und dessen Bedeutung im Amateurfunk}{
  ,,Elektromagnetische Verträglichkeit``

  \begin{itemize}
    \item Beeinflussung anderer Kommunikationsanlagen (zB Telefonanlagen, Funkanlagen)
    \item Beeinflussung von elektrischen und elektronischen Geräten und Anlagen (zB Türmelder, Klingel, Lautsprecher)
  \end{itemize}
  Als störend bewertet, da Funktion der beeinflussten Anlagen bzw. Geräte beeinträchtigt.
}
\card{88}{Erklären Sie den Begriff ,,EMVU`` und dessen Bedeutung im Amateurfunk}{
  \small
  ,,Elektromagnetische Umweltverträglichkeit``

  \begin{itemize}
    \item Verhalten von biologischem Gewebe gegenüber elektromagnetischen Feldern
    \item Mögliche Gefährdung des Menschen: Biologisches Gewebe erwärmt sich durch Absorption der Felder
    \item zB Mikrowellenherd, Mobiltelefon (Langzeitfolgen?)
    \item Abhängig von der Frequenz von Wechselfeldern, kommt es zu Resonanz (Magnetresonanz)
    \item kritische Kenngröße: Abstand zur Strahlungsquelle (\SI{2,5}{\metre} bei $\lambda = \SI{10}{\metre}$ und $\SI{100}{\watt}$)
  \end{itemize}
}
\card{89}{Erklären Sie den Begriff ,,Trap``, Aufbau und Wirkungsweise}{
  \small
  ,,Fallen'' (engl. traps) können einen Dipol zu einer Mehrbandantenne machen.
  \begin{itemize}
    \item Aufbau als Parallelresonanzkreis
    \item sperrt für höhere Frequenz, wirkt für tiefere Frequenz als Verlängerungsspule
    \item {\footnotesize \SI{40}{\metre}/\SI{80}{\metre} Antenne: \SI{32,9}{\metre} \SI{7}{\mega\Hz} Trapabstand \SI{16,5}{\metre}}
    \item Anwendung: W3DZZ Antenne, Mehrband Yagi, VK2AOU Mehrband Quad Antenne
    \item Einsatz auch in Empfängern und Sendern als Sperrkreis, Unterdrückung unerwünschter Frequenzen
  \end{itemize}
}
\card{90}{Was versteht man unter einem Hohlraumresonator, Anwendung.}{
  \begin{description}
    \item[Hohlraumresonator]
      rechteckiger oder runder Hohlzylinder mit einer geeigneten HF-Ankopplung.
      Resonanz zur Verstärkung einer Welle,
      Resonanzentwicklung im GHz-Bereich
    \item[Anwendung]
      Verwendung als Schwingkreis oder Filter, zB Mikrowellenherd
  \end{description}
}
\card{91}{Funkentstörmaßnahmen im Bereich Stromversorgung der Amateurfunkstelle}{
  korrekte Verdrosselung und Abblockung der Netzzuleitungen kann Abfließen von HF in das Stromnetz verhindern
  \begin{itemize}
    \item Breitbandnetzfilter (Tiefpassfilter)
    \item Typische Werte:
      \begin{itemize}
        \item Induktivität \SIrange{10}{50}{\milli\henry}
        \item Kapazität \SIrange{10}{100}{\nano\farad}
      \end{itemize}
  \end{itemize}
}
\card{92}{Funkentstörmaßnahmen bei Beeinflussung durch hochfrequente Ströme und Felder}{
  \begin{itemize}
    \item Tiefpassfilter
    \item Ringkerndrosseln
    \item Ferritstabdrosseln

  %  \item Entkoppelung der Antennen
  %  \item Einbau von Tiefpassfiltern
  %  \item Verhinderung von HF-Einströmung in Lautsprechern und NF-Leitungen durch Ferritstabdrosseln
  %  \item Abschirmung des beeinflussten Gerätes
  \end{itemize}
  Unerwünschte Ausbreitung von HF durch Stromnetz, Speiseleitung, Antenne, Einströmung/ Einstrahlung
}
\card{93}{Was sind Tastklicks, wie werden sie vermieden?}{
  \item
  Tastklicks sind Störsignale, die auftreten, wenn Sendertastung eines Morsesignals zu hart ist (also rechteckförmig). Dabei kommt es auch zu einer Vergrößerung der Bandbreite.

  \item
  Störungen auf benachbarten Frequenzen als Folge.

  \item
  Kann durch RC-Glieder weicher gemacht werden (dadurch kleinere Bandbreite).
}
\card{94}{Erklären Sie die Begriffe: ,,Unerwünschte Aussendungen``, ,,Ausserbandaussendungen``, ,,Nebenaussendungen`` (spurious emissions)}{
  \small
  \begin{description}
    \item[Unerwünschte Aussendung]
      Störsignale zugeführt der Antennenspeiseleitung am Ausgang des Sende-Empfängers,
      samt zugehörigen Seitenbänder aus Modulationsprozess
    \item[Ausserbandaussendungen]
      zB nichtunterdrückte Oberwellen, unerwünschte Aussendungen in nicht für den Funkverkehr zugelassenene Frequenzbänder
    \item[Nebenaussendungen]
      \begin{itemize}
        \item Sendesignal ist Mischsignal und unerwünschtees Mischprodukt wird nicht korrekt ausgefiltert
        \item Selbsterregung einer Verstärkerstufe im Sender
      \end{itemize}
  \end{description}
}
\card{95}{Erklären Sie den Begriff: ,,Splatter`` - Ursachen und Auswirkungen}{
  \begin{itemize}
    \item verursacht durch Übermodulation/Übersteuerung bei AM/SSB Sendern (Fehlbedienung)
    \item Sender wird nicht linearer Zustand erreicht
    \item Störung benachbarter Frequenzen
    \item Ausgesteuert erhöhte Bandbreite, schlechte Verständlichkeit
    \item Vermeidung durch korrekte Bedienung oder Reparatur
  \end{itemize}
}
\card{96}{Erklären sie den Begriff ,,schädliche Störungen``}{
  \item
  Störungen, welche die Abwicklung des Funkverkehrs bei einem anderen Funkdienst, Navigationsfunkdienst,
Sicherheitsfunkdienst beeinträchtigt, behindert, oder wiederholt unterbricht

  \item
  AMF der in Übereinstimmung der Vorschriften wahrgenommen wird, kann auch von schädlichen Störungen betroffen sein.
}
\card{97}{Prinzipieller Aufbau einer Relaisfunkstelle und einer Bakenfunkstelle}{
  \begin{description}
    \item[Relais] Sender und Empfänger auf unterschiedlichen Frequenzen, gemeinsame Antenne, hochgelegener Standort, Empfangssignal moduliert den Sender daher große Reichweite mit UKW
    \item[Bake] Hochgelegener Standort, zur Beobachtung der Ausbreitungsverhältnisse
  \end{description}
}
\card{98}{Definieren Sie den Begriff ,,Senderleistung``}{
  \item
  Laut AFV~§~1:

  \item
  Sendeleistung ist die der Antennespeiseleitung zugeführte Leistung

  \item
  Messgröße ist Watt
}
\card{99}{Definieren Sie den Begriff ,,Spitzenleistung``}{
  \item
  \emph{Spitzenleistung} ist die Effektivleistung, die ein Sender während einer Periode der Hochfrequenzschwingung während der höchsten Spitze der Modulationshüllkurve unverzerrt der Antennenspeiseleitung zuführt

  \item Diese Spitzenleistung ist identisch mit dem Begriff \emph{PEP} (peak envelope power)
}
\card{100}{Definieren Sie den Begriff ,,belegte Bandbreite``}{
  \small
  Laut AFV~§~1: Bezeichnet die Frequenzbandbreite, bei der die unterhalb ihrer unteren und oberhalb ihrer
oberen Frequenzgrenzen ausgesendeten mittleren Leistungen \SI{0,5}{\percent} der gesamten mittleren Leistung einer gegebenen Aussendung betragen.

  \begin{itemize}
    \item Kenngröße: \SI{}{\kilo\Hz}
    \item \SI{0.5}{\percent} (= $\frac1{200}$) der gesamten mittleren Leistung = $\frac{1}{100} (-\SI{20}{\dB})$ davon die $\frac12$ somit \SI{-23}{\dB}.
    \item Die oberen und unteren Frequenzgrenzen ergeben sich aus dem Modulationsprozess (SSB)
  \end{itemize}
}
\card{101}{Definieren Sie den Begriff ,,Interferenz in elektronischen Anlagen``; beschreiben Sie Ursachen und Gegenmassnahmen}{
  \begin{itemize}
    \item Interferenz bedeutet Überlagern bzw. Störung (Ergebnis ist eine unerwünschte Aussendung)
    \item Durch unerwünschte, störende Aussendungen (meist mehrere) kann schädliche Störungen verursachen.
      Die Ursache ist im Aufbau/Konzept der Empfangsanlage zu suchen.
    \item Gegenmaßnahmen: selektive Eingangsfilter, hochwertige Filter im ZF Bereich
  \end{itemize}
}
\card{102}{Erklären Sie die Begriffe ,,Blocking``, ,,Intermodulation``}{
  \begin{description}\itemsep-7pt
    \item[Blocking]
      extrem starkes Fremdsignal abseits der Empfangsfrequenz
      übersteuert Vorstufe derart, dass kein Empfang schwächerer Signale möglich ist
    \item[Intermodulation]
      unbeabsichtigte Mischung in einer Empfängerstufe mit 2 oder mehren Signalen;
      Entstehung von unerwünschten Mischprodukten, Vortäuschung von nicht existenten Signalen;
      zu unterscheiden von unerwünschten Nebenausstrahlungen, die durch Intermodulation im Sender entstehen
  \end{description}
}
\card{103}{Welche Gefahren bestehen für Personen durch den elektrischen Strom?}{
  \begin{itemize}
    \item Stromschlag kann Verbrennungen, Herzflimmern und Herzstillstand verursachen
    \item je nach Hautfeuchtigkeit mehr oder weniger gute Leitfähigkeit
    \item Spannungen ab \SI{50}{\volt} (Effektivwert) gelten als gefährlich
  \end{itemize}
}
\card{104}{Was ist beim Betrieb von Hochspannung führenden Geräten zu beachten?}{
  \begin{itemize}
    \item Einbau nur mit allseitig geschlossenen Hochspannungskäfig mit Deckelschalter
    \item Entladewiderstände über Elektrolytkondensator
    \item Vor jedem Eingriff: Netzstecker ziehen, Entladen der Elkos abwarten
    \item \emph{Niemals} im eingeschalteten Zustand daran arbeiten!
  \end{itemize}
}
\card{105}{Definieren Sie die Gefahren durch Gewitter für die Funkstation und das Bedienpersonal, beschreiben Sie Vorbeugemassnahmen}{
  \begin{description}\itemsep0pt
    \item[Primärblitzschlags] Geht direkt in Antenne, durch meist hoch angebrachte Antennenanlage wahrscheinlich
    \item[Sekundärblitzschlag] schlägt in \SI{230}{\volt} Leitung ein und beschädigt durch induktive Spannungsspitzen angeschlossene Geräte
    \item[Vorbeugungsmaßnahmen] korrekter Blitzschutz, beim Herrannahen eines Gewitters Antennen erden, Antennenkabel vom Gerät trennen, Funkbetrieb einstellen
  \end{description}
}

\def\chap{Technische Grundlagen \class{3,4}}

\card{01}{In welchem Zusammenhang stehen die Größen Strom – Spannung - Widerstand in einem Stromkreis?}{
  Damit Strom fließen kann, müssen zwischen zwei Polen eine Spannung und eine leitende Verbindung vorhanden sein. Je höher die Spannung, umso mehr Strom fließt. Der Widerstand behindert die elektrische Ladung. Mehr Widerstand bedeutet bei gleicher Spannung, dass weniger Strom fließt.
  \begin{description}\itemsep0pt
    \item[Maßzahl] Ohm
    \item[Symbol] $R$
    \item[Formel] $R = \frac{U}{I}$
  \end{description}
}
\card{02}{Was versteht man unter einem Kurzschluß - wie entsteht er?}{
  Wenn der Widerstand eines Verbrauchers $0$ ist, kann so viel Strom fließen, dass die Leitungen oder die Stromquellen Schaden nehmen. Sicherungen trennen bei einem Kurzschluss den Stromkreis von der Stromquelle.
}
\card{03}{Nennen Sie Stromquellen}{
  \small
  \begin{description}
    \item[Primärbatterien] Spannung zwischen Polen entsteht durch einen chemischen Prozess. Strom kann entnommen werden. Entladung ist nicht umkehrbar.
    \item[Sekundärbatterien] Spannung zwischen Polen entsteht durch einen chemischen Prozess. Strom kann entnommen werden. Entladung ist umkehrbar (Ladevorgang).
    \item[230V Steckdose] liefert 50 Hz Wechselstrom
    \item[Kenngrößen]
      \begin{itemize*}
        \item Spannung
        \item Strombelastbarkeit
        \item Kapazität in Ah
      \end{itemize*}
  \end{description}
}
\card{04}{Kenngrößen einer Gleichstromquelle. Kenngrößen einer Wechselstromquelle - Gefahrengrenze?}{
  \small
  \begin{description}
    \item[Gleichstrom] Die Spannung ist konstant, Polarität verändert sich nicht.

      \begin{itemize*}
        \item Spannung
        \item Strombelastbarkeit der Quelle
        \item Kapazität in Ah (Batterie, Akkus)
      \end{itemize*}
    \item[Wechselstrom] Spannung und Polarität ändern sich laufend, Kurvendarstellung 

      \begin{itemize*}
        \item Spannung (Amplitude)
        \item Frequenz
        \item Kurvenform (Signalform)
        \item Strombelastbarkeit der Quelle
      \end{itemize*}
  \end{description}
  Die Gefahrengrenze liegt bei 25~V, Lebensgefahr besteht bei 40~V.
}
\card{06}{Nennen Sie die wichtigsten Eigenschaften von Ohm'schen Widerständen, Induktivitäten und Kapazitäten.}{
  \footnotesize
  \begin{description}
    \item[Widerstand] Hemmung entgegen Stromfluss. Abhängig von Material und Maßen des Leiters. Widerstand steigt mit Länge und abnehmendem Durchmesser des Leiters.
    \item[Spule] Einheit Henry $H$. Bei Gleichspannung ein ohmscher Widerstand. Bei Wechselspannung auch ein induktiver Blindwiderstand. Höhere Frequenz führt zu größerem Blindwiderstand
    \item[Ladungsspeicher] zwei gegenüberstehenden Metallplatten. Einheit Farad $F$. Nur bei Wechselspannung fließt Strom. Höhere Frequenz bedeutet kleinerer kapazitiver Blindwiderstand
  \end{description}
}
\card{07}{Was verstehen Sie unter dem Begriff ,,Fehlanpassung``?}{Eine Fehlanpassung liegt vor, wenn die Anpassungsbedingungen bei Strom-, Spannungs- und Leistungsanpassung nicht erfüllt sind.}
\card{08}{Was verstehen Sie unter dem Begriff ,,Transformation``?}{Transformation ist der allgemeine Begriff für ,,Wandlung`` (zB. Spannungstransformation, Impedanztransformation). Auf- oder Abwärtstransformation von Wechselspannungen in der Stromversorgungs-, Niederfrequenz- und Hochfrequenztechnik.}
\card{09}{Prinzipieller Aufbau eines Kommunikationssystems. Erläutern Sie die Wirkungsweise von Mikrophon und Lautsprecher bzw. Kopfhörer.}{
  \small
  \begin{itemize*}
    \item Signal- Eingabegerät (Mikrophon)
    \item Sender
    \item Antennenanpassgerät
    \item Antenne
    \item Empfänger
    \item Signal Ausgabegerät (Kopfhörer)
  \end{itemize*}
  \vspace{10pt}

  Ein Mikrophon ist ein Schallwandler, der Schall in elektrische Spannungsänderungen als Signal umwandelt. Ein Wandler - gekoppelt mit einer Membran - generiert Tonfrequenz-Wechselspannung oder eine pulsierende Gleichspannung. Ein Lautsprecher ist ein Wandler der elektrische Signale in Schall (Ton) umwandelt. Tonerzeugung in für Menschen hörbaren Frequenzbereichen.
}
\card{11}{Prinzipieller Aufbau eines Senders}{
  \small
  \begin{minipage}{0.4\textwidth}
  \begin{itemize}
    \item Oszillator (CO oder VFO)
    \item Modulator
    \item Pufferstufe
  \end{itemize}
  \end{minipage}
  \begin{minipage}{0.5\textwidth}
  \begin{itemize}
    \item Frequenzvervielfacher
    \item Treiber
    \item Endstufe
  \end{itemize}
  \end{minipage}
  \vspace{10pt}

  Moderne Sender arbeiten nach dem Überlagerungsprinzip, allerdings verläuft der Signalweg in umgekehrte Richtung. Viele Baugruppen sind für das Senden und Empfangen nutzbar, deshalb ist dieses Konzept in Sendeempfängern (,,Transceiver``) verbreitet.
}
\card{12}{Funktionsprinzip des Oszillators}{
  Ein Oszillator erzeugt ein Wechselspanungssignal gewünschter Frequenz und Kurvenform.
  Jeder Oszillator ist ein Verstärker, bei dem ein Teil des Ausgangssignals wieder an den Eingang zurückgeführt wird. Dadurch kommt es zur Selbsterregung (Rückkopplung). Befindet sich im Rückkopplungsweg ein frequenzbestimmtes Bauteil (Filter), meist ein Schwingkreis (oder Quarz), so kann Selbsterregung nur auf dessen Resonanzfrequenz stattfinden.
}
\card{13}{Prinzipieller Aufbau eines Empfängers}{
  In einem Empfänger wird das NF-Modulationssignal aus dem modulierten HF-Signal zurückgewonnen.
  Die einfachsten Bauweisen bestehen aus einem Filter, HF-Verstärker, Demodulator, NF-Verstärker.
  Demodulator bezeichnet eine Baugruppe, die der Wiedergewinnung des Modulationssignals aus dem HF-Signal dient. Je nach Modulationsart ist der Demodulator unterschiedlich aufgebaut.
}
\card{14}{Prinzip des Überlagerungsempfängers. Was verstehen Sie unter dem Begriff Zwischenfrequenz?}{
  Bandfilter; Verstärkung im HF-Verstärker; Signale werden im Mischer mit Signal eines VFO gemischt; Filter wird ZF herausgefiltert und zu ZF Verstärker; Produktdetektor erfolgt Mischung mit Signal des BFO; Aus Mischprodukt wird NF-Signal verarbeitet. Über NF Verstärker und NF Endstufe zu Lautsprecher.
  Mischung von zwei HF-Signalen, entstehen 2 neue Signale (Summe oder Differenz). Ein Mischprodukt kann gefiltert werden und weiter verarbeitet- Zwischenfrequenz} % TODO
\card{16}{Was verstehen Sie unter dem Begriff Modulation?}{
  Modulation ist ein zentraler Begriff jeder technischen Form von Nachrichtenübertragung. Man muss zwischen dem Träger, der dauernd ausgesandt wird (zB. elektromagnetische Strahlung), und dem eigentlichen Signal, das mittels des Trägers übertragen werden soll, unterscheiden. Modulation bezeichnet den Vorgang, bei dem einen hochfrequenten Träger ein NF Signal aufgeprägt wird. Es gibt analoge und digitale Verfahren der Modulation.}
\card{17}{Kenngrößen der Amplitudenmodulation}{
  \begin{description}
    \item[Modulationsgrad] \hfill{} \\ (NF-Amplitude / HF-Amplitude) $\cdot$ 100 (\%)
    \item[Bandbreite]
      2~fm wobei fm die maximale zu übertragende Frequenz des Modulationssignales ist.
      Im Amateurfunk wird die Amplitudenmodulation auf den Kurzwellenbändern benützt.
  \end{description}
}
\card{18}{Kenngrößen der Frequenzmodulation}{
  \begin{description}
    \item[Frequenzhub] die maximale Ablenkung der Trägerfrequenz von der Grundfrequenz in kHz,
      im Amateurfunk: 5~kHz
    \item[Modulationsindex]
      Frequenzhub (kHz) / Modulationsfrequenz (kHz)
      Im Amateurfunk wird die Frequenzmodulation auf den 2~m und \SI{70}{\centi\metre} Bändern benützt.
      Der Frequenzhub beträgt in der Regel 5~kHz. Die Modulationsfrequenz beträgt 3~kHz.
      Modulationsindex von $\frac53 = 1,7$
  \end{description}
}
\card{19}{Definieren Sie den Begriff ,,belegte Bandbreite``. Arten und Vorteile der Einseitenbandmodulation?}{
  \small
  Jene Frequenzbandbreite, bei der die unterhalb und oberhalb ihrer Frequenzgrenzen ausgesendeten mittleren Leistungen \SI{0,5}{\percent} der gesamten mittleren Leistung einer gegebenen Aussendung betragen.
  Der Vorteil der Einseitenbandmodulation liegt in der weit günstigeren Leistungausbeute und der halben Bandbreite. Beides ergibt eine geringere Störanfälligkeit der Signalübertragung.

  Methoden:
  \begin{itemize}
    \item Filtermethode
    \item Phasenmethode
  \end{itemize}
}
\card{21}{Begriff Dezibel (Werte fragen: zB 3~dB, 6~dB, \SI{10}{\dB}, \SI{30}{\dB} Leistungssteigerung)}{
  {\small
    Dezibel ist ein logarithmisches Maß für das Verhältnis von zwei gleichartigen Leistungsgrößen
    $P_1$ und $P_2$ bzw. Spannungsgrößen $U_1$ und $U_2$.}

  {\scriptsize
    \begin{minipage}{0.45\textwidth}
      \vspace{5pt} Leistungsverhältnisse in dB\vspace{-8pt}
      \begin{description}\itemsep0pt
        \item[3~dB] 2fach
        \item[6~dB] vierfach
        \item[\SI{10}{\dB}] zehnfach
        \item[\SI{13}{\dB}] 20-fach
        \item[\SI{20}{\dB}] 100-fach
        \item[-3~dB] halb
        \item[-\SI{10}{\dB}] ein Zehntel
      \end{description}
      \end{minipage}
      \begin{minipage}{0.5\textwidth}
      Spannungsverhältnisse in dB:
      \begin{description}\itemsep0pt
        \item[6~dB] doppelte Spannung
        \item[\SI{12}{\dB}] vierfache Spannung
        \item[\SI{20}{\dB}] zehnfache Spannung
        \item[-6~dB] halbe Spannung
      \end{description}
    \end{minipage}
  }
}
\card{22}{Was ist eine Diode - Wirkungsweise, Verwendung?}{
  Eine Diode ist ein Halbleiterbauelement mit einem P-N Übergang. Die P-Schicht bildet die Anode, die N-Schicht die Kathode. Die Anwendung erfolgt als Gleichrichter, da Strom nur in einer Richtung fließen kann.
  \begin{description}
    \item[Durchlassrichtung] +Pol der Stromquelle an der Anode
    \item[Sperrrichtung] +Pol der Stromquelle an der Kathode (durch Ring gekennzeichnet)
  \end{description}
}
\card{23}{Was ist ein Transistor - Wirkungsweise, Verwendung?}{
  \small
  Ist ein Halbleiterbauelement, aus zwei N-Leitern, und dünnen Schicht eines P-Leiters, Emitter-Basis-Kollektor. Zwischen Basis, Emitter und Kollektor bilden sich zwei Sperrschichten. Weil Basis schwach dotiert ist, können Elektronen bei fließendem Basisstrom auch die B-K Sperrschicht überwinden und über Kollektor abfließen. Transistor verhält sich wie elektrisch gesteuerter Widerstand zwischen E und K.

  \begin{itemize}\itemsep-1pt
    \item NF/HF Verstärker
    \item Schalter
    \item Oszillatoren
  \end{itemize}
}
\card{24}{Was versteht man unter ,,AGC`` und ,,AFC``? Erklären Sie die Empfängerkenngrößen - Empfindlichkeit, Eigenrauschen, Empfangsmischprodukte}{
  \footnotesize
  \begin{description}
    \item[AGC] Lautstärke des NF-Signals eines Empfängers konstant gehalten. Notwendig, da Amplituden von Antenne kommende Signale Bereich von \SI{120}{\dB} übersteigen können.
    \item[AFC] Aus FM Demodulator Nachstimmspannung gewonnen, zur Nachstimmung der Oszillator-Frequenz,Schwankungen Empfangsfrequenz ausgeglichen.
  \end{description}
  \begin{enumerate}
    \item kleinster Signalpegel Empfangen werden kann
    \item Rauschquellen aller Bauteile, kein Eingangssignal
    \item Empfangsfrequenz gemischt- 2 Mischprodukte entstehen
  \end{enumerate}
}
\card{26}{Was versteht man unter dem S/N - Verhältnis?}{
  Das Zahlenverhältnis von Signalpegel zu Rauschpegel. S/N wird in dB angegeben und auch zur Messung der Grenzempfindlichkeit von Empfängern benützt.
  Ein S/N von 3~dB bedeutet, dass die Signalamplitude 1,4~mal größer als die Rauschamplitude ist.
}
\card{27}{Erklären Sie die Begriffe ,,digital`` und ,,analog``.}{
  Ein analoges Signal kann zwischen den Spitzenwerten jeden beliebigen Zwischenwert annehmen. Die Verarbeitung setzt Linearität voraus. Lautsprecher und Kopfhörer benötigen analoge Signale.

  Digitale Signale weisen nur zwei (binäre; 0 oder 1) Spannungszustände auf und keine Zwischenwerte. Zur Verarbeitung ist Linearität nicht erforderlich. Nichtlinearität ist sogar von Vorteil. Beispiel Lichtschalter: ,,An`` oder ,,Aus``}
\card{28}{Was versteht man unter der Ausgangsleistung, was unter der Verlustleistung?}{
  Die Ausgangsleistung ist jene Leistung, die ein Sender an eine definierte Schnittstelle abgibt (Sendeausgangsbuchse, meist 50~Ohm). Durch den nicht \SI{100}{\percent}igen Wirkungsgrad eines Senders muss der Sender bei einer vorgegebenen Ausgangsleistung mehr Energie zugeführt werden, als er abgeben kann. Die Differenz zwischen zugeführter und abgegebener Leistung (Ausgangsleistung) wird als Verlustleistung bezeichnet.
}
\card{29}{Was versteht man unter der Strahlungsleistung? (Beispiel vorgeben, zB. Sender mit \SI{10}{\watt} Ausgangsleistung; Antennenkabel mit 3 dB Dämpfung; Antenne mit \SI{10}{\dB} Gewinn)}{
  \small
  Die \emph{effektive Strahlungsleistung} ergibt sich aus der in eine Sendeantenne eingespeisten Leistung, vermehrt um den Antennengewinn in dB. ERP bezieht sich auf einen Halbwellendipol ($\Rightarrow$ dBd). Bezieht man den Antennengewinn auf den Isotropstrahler ($\Rightarrow$ dBi), spricht man von EIRP (Watt):
  \[ \text{EIRP} = \text{ERP} \cdot 1,64 \]
  \[ \text{ERP} = \SI{10}{\watt}- 3~dB + \SI{10}{\dB} = 10 \cdot 0,5 \cdot 10 = \SI{50}{\watt} \]
  Die Strahlungsleistung beträgt $\text{ERP} = \SI{50}{\watt}$
  \[ \text{EIRP} = 50 \cdot 1,64 = \SI{82}{\watt} \]
}
\card{30}{Begriff Speiseleitung (Antennenzuleitung) - Kenngrößen?}{
  \footnotesize
  \begin{description}
    \item[Symmetrische Speiseleitung]
      Zweidrahtleitungen (Paralleldrahtleitung).
      2 Leiter mit isolierendem Abstandshalter.
    \item[Asymmetrische Speiseleitung]
      Koaxialkabel. Konzentrische Anordnung Innenleiter, Dielektrikum, Außenleitergeflecht,
      Außenisolation
  \end{description}
  \begin{description}
    \item[Hohlleiter] Rechteckige / runde Rohre ohne Innenleiter (Verwendung im GHz-Bereich).
    \item[Elektrische Kenngrößen]
      \begin{itemize*}
        \item Impedanz
        \item Dämpfung
        \item Verkürzungsfaktor
        \item Belastbarkeit
      \end{itemize*}
    \item[Mechanische Kenngröße]
      \begin{itemize*}
        \item Durchmesser
        \item Gewicht
        \item Zugfestigkeit
      \end{itemize*}
  \end{description}
}
\card{31}{Auswirkung(en) des Stehwellenverhältnisses (SWR)?}{
  Bei Fehlanpassung wird ein Teil der Leistung am fehlangepassten fernen Ende reflektiert, läuft zurück und wird am nahen Ende teilweise reflektiert. Die Überlagerung von hin- und rücklaufenden Wellen führt zu Stehwellen. Es kommt zur Überlastung der Endstufe und zu einem zusätzlichen Leistungsverlust auf der fehlangepassten Leitung. Die Reflektionsverluste bei hohem SWR sind Verluste auf realen Leitungen.
}
\card{32}{Kenngrößen einer Antenne am Beispiel des Dipols}{
  \begin{itemize}
    \item Wellenwiderstand im Speisepunkt: ca. 50~Ohm, Speisung mit Koaxialkabel und Balun
    \item Strahlungsdiagramm: hat die Form einer Acht, d.h. Strahlungsmaxima quer zur Antennenachse, axiale Minima
    \item Gewinn: 2,\SI{15}{\dB} in Hauptstrahlrichtung
    \item Im Amateurfunk werden häufig gestreckte und abgewinkelte Dipole verwendet
  \end{itemize}
}
\card{33}{Vertikalantenne - Eigenschaften}{
  Vertikalantennen sind senkrecht zur Erdoberfläche angeordnete Antennen, deren Strahlung vertikal polarisiert ist. Im Resonanzfall zeigen Viertelwellenstrahler einen Fußpunktwiderstand von etwa 30~Ohm. Das horizontale Strahlungsdiagramm zeigt die Charakteristik eines Rundstrahlers, die vertikale Charakteristik ist stark von den umgebenden Untergrundeigenschaften abhängig. Werden als Mobilantennen verwendet.}
\card{34}{Die Yagi-Antenne - Aufbau, Eigenschaften, Kenngrößen}{
  Form der Richtantenne im KW/UKW Bereich.
  Resonanter Halbwellendipol wird durch zwei oder mehrere Elemente ähnlicher Länge ergänzt. Längeres Element als Reflektor, kürzere als Direktor bezeichnet. Neben Reflektoren kann man beliebig viele Direktoren verwenden. Yagi-Antenne zeigt eine einseitige Richtwirkung, Bündelung Richtung kürzeren Elemente. Mehr Direktoren - größere Richtwirkung:

  \begin{itemize*}
    \item Frequenz 
    \item Impedanz 
    \item Gewinn (dB) 
    \item Strahlungsdiagramm 
    \item Vor/Rückverhältnis
  \end{itemize*}
}
\card{35}{Dipolkombinationen (Zeilen, Spalten)}{
  Einfacher Dipol-Achter Charakteristik. Kombination von Dipolen untereinander (Spalten) oder nebeneinander (Zeilen) kann Antennencharakteristik verändern - Gewinn steigt. Kombiniert man Spalten und Zeilen zu Antennenfläche - erfolgt Strahlungseinzug nicht nur einer Ebene, sondern räumlich entsteht Diagramm einer ,,Doppelzigarre``. Diagrammform und Gewinn vom Abstand Dipole untereinander, Verhältnis der Ströme und Phasenwinkel zwischen Strömen abhängig.
}
\card{36}{Die Parabolantenne - Aufbau, Eigenschaften, Kenngrößen}{
  \small
  Im UKW/UHF Bereich verwendet.
  Hinter Strahler Parabolspiegel aus Metall angebracht. Durchmesser des Spiegels muss gegenüber Wellenlänge groß sein. Strahler im Brennpunkt des Spiegels angebracht. Oft Strahler selbst eine Richtantenne die auf den Spiegel zeigt. Parabolantenne zeigt ausgeprägte Richtwirkung. Strahlungskeule nur Winkelgrad, Ausrichtung muss sehr präzise sein.

  \vspace{5pt}
  \begin{minipage}{0.5\textwidth}
    \begin{itemize}
      \item Frequenz 
      \item Gewinn 
      \item Strahlungsdiagramm 
    \end{itemize}
  \end{minipage}
  \begin{minipage}{0.49\textwidth}
    \begin{itemize}
      \item Öffnungswinkel 
      \item {\footnotesize Rück/Seitendämpfung}
    \end{itemize}
  \end{minipage}
}
\card{37}{Mobilantennen - Aufbau, Eigenschaften, Kenngrößen, Montageort}{
  \small
  Verbreitet sind Viertelwellenstrahler, die aus einem Element bestehen. Zum Dipol wird die fehlende Hälfte durch Gegengewicht, zB. Fahrzeugkarosserie, ersetzt. UKW-Bereich Verlängerung nicht nötig. Im KW-Bereich induktiv verlängerte Antennen. Resonanzfall zeigen sie Fußpunktwiderstand ca. 30~Ohm. Horizontale Strahlungsdiagramm-Charakteristik Rundstrahlers, vertikale Charakteristik-Untergrundeigenschaften abhängig.

  \begin{minipage}{0.45\textwidth}
    \begin{itemize}
      \item Frequenz
      \item Gewinn
      \item Bauhöhe
    \end{itemize}
  \end{minipage}
  \begin{minipage}{0.5\textwidth}
    \begin{itemize}
      \item Gegengewicht
      \item Bandbreite
    \end{itemize}
  \end{minipage}
}
\card{38}{Grundausrüstung einer Amateurfunkstelle für Sprechfunk (Komponenten)}{
  \footnotesize
  \begin{minipage}{0.49\textwidth}
    \begin{itemize}
      \item Mikrofon
      \item PC mit Soundkarte (wahlweise zur Logbuchführung)
      \item Leistungsverstärker (wahlweise im Rahmen der Vorschriften)
      \item Antennentuner (wahlweise nach technischen Erfordernissen, vornehmlich auf Kurzwelle)
    \end{itemize}
  \end{minipage}
  \begin{minipage}{0.5\textwidth}
    \begin{itemize}
      \item Sender / Empfänger
      \item Sendeantenne / Empfangsantenne
      \item Lautsprecher / Kopfhörer
      \item Mess- und Kontrollgeräte, Blitzschutz (nach Maßgabe der geltenden Vorschriften)
    \end{itemize}
  \end{minipage}
}
\card{39}{Grundausrüstung einer Amateurfunkstelle für Packet Radio}{
  \begin{itemize}
    \item PC mit Soundkarte
    \item Modem / Controller
    \item Sender / Empfänger
    \item Leistungsverstärker (wahlweise im Rahmen der Vorschriften)
    \item Sendeantenne / Empfangsantenne
    \item Mess- und Kontrollgeräte, Blitzschutz (nach Maßgabe der geltenden Vorschriften)
  \end{itemize}
}
\card{40}{Grundausrüstung einer Amateurfunkstelle für ATV-Betrieb}{
  \begin{itemize}
    \item TV Kamera
    \item Sender / Empfänger
    \item Leistungsverstärker (wahlweise im Rahmen der Vorschriften)
    \item Sendeantenne / Empfangsantenne
    \item TV Monitor
    \item Mess- und Kontrollgeräte, Blitzschutz (nach Maßgabe der geltenden Vorschriften)
  \end{itemize}
}
\card{41}{Was versteht man unter Betriebserde; was unter Blitzschutzerde?}{
  Die \emph{Betriebserde} dient der Schutzmaßnahme (FI-Schalter, Nullung) und darf nicht für die Blitzableitung verwendet werden.

  Die \emph{Blitzschutzerde} stellt eine Schutzmaßnahme gegen Blitzeinwirkungen dar. Diese ist regelmäßig auf Funktionstüchtigkeit zu überprüfen. Neben den äußeren Blitzschutz des Gebäudes und der Antennenanlage sind die Antennenzuleitungen bei Beendigung des Funkbetriebes zu erden, daher mit dem Gebäudeblitzschutz zu verbinden.
}
\card{42}{Was versteht man unter BCI, TVI?}{
  \begin{description}
    \item[BCI] Störungen des Rundfunkempfanges durch eine andere Funkstelle. BCI wird durch Einstrahlung in die Empfangsantennenanlage, die Antennenzuleitung oder direkte Einstrahlung in den Rundfunkempfänger verursacht.
    \item[TVI] Störungen des Fernsehempfanges. Auch hier erfolgt die Einstrahlung in die Antennenanlage, die Zuleitungen oder direkt in den Fernsehempfänger. Besonders FS-Verstärkeranlagen und Hausverteiler sind gegen Einstrahlung anfällig.
  \end{description}
}
\card{43}{Maßnahmen gegen BCI, TVI?}{
  Gegen BCI und TVI richten sich die notwendigen Maßnahmen nach der Ursache der Störung.
  Grundsätzlich ist die Amateurfunkstelle so zu errichten und zu betreiben, dass Störungen anderer Funkdienste vermieden werden.
  Dies wird durch eine entsprechend ober- und nebenwellenfreies Sendesignal und der Einhaltung der zulässigen Sendeleistung sichergestellt.
}
\card{44}{Was versteht man unter dem ``SQUELCH`` - wozu dient er?}{
  Unter \emph{Squelch} versteht man eine Rauschsperre bei FM-Empfängern, wenn kein HF-Signal empfangen wird. Der NF-Verstärker wird ,,stumm`` geschaltet, wenn das Eingangssignal unter einer gewissen Schwelle (einstellbar am Gerät) liegt.
}
\card{45}{Wie bestimmt man die Resonanzfrequenz einer Antenne?}{
  Die \emph{Resonanzfrequenz} einer Antenne wird mit dem \emph{Griddipmeter} bestimmt.
  Dabei nähert man sich dem zu untersuchenden Schwingkreis mit der Koppelspule des Messgerätes an und durch Verändern der Oszillatorfrequenz des Griddipmeters wird diesem bei Resonanz mit dem Prüfling Energie entzogen. Das kann an einem Messinstrument (Rückgang des Gitterstroms) abgelesen werden. Somit kann die Frequenz festgestellt werden.
}
\card{46}{Was ist ein SWR-Meter, wo und wie wird es eingesetzt?}{
  Unter einem \emph{SWR-Meter} versteht man ein Messgerät zur Messung von Stehwellen. Das SWR wird in die Antennenzuleitung unmittelbar nach dem Antennenausgang eingeschliffen. Mit Hilfe des SWR-Meters kann festgestellt werde, ob auf der Antennenleitung stehende Wellen auftreten, daher der Antennenfußpunktwiderstand nicht mit dem Wellenwiderstand des Antennekabels übereinstimmt.
  Das SWR-Meter wird zur Abstimmung eines Antennenanpassgerätes benötigt.
}
\card{47}{Was versteht man unter einem ``Antennen-Tuner``?}{
  Der Antennentuner sitzt idealerweise an der Antennenschnittstelle und dient der Transformation der Kabelimpedanz auf die Impedanz des Antennenspeisepunktes.
}
\card{48}{Was versteht man unter ``Dopplershift``?}{
  Auf Grund der großen orbitalen Geschwindigkeit eines Satelliten ändern sich die Uplink und Downloadfrequenzen für die Bodenstation während seines Überflugs. Dieses Phänomen wird als \emph{Dopplershift} (auf Basis des Doppler-Effekts) bezeichnet.}
\card{49}{Komponenten einer Amateurfunkstation für Satellitenfunk}{
  \footnotesize
  \vspace{2pt}
  \begin{minipage}{0.43\textwidth}
    \begin{itemize}
      \item Mikrofon
      \item Sende-/Empfangsantenne
      \item Lautsprecher
      \item Leistungsverstärker (im Rahmen der Vorschriften)
    \end{itemize}
  \end{minipage}
  \begin{minipage}{0.56\textwidth}
    \begin{itemize}
      \item Sender/Empfänger
      \item PC mit Soundkarte (Bahndatenberechnung und Steuerung der Frequenz)
      \item Mess- und Kontrollgeräte, Blitzschutz (nach geltenden Vorschriften)
    \end{itemize}
  \end{minipage}

  \vspace{10pt}
  Für Satellitenfunk werden eine nachführbare Richtantennenanlage und ein Antennenvorverstärker benötigt, der unmittelbar an Antennenanlage montiert werden soll und bei Sendebetrieb zu schützen ist.
}
\card{50}{Abstrahlung und Ausbreitung elektromagnetischer Wellen, Feldstärke?}{
  \small
  HF-Schwingungen breiten sich in Leitern als Leitungswellen aus. Öffnet man den Leiter, beginnt er elektromagnetische Wellen abzustrahlen. Diese Leitungswellen gehen in Freiraumwellen über. Das auftretende Feld ist ein elektromagnetisches Feld. Dieses Feld wird beschrieben durch:
  \begin{itemize}
    \item elektrischen Feldanteil
    \item Frequenz des Wechselfeldes (in Hz)
    \item die elektromagnetische Feldstärke (in V/m)
    \item die Polarisation des elektrischen Feldvektors (als Feldgrößen)
  \end{itemize}
}
\card{51}{Was versteht man unter Freiraumausbreitung?}{
  Unter der \emph{Freiraumausbreitung} versteht man die Ausbreitung des elektromagnetischen Feldes im materiefreien Raum (Vakuum). Bei Freiraumausbreitung nimmt die Feldstärke mit wachsender Entfernung nur auf Grund der Entfernung ab (Entfernungsdämpfung). Freiraumbedingungen herrschen praktisch im Weltraum und noch mit sehr guter Näherung innerhalb des optischen Horizontes, wenn sonst keine störenden Effekte auftreten (Niederschlag, Reflexionen).
}
\card{52}{Welche Einflüsse haben Hindernisse auf die UKW-Ausbreitung?}{
  Ausbreitung über \SI{100}{\mega\Hz} erfolgt quasi optisch. Unter der Annahme einer Standardatmosphäre, die eine Ablenkung der Funkstrahlen zum Boden bewirkt, ergibt sich für einen Standort eine max. Reichweite, die man als Funkhorizont bezeichnet. Je höher der Standort, desto größer die Reichweite. Durch Reflektion kann es zu einem Funkschatten kommen, der eine Funkverbindung unmöglich macht. Neben der Lage spielt also auch die Hindernisfreiheit eine wichtige Rolle.
}
\card{53}{Definieren Sie den Begriff ,,Schädliche Störung``?}{
  Ist eine Störung, welche die Abwicklung des Funkverkehrs bei einem anderen Funkdienst, Navigationsfunkdienst, Sicherheitsfunkdienst gefährdet oder den Verkehr bei einem Funkdienst, der in Übereinstimmung mit den für den Funkverkehr geltenden Vorschriften wahrgenommen wird, beeinträchtigt, behindert oder wiederholt unterbricht. Auch Amateurfunk kann von schädlichen Störungen betroffen sein.
}
\card{54}{Definieren Sie den Begriff ,,Senderleistung``?}{
  Die Sendeleistung ist die der Antennenspeiseleitung zugeführte Leistung. Messgröße ist Watt. Gemäß Amateurfunkverordnung.
}
\card{55}{Definieren Sie den Begriff ,,Spitzenleistung``?}{
  Die Spitzenleistung ist eine Effektivleistung, die ein Sender während einer Periode der Hochfrequenzschwingung während der höchsten Spitze der Modulationshüllkurve unverzerrt der Antennenspeiseleitung zuführt. Ident mit dem Begriff \emph{PEP (peak envelope power)}
  \[ \text{PEP} = (0,707 \cdot \text{Uss}/2)^2 / R_0 \]
}
\card{56}{Definieren Sie den Begriff ,,unerwünschte Aussendung``?}{
  Die der Antennenspeiseleitung am Ausgang des Sende-Empfängers zugeführten Störsignale auf jeder anderen Frequenz als der Trägerfrequenz samt den Seitenbändern, die sich aus dem Modulationsprozess ergeben. Gemäß Amateurfunkverordnung.
}

\end{document}
