\documentclass[avery5371,grid,frame,a4paper]{flashcards}
%\usepackage[T1]{fontenc}
\usepackage[utf8]{inputenc}
\usepackage[german]{babel}
\usepackage{amsmath}
\usepackage[inline]{enumitem}
%\newcommand{\cardpaper}{a4paper}
%\newcommand{\cardpapermode}{portrait}
\renewcommand{\cardrows}{4}
\renewcommand{\cardcolumns}{2}
\cardfrontstyle[\large\slshape]{headings}
\cardbackstyle{empty}
%\cardbackstyle{plain}

\newcommand\question[2]{
  \begin{flashcard}[{\chap} -- #1]{#2}\end{flashcard}
}
\newcommand{\card}[3]{
  \begin{flashcard}[{\chap} -- #1]{#2}#3\end{flashcard}
}
\newcommand\class[1]{{\footnotesize [Klassen: #1]}}

\begin{document}
% Stand 15. November 2014
% http://www.bmvit.gv.at/bmvit/telekommunikation/funk/funkdienste/downloads/amateur_fragen.pdf

% Rechtliches

\def\chap{Rechtliches \class{1,3,4}}

\card{01}{Welche gesetzlichen Bestimmungen sind für den Amateurfunk maßgeblich?}{\begin{itemize}\itemsep1pt
\item Internationaler Fernmeldevertrag, \item Vollzugsordnung f. Funkdienst (VO-Funk), \item Telekommunikationsgesetz, \item Amateurfunk-Gesetz, \item Amateurfunk -Verordnung, \item Amateurfunkgebühren-Verordnung, \item Kundmachung d.Staaten, die Einwände gegen Amateurfunk erhoben haben. \end{itemize}}

\card{02}{Was ist die „ITU“?}{\begin{itemize}\itemsep1pt
\item Internationale Fernmeldeunion, \item völkerrechtlicher Verein, \item anerkennt Hoheitsrechte, \item fördert Beziehungen und Zusammenarbeit der Länder durch guten Fernmeldedienst \end{itemize}}

\card{03}{Welche Zwecke verfolgt der internationale Fernmeldevertrag?}{\begin{itemize}\itemsep1pt
\item Aufrechterhaltung, Ausbau der Zusammenarbeit zur Verbesserung, \item Verwendung der Fernmeldeeinrichtungen, \item technische Entwicklung, \item Leistungserhöhung der Dienste, \item Steigerung der Inanspruchnahme (öffentlich), \item Verbilligung \end{itemize}}

\card{04}{Welche Aufgaben hat das Radiocommunication Bureau?}{\begin{itemize}\itemsep1pt \item Registrierung der Frequenzen, \item Anerkennung der Frequenzen, \item Beratung, auch im Hinblick gestörter Frequenzen
 \end{itemize}}
\card{05}{Was ist die CEPT und welche Bedeutung hat sie?}{\begin{itemize}\itemsep1pt \item	Konferenz der europ. Post und Fernmeldeverwaltungen, \item 43 europäische Staaten, \item Australien, USA erkennt sie an, \item Zweck: \begin{itemize}\itemsep1pt \item Beziehungen vertiefen \item Zusammenarbeit fördern \item Markt für TK schaffen \end{itemize} \end{itemize}}

\card{06}{Was ist die VO Funk (Radio Regulations) und was regelt sie?}{\begin{itemize}\itemsep1pt \item	Vollzugsordnung f.d. Funkdienst \item	Bestandteil des Internationalen Fernmeldevertrags \item Bestimmungen über die Praxis \item für Amateurfunker wichtig, weil alle Bestimmungen auch für AF gelten \item	Frequenz muss stabil und frei von Nebenaussendungen sein (state-of-the-art) \end{itemize}}

\card{07}{Definieren Sie den Begriff ,,Funkanlage`` im Sinne des TKG.}{\begin{itemize}\itemsep1pt \item	Sende/Empfangseinrichtung \item beabsichtigte Informationsübertragung \item	ohne Verbindungsleitungen \item	mittels elektromagnetischer Wellen \end{itemize}}

\card{08}{Erläutern Sie den Unterschied zwischen einem Telekommunikationsdienst und dem Amateurfunkdienst?}{KD: gewerblich, Signalübertragung über Kommunikationsnetze einschl. Telekomm. (alles außer Rundfunk)- und	Übertragungsdienste in Rundfunknetze \\ AF: \begin{itemize}\itemsep1pt \item technisch/experimentell \item Erd/Weltraumfunkstellen \item	eigene Ausbildung, Verkehr mit anderen, Not/Katastrophendienst, technische Studien \end{itemize}}

\card{09}{Wann erlischt eine Bewilligung? Was kann passieren, wenn Sie ohne oder ohne entsprechende Amateurfunkbewilligung Amateurfunk betreiben?}{\begin{itemize}\itemsep1pt \item Tod \item Ablauf der Zeit \item Verzicht \item Widerruf (Verstoß gegen Bestimmungen) \end{itemize}	Urkunde ist innerhalb 2 Monaten ans Fernmeldebüro zurückzusenden}

\card{10}{Was passiert, wenn man ohne Bewilligung funkt?}{Verwaltungsübertretung / Verwaltungsstrafe 3.633 EUR}

\card{11}{Welche Funkanlagen sind bewilligungspflichtig, welche Art der Bewilligungen gibt es?}{Funkanlagen grundsätzlich bewilligungspflichtig\\ BMVIT kann für Gerätearten/type generell Errichtung und Betrieb bewilligen; BMVIT kann Einfuhr, Vertrieb und Besitz generell für bewilligungspflichtig erklären (öff. Sicherheit, Behörden). \\ AF-Bewilligung berechtigt zum Besitz von AF-Sendeanlagen, zu Änderung und Selbstbau, zur Einfuhr, zum vorübergehenden Besitz von Funkanlagen, die keine AF sind (3 Monate), zwecks Umbau zur AF für Eigenbedarf}

\card{12}{Sie ändern den Standort Ihrer Funkanlage – was haben Sie zu tun?}{Wenn Bestimmungen in der Bewilligung betroffen sind, bedarf einer Bewilligung:\\ Standortänderung,  Verwendung außerhalb des bewilligten Einsatzgebietes, technische Änderung \\
Behörde kann Bewilligungen ändern: \\ zur Sicherheit des TK-Verkehrs, aus technischen/betrieblichen Belangen, aus internationalen Gründen (Fernmeldevertragsrecht, geänderte Frequenznutzung). Schonung wirtschaftl./betrieblicher Interessen; man muss auf eigene Kosten nachkommen (ang. Frist)}

\card{13}{Was versteht man unter dem Aufsichtsrecht der Fernmeldebehörden über Telekommunikationsanlagen?}{\begin{itemize}\itemsep1pt \item TKG Kommunikationsdienste unterliegen d. Aufsicht	d. Regulierungsbehörde (Organe der Fernmeldebehörden, des Büros für Funkanlagen und TK-Endeinrichtungen) \item Die Organe haben der Reg.behörde Hilfe insb. bei 	fernmeldetechnischen Fragen zu leisten. \item	TK-Anlagen unterliegen d. Aufsicht d. 	Fernmeldebehörden. TK-Anlagen sind Anl./Geräte zur Abwicklung v. Kommunikation, 	Kabelrundfunknetze, Funkanlage, TK-Endeinrichtungen. \end{itemize}}

\card{14}{Ein Organ der Fernmeldebehörde will ihre Funkanlage überprüfen, was haben Sie zu tun?}{\begin{itemize}\itemsep1pt \item Organen (Ausweis!) derbFMB sind berechtigt, TK-Anlagen (Funkanlagen, Endgeräte) bzw. Teile auf Einhaltung der Gesetze u. Verordnungen zu prüfen \item Der Zugang ist ihnen zu gestatten. \item Auskünfte, Unterlagen. \item			„Vorführung“ der Anlagen, auf eigene Kosten. \end{itemize}}

\card{14}{ Welche Geheimhaltungspflichten treffen Sie als Funkamateur?}{Werden mittels Anlage Nachrichten empfangen, die nicht für die Anlage, das Endgerät, den Benutzer bestimmt sind:\begin{itemize}\itemsep1pt \item Inhalt der Nachricht / Tatsache des Empfangs dürfen nicht aufgezeichnet / anderen mitgeteilt / verwertet	werden. \item Aufgezeichnete Nachrichten sind zu löschen.\end{itemize}}

\card{16}{Was kann die Fernmeldebehörde machen, falls Sie einen anderen Funkdienst stören?}{Bei Störungen einer TK-Anlage durch eine andere können zweckmäßige Maßnahmen angeordnet und vollzogen werden, die zum Schutz der gestörten	Anlagen notwendig sind. Vermeidung überflüssiger Kosten. \\ Unbefugt errichtete / betriebene TK-Anlagen können ohne Androhung außer Betrieb gesetzt werden. \\ Für sonstige entgegen den Bestimmungen errichtete / betrieben TK-Anlagen gilt das nur zur Sicherung / Wiederherstellung ungestörter Kommunikation.}

\card{17}{Welche Gebühren müssen als Funkamateur entrichtet werden?}{
  \begin{tabular}{ccc}
    A & 100~W & 1,45~EUR \\
    B & 200~W & 2,91~EUR \\
    C & 400~W & 4,36~EUR \\
    D & 1000~W & 6,54~EUR
  \end{tabular}
  \begin{itemize}
    \item Klubfunkstelle: 6,54~EUR
    \item Klubfunkstelle (Vereinsräume, Räume Organisationen im öffentlichen Interesse) zu Unterrichtszwecken ohne strahlender Antenne / Fernwirkung: 1,45~EUR
  \end{itemize}
}

\card{18}{Definieren Sie den Begriff ,,Amateurfunkdienst``?}{\begin{itemize}\itemsep0pt \item technisch / experimentell \item	Erd / Weltraumfunkstellen \item	von Funkamateuren für: \begin{itemize}\itemsep0pt \item Ausbildung \item Verkehr untereinander \item Not / Katastrophenfunk \item	technische Studien\end{itemize}\end{itemize}}

\card{19}{Definieren Sie den Begriff ,,Funkamateure``?}{Das ist eine Person\begin{itemize}\itemsep0pt \item Amateurfunkbewilligung erteilt \item	beschäftigt mit Funktechnik/Betrieb \item	persönliche Neigung bzw. Organisation im öffentlichen Interesse \item jedoch nicht kommerziell / politisch\end{itemize}}

\card{20}{Definieren Sie den Begriff ,,Amateurfunkstelle``?}{\begin{itemize}\itemsep1pt \item Einer od. mehrere, od. Gruppe von Sendern und Empfängern (Zusatzeinrichtungen) \item	zum Betrieb des Amateurfunkdienstes an einem bestimmten Ort \item	erfassen von in Österreich dem Afu-Dienst zugewiesene Frequenzbereiche, auch wenn	der Sende/Empfangsbereich über diese	Frequenzbereiche hinausgeht\end{itemize}}

\card{21}{Definieren Sie den Begriff ,,Stationsverantwortlicher``?}{Natürliche Person, namhaft gemacht \begin{itemize}\itemsep1pt \item von Amateurfunkverein / von einer Organisation im öffentlichen Interesse \item verantwortlich für die Einhaltungen der Bestimmungen / Verordnungen des AFG\end{itemize}}

\card{22}{Definieren Sie den Begriff ,,Klubfunkstelle``?}{Amateurfunkstelle eines Amateurfunkvereins oder einer im öffentlichen Interessen tätigen Organisation}

\card{23}{Definieren Sie den Begriff ,,Bakensender``?}{automatische Amateurfunksendeanlage\begin{itemize}\itemsep1pt \item fester Standort \item sendet ständig technische und betriebliche Merkmale \item Zweck: Frequenzmessung / Erforschung der Funkausbreitungsbedingungen\end{itemize}}

\card{24}{Definieren Sie den Begriff ,,Relaisfunkstelle``?}{automatische Amateurfunksendeanlage\begin{itemize}\itemsep1pt \item Amateurfunkstelle, die der automatischen Informationsübertragung dient\end{itemize}}

\card{25}{ Darf Amateurfunk von Nichtamateuren abgehört werden?}{\begin{itemize}\itemsep1pt \item Ja, jeder darf abhören.\end{itemize}}

\card{26}{Voraussetzungen zur Erlangung einer Amateurfunkbewilligung?}{Errichtung/Betrieb AF-Stelle nur mit Bewilligung. Ausnahmen:  Mitbenutzung, Funkempfangsanlage, die nur AF-Frequenzbereiche abdeckt. Bewilligung ist Personen auf Antrag zu erteilen, wenn: 14. Lebensjahr vollendet, Amateurfunkprüfung abgelegt, befreit oder §25. Nichtvollhandlungsfähige: Haftung einer vollhandlungsfähigen Person bez. Gebührenforderung. Bewilligung für AF-Verein/Organisation: Stationsverantwortlicher mit Hauptwohnsitz im Inland (handlungsfähig, AF-Prüfung abgelegt, befreit oder §25)}

\card{27}{Wie und wo ist ein Antrag auf Erteilung einer Amateurfunkbewilligung zu stellen?}{Schriftlich, Daten des Antragstellers/des Stationsverantwortlichen:\\ Vor- / Zuname, Geburtsdatum, Hauptwohnsitz, Standort und Gebiet der AF-Stelle , Leistungsstufe, Bewilligungsklasse, technisch Merkmale \\ Beizulegen: Amateurfunkprüfungszeugnis, Bescheid ü. Befreiung, §25-Zeugnis, Vorschlag Rufzeichen, kein Anspruch. \\ Entscheidung über Antrag: zuständig. Fernmeldebüro (für Ausländer: FMB f. W/Nö/B)}

%\card{28}{Welche Angaben stehen in einer Amateurfunkbewilligung?}{\begin{itemize}\itemsep0pt \item Vor- / Zuname \item Geburtsdatum \item Hauptwohnsitz \item Standort und Gebiet der AF-Stelle \item Leistungsstufe \item Bewilligungsklasse \item Rufzeichen \item technisch Merkmale\end{itemize}}

\card{28}{Rufzeichen und Sonderrufzeichen?}{In der Amateurfunkbewilligung ist ein Rufzeichen zuzuweisen. Auf Antrag kann BMVIT zu besonderen Anlässen Sonderrufzeichen befristet zuweisen. BMVIT kann FMB ermächtigen Sonderrufzeichen zuzuweisen. Rufzeichen aussenden: zu Beginn, während Übertragung wiederholt, am Ende. Bei Klubfunkstelle: Klubfunkstellenrufzeichen mit Zustimmung d. Stationsverantwortlichen auch eigenes Rufzeichen (nur Berechtigungsumfang!)}

\card{29}{Wozu berechtigt eine Amateurfunkbewilligung?}{Berechtigt zur Errichtung, zum Betrieb \\
• einer/mehrerer fester AF-Stellen (angegebene Standorte)
• einer/mehrerer beweglicher AF-Stellen (gesamtes Bundesgebiet)
• vorübergehend (3 Monate) feste AF-Stelle an einem anderen Ort im Bundesgebiet als angegeben. \\
Berechtigt zum Besitz von AF-Sendeanlagen und: \\
• Änderung / Selbstbau
• Einfuhr für den Eigenbedarf
• Besitz von Nicht-AF-Anlagen zum Zweck des Umbaus (vorübergehend, 3 Monate)}

\card{30}{Unter welchen Voraussetzungen dürfen Aussendungen durchgeführt werden?}{Aussendungen mit einer AF-Stelle nur: \\
• in den zugewiesenen Frequenzen (AF-Dienst/Bewilligungsklasse)
• in der festgesetzten Sendeart (BWK)
• mit der erlaubten Sendeleistung (abh. von Leistungsstufe des Frequenzbereichs und AF-Bewilligung)
• mit der erlaubten Bandbreite
• bei persönlicher Anwesenheit (ausser Relais/Baken)
• AF-Stellen nicht mit TK-Netzen verbinden!
• BMVIT kann Ausnahmen vorsehen (Technikerprobung: Bandbreite, Ausbildung: Sendeleistung)}

\card{31}{Wie ist der Amateurfunkverkehr abzuwickeln?}{Offene Sprache, nicht verschlüsselt.
Inhalt:\\ • Übertragungsversuche \\ • technische/betriebliche Mitteilungen \\ • Bemerkung persönlicher Natur, bildliche Darstellungen, bei denen wegen Belanglosigkeit eine Inanspruchnahme von TK-Diensten nicht verlangt werden kann \\ • Verkehr nur unmittelbar zwischen bewilligten AF-Stellen ohne Benutzung anderer TK-Anlagen.}

\card{32}{Definieren Sie den Begriff Not- und Katastrophenfunkverkehr?}{\begin{itemize}\itemsep0pt \item Notfunkverkehr: Nachrichtenübermittlung zwischen Funkstelle in Not/beteiligt/Zeuge und einer/mehreren hilfeleistenden Funkstellen. \item Notfall: menschliches Leben in Gefahr \item Katastrophenfunkverkehr: Nachrichtenübermittlung (nat./int. Hilfeleistung betreffend) zwischen Funkstelle im Katastrophengebiet (geogr. Gebiet, für die Dauer) und Hilfe leistenden Organisationen.\end{itemize}}

\card{33}{Wo können Sie erfahren, unter welchen technischen Parametern (Sendeart, Leistungsstufe, Einschränkungen, etc.) Sie mit Ihrer Lizenzklasse in welchem Frequenzband Amateurfunk betreiben dürfen?}{In der \emph{Anlage 2} der \emph{Amateurfunkverordnung} werden die dem Amateurfunk zugewiesenen Frequenzbereiche, der Status, die zulässige Bewilligungsklasse und Leistungsstufe sowie eventuelle Bemerkungen bzw. Einschränkungen definiert.}

\card{34}{Was ist ein und wozu gibt es ein Funktagebuch?}{\begin{itemize}\itemsep0pt \item Zur Klärung frequenztechnischer Fragen wenn von der FMB verlangt. \item Auch mit Hilfe von EDV. \item Bei Notfunkverkehr komplette Nachricht aufzeichnen. \item 1 Jahr aufbewahren, den Organen des FMB unmittelbar lesbar vorweisen. \end{itemize}}

\card{35}{In welchem Umfang ist Mitbenutzung einer Amateurfunkstelle möglich?}{Inhaber der AF-Bewilligung/Stationsverantwortliche (bleibt für Einhaltung der Bestimmungen verantwortlich, muss überwachen) können Personen, die die AF-Prüfung bestanden haben, die Mitbenutzung gestatten. Mitbenützer darf das nur im Umfang: \\ • der Prüfungskategorie des AF-Prüfungszeugnisses \\ • der Bewilligungsklasse / Leistungsstufe der AF-Bewilligung des AF-Stellen Inhabers \\ • Der BMVIT kann zum Zweck der Ausbildung Ausnahmen vorsehen.}

\card{36}{Wer ist für Amtshandlungen nach dem Amateurfunkgesetz zuständig?}{\begin{itemize}\itemsep1pt \item Für die Amtshandlungen zuständig ist das örtliche FMB (entspr. Hauptwohnsitz). \item Bei mehreren FMBs ist einvernehmlich vorgehen. \item Der BMVIT ist zuständig für die Entscheidung über Rechtsmittel gegen Bescheide des FMB, soweit nicht der UVS zuständig ist.\end{itemize}}

\card{37}{Nennen Sie einige Verwaltungsstrafbestimmungen in Bezug auf den Amateurfunk?}{
\scriptsize
  \begin{minipage}{0.48\textwidth}
    \begin{itemize}[leftmargin=10pt,itemsep=0pt]
      \item Senden in AF-Frequenz, aber nicht Bewilligungsklasse
      \item Sendearten nicht in der Bewilligungsklasse
      \item höhere Sendeleistung / Bandbreite*
      \item nicht persönlich anwesend
      \item Verbindung AF-Stellen / TK-Anlagen*
    \end{itemize}
    * \quad Ausnahme nicht vorliegend
  \end{minipage}
  \begin{minipage}{0.5\textwidth}
    \begin{itemize}[leftmargin=10pt,itemsep=0pt]
      \item vorsätzlich Verkehr mit nicht bewilligter Funkstelle
      \item nicht unmittelbarer Verkehr mit bewilligter Funkstelle
      \item Verkehr mit Funkstellen in Ländern, die Einwand erhoben haben
      \item Gestattung von Mitbenutzung durch Personen ohne Prüfung
      \item Mitbenutzung ohne Prüfung
      \item mangelhafte Überwachung der Mitbenutzung (Einhalten der Bestimmungen)
    \end{itemize}
  \end{minipage}
%  \begin{itemize}\itemsep0pt
%    \item senden in Frequenzbereichen, die nicht dem AF-Dienst zugewiesen sind
%    \item wenn im Verkehr mit anderen Funkstellen Ansehen/Sicherheit/Wirtschaftsinteressen gefährdet werden, gegen öffentliche Ordnung/Sittlichkeit verstoßen wird
%    \item wenn Notrufe gestört/nicht beantwortet werden
%    \item wenn ein anderes oder kein Rufzeichen gesendet wird
%  \end{itemize}
%  \begin{itemize}\itemsep1pt
%    \item Errichten oder Betreiben einer AF-Stelle ohne AF-Bewilligung
%    \item Verwendung von Daten der Rufzeichenliste für andere Zwecke als AF
%  \end{itemize}
%  Wenn Tatbestand strenger bestrafbar (Gerichte zuständig), keine Verwaltungsübertretung.
}

\card{38}{Was ist eine CEPT-Lizenz? \\ (oder CEPT-Novizen-Lizenz)}{
  \begin{itemize}\itemsep1pt
    \item Eine AF-Bewilligung oder eine Urkunde, die einen Hinweis darauf enthält, dass sie eine CEPT-Lizenz ist.
    \item Erteilung/Ausstellung: Von der Behörde eines Staates, der die CEPT-Empfehlung T/R61-01 anwendet.
    \item CEPT-Novice-Lizenz: entsprechend ERC/REC~05(06)
  \end{itemize}
}

\card{39}{Was darf ein ausländischer CEPT-Lizenz Inhaber oder CEPT-Novizen-Lizenz in Österreich ohne eigene österreichische Bewilligung?}{Inhaber einer ausländischen CEPT-Lizenz, älter als 14 Jahre, dürfen 3 Monate ab Einreisetag eine AFU-Stelle errichten und betreiben.}

\card{40}{Was bedeutet der Begriff Reziprozität und nennen Sie ein Beispiel?}{
  \begin{itemize}\itemsep1pt
    \item Begriff aus dem Völkerrecht
    \item Angehörige anderer Staaten werden in Österreich so behandelt, wie Österreicher im anderen Staat.
  \end{itemize}
  Beispiel:
  \begin{itemize}\itemsep1pt
    \item Ausländern wird Bewilligung nur erteilt, wenn Österreichern in diesem Staat auch das Errichten und Betreiben einer AFU-Stelle gestattet ist
  \end{itemize}
}

\card{41}{Nennen Sie die Bewilligungsklassen und wozu berechtigen diese?}{•  3 Klassen (1, 3 und 4) •  international Klasse 1 (CEPT AFU-Bewilligung), Klasse 4 (CEPT NOVICE-Lizenz), Klasse 3 national •  Klasse 1 darf alle Frequenzbereiche und Sendearten (Einschränkungen beachten) nutzen. •  Klasse 3 darf nur 2m und 70cm und bestimmte Sendearten (Einschränkungen beachten) nutzen. Keine Selbstbauanlagen, nur kommerziell gefertigte, nicht veränderte, Leistungsstufe A •  Klasse 4: 2m und 70cm, 4 KW-Bereiche, sonst wie Klasse 3 •  Mitbenutzung von Klubfunkstellen ist gestattet.}

\card{42}{Welche Leistungsstufen kennen Sie und nennen Sie deren Merkmale?}{
  \begin{center}
    \vspace{5pt}
    \begin{tabular}{cl}
      A & 100~Watt max \\
      B & 200~Watt max \\
      C & 400~Watt max \\
      D & 1000~Watt max
    \end{tabular}
  \end{center}
  Überschreitung der Grenzwerte um 20\% tolerabel.
}

\card{43}{Unter welchen Voraussetzungen kann eine Amateurfunkbewilligung für die Leistungsstufe C erteilt werden?}{wenn am genannten Standort seit mind. 1 Jahr eine AF-Stelle mit ,,Leistungsstufe B`` störungsfrei betrieben wurde.}

\card{44}{Unter welchen Voraussetzungen kann eine Amateurfunkbewilligung für die Leistungsstufe D erteilt werden?}{Bewilligung für ,,Leistungsstufe D``: \begin{itemize}\itemsep1pt \item nur AFU-Vereinen und im öffentlichen Interesse tätigen Organisationen \item kann von Ergebnissen eines Probebetriebs (6 Monate) abhängig gemacht werden\end{itemize}}

\card{45}{Was bedeutet der Status eines Funkdienstes (Primär, Primär/Exklusiv(Pex), Sekundär, ISM)?}{
 \footnotesize
  \begin{description}\itemsep0pt
    \item[Pex] primärer Funkdienst (exklusiv für Amateurfunk)
    \item[P] primärer Funkdienst (Mitbenutzung durch andere FD)
    \item[S] sekundärer Funkdienst (primärer Funkdienst hat Vorrang),
      \begin{itemize}[leftmargin=0pt,itemsep=0pt]
        \item dürfen keine Störungen bei primären verursachen
        \item können keinen Schutz gegen Störungen von primären verlangen
        \item können Schutz gegen Störungen von sekundären verlangen
      \end{itemize}
    \item[ISM] Hochfrequenzbereich für industrielle, wissenschaftliche, medizinische Anwendung
  \end{description}
}

\card{46}{Ist die Verwendung der Betriebsart Telegraphie an eine bestimmte Voraussetzungen gebunden?}{\begin{itemize}\itemsep1pt \item Nein, Verwendung aller Betriebsarten bei Klasse 1, 4 und Klasse 3 zulässig. \item Einige Länder außerhalb der CEPT verlangen für die Erteilung einer Gastlizenz
unter 30 MHz eine Telegrafieprüfung.\end{itemize}}

\card{47}{Wann wird eine schädliche Störung als solche behandelt?}{\small{\begin{itemize}\itemsep0pt \item Wenn die Funkanlagen entsprechend Bewilligungen errichtet sind und die gestörte Empfangsanlage vorschriftsmäßig betrieben wird. \item Nicht, wenn Störung durch andere, ordnungsgemäß errichtete/betriebene AF-Stellen verursacht wird. \item Nicht in ISM Bändern. \item Bei Störung durch TK-Einrichtungen kann die FMB (wenn alle beteiligten Anlagen den Vorschriften entsprechen) unter Abwägung des wirtschaftlichen Aufwands techn./betriebl. Maßnahmen zur Behebung anordnen.\end{itemize}}}

\card{48}{Was gilt für einen Amateurfunkbetrieb auf Schiffen und in Flugzeugen?}{Es entscheidet der Pilot / der Kapitän, ob AFU durchgeführt werden darf.}

\card{49}{Welche Aussendungen dürfen von einer Amateurfunkstelle empfangen werden?}{Mit einer Empfangsanlage dürfen empfangen werden: \begin{itemize}\itemsep1pt \item Aussendungen anderer AF-Stellen \item Rundfunk \item Nachrichten an alle, sofern diese für den
Gebrauch durch die Öffentlichkeit bestimmt \item Not/Katastrophenverkehr\end{itemize}}

\card{50}{Was darf der Nachrichteninhalt einer Amateurfunkaussendung sein?}{\small{Offene Sprache (Abkürzungen, Zeichen, Esperanto, Latein), Nachricht muss verständlich bleiben, nur normierte Übertragungsverfahren:
•  Morsealphabet, Telegraphiealphabet Nr. 2, AMTOR/PACTOR, ITU-R-Empf. M476/M625, HELL-System, (Fernsehen AM), im ITU-R-Report 624 beschriebene, (Packet Radio) AX-25
Protokoll (alle Übertragungsgeschwindigkeiten), DVBT (EN300744), DVBS (EN300421)
•  Verwendung anderer Verfahren: Rufzeichen in offener Sprache/normiert, Inhalt 3 Wochen reproduzierbar dokumentiert
•  Aussendung von reinem Träger nur zu Mess/Testzwecken}}

\card{51}{Gibt es eine Möglichkeit, dass ein Funkamateur, der die Prüfungskategorie 3 erfolgreich abgelegt hat, auf anderen Frequenzen als dem 2m / 70cm-Band Funkverkehr haben darf?}{\begin{itemize}\itemsep1pt \item Klubfunkstelle mit Bewilligungsklasse 1 \item darf auf allen, dem AF zugewiesenen Frequenzen \item von Personen mit Klasse 3 und 4 \item zum Zweck der Ausbildung \item unter Überwachung eines Inhabers (Klasse 1) \item mitbenutzt werden\end{itemize}}

\card{52}{Wer darf eine Relaisfunkstelle errichten / betreiben / benutzen und wie ist deren Rufzeichen auszusenden?}{• Bewilligung für eine Relaisfunkstelle wird nur einem
Amateurfunkverein/einer im öffentlichen Interesse tätigen Organisation erteilt,
• wenn der Einsatz der Betriebsfrequenzen (hinsichtl. zugeteilter Frequ.) störungsfrei erfolgen kann.
• eigenes Bewilligungsverfahren
• Benutzung ist allen AF-Stellen zu gestatten
• Bei Sprachübertragungsrelais: Aussendung des Rufzeichens in Sprache oder mit 60-100 Zeichen pro Minute in Telegraphie.
• Bei anderen: Aussendung des Rufzeichens in der jeweiligen Sendeart.}

\card{53}{Was haben Sie zu tun, wenn Sie Funkverkehr mit einer nicht bewilligten Amateurfunkstelle haben und mit wem dürfen Sie keinen Amateurfunkverkehr haben?}{\begin{itemize}\itemsep1pt \item Nicht bewilligte AF-Stelle: Verkehr abbrechen. \item Alles unterlassen, was das Ansehen, die Sicherheit, die Wirtschaftsinteressen gefährdet, was gegen die öffentliche Ordnung oder Sittlichkeit verstößt. \item Unzulässiger Verkehr: Mit AFU-Stellen in Ländern, die Einwand erhoben haben \item Kundmachung durch BMVIT im Bgbl.\end{itemize}}

\card{54}{Welche besonderen Aufgaben hat die ITU in Bezug auf Funkdienste und welche Ausschüsse sind dafür zuständig?}{\small{Aufgaben:
•  Zuweisung der Frequenzen
•  Verhinderung gegenseitiger Störungen
•  Verbesserung der Ausnutzung der Bänder
•  Förderung der Zusammenarbeit der Hilfsdienste zur Erhaltung menschlichen Lebens \\
Ausschüsse:
•  Radiocommunication Bureau: zugeteilte Frequenzen (Länder) registrieren, Anerkennung sichern, Beratung bei Störungen
•  Radiocommunication Sector: Studien über technische und betriebliche Fragen, Mitglieder beraten
•  Telecommunication Sector: Beratung, Studien: Technisches, Betriebs/Gebührenfragen (so billig wie möglich, trotzdem dotiert)}}

\card{55}{Was bedeutet missbräuchliche Verwendung von Funkanlagen?}{\small{• Nachrichtenübermittlung, die öffentliche Ordnung und Sicherheit gefährdet, gegen Gesetze verstößt
• Belästigung oder Verängstigung anderer
• Verletzung der geltenden Geheimhaltungspflicht
• Nachrichtenübermittlung, die nicht dem bewilligten Zweck der FA entspricht
• Inhaber (nicht Zugangsanbieter) müssen zumutbare Maßnahmen zur Vermeidung von Missbrauch treffen
• bewilligter Zweck, Standort / im Einsatzgebiet
• bewilligte Frequenzen, Rufzeichen
• nicht zugelassene FA / TK-Einrichtungen dürfen nicht mit einem öffentl. Komm.netz verbunden/betrieben werden}}

\card{56}{Was hat der Inhaber einer Amateurfunkstelle zu tun, wenn er nicht bei dieser Stelle anwesend ist?}{Der Inhaber einer Amateurfunkstelle hat \emph{geeignete Maßnahmen} zu treffen, die Inbetriebsetzung seiner Funkstelle durch \emph{unbefugte Personen} auszuschließen. Aussendungen dürfen nur durchgeführt werden, wenn der Inhaber einer Amateurfunkbewilligung oder der Mitbenützer der Amateurfunkstelle während der gesamten Dauer der Aussendung \emph{persönlich} an der Amateurfunkstelle \emph{anwesend} ist, \emph{außer} es handelt sich um eine Relaisfunkstelle oder einen Bakensender.}

\card{57}{Welche Bestimmungen sind beim Betrieb einer Amateurfunkstelle im Ausland zu beachten?}{Die Bestimmungen des Gastlandes.}

\card{58}{Unter welchen Voraussetzungen darf der Inhaber einer Amateurfunkbewilligung der Bewilligungsklasse 3 im Ausland Amateurfunkbetrieb durchführen?}{Er muss eine Gastlizenz beantragen.}

\card{59}{Wozu berechtigt eine Amateurfunkbewilligung der Klasse 4?}{\begin{itemize}
  \item Sendebetrieb im 160, 80, 15, 10, 2m und 70cm Band
  \item Leistungsstufe A (max. 100~W)
  \item nur kommerzielle, unmodifizierte Geräte verwenden
\end{itemize}}

\card{60}{Aufgrund welcher internationalen Regelung dürfen Funkamateure aus bestimmten Ländern auch ohne individuelle Gastzulassung vorübergehend in Österreich Amateurfunk ausüben?}{\small Die Empfehlung T/R~61-01 regelt die Gültigkeit von Amateurfunkbewilligungen für die CEPT-Mitgliedsländer. Mit der Bewilligungsklasse 1 (= CEPT-Zertifikat für Funkamateure) darf in den CEPT-Mitgliedsländern auf die Dauer von 3~Monaten ohne Gastlizenz Amateurfunkbetrieb unter Beachtung nationaler Bestimmungen durchgeführt werden.

\begin{description}
  \item[T/R~61-02] Umfang und Inhalt der Amateurfunkprüfung zur Erlangung eines CEPT-Zertifikats
  \item[ERC/REC~05/06] Umfang und Inhalt der Amateurfunkprüfung zur Erlangung eines CEPT-Novice-Zertifikates
\end{description}
}

\card{61}{Unter welchen Voraussetzungen ist die Verbindung von Amateurfunkstellen mittels Internettechnologie zulässig?}{Folgende Voraussetzungen müssen erfüllt sein:
\begin{itemize}
  \item zwei oder mehrere Amateurfunkstellen werden verbunden
  \item Erprobung neuer Übertragungstechnologien
  \item kein gewerblich-wirtschaftliche Zwecke
  \item kein reiner Internetzugang
\end{itemize}}


\def\chap{Betrieb und Fertigkeiten \class{1,4}}

\card{01}{Wie eröffnen Sie einen Funkverkehr in Phonie, wie in Telegraphie?}{
  \small
  \begin{enumerate}
    \item Reinhören, ob Frequenz frei ist
    \item
      Phonie: ,,is this frequency in use?``,
      CW: ,,QRL?``
    \item
      Phonie: ,,this frequency in use!`` $\rightarrow$ ,,sorry!``, \\
      CW: ,,QRL`` $\rightarrow$ ,,SRI``
    \item Wenn frei, 3~mal \\
      Phonie: ,,CQ, CQ, CQ - this is call, call`` \\
      CW: ,,CQ CQ CQ DE call``
  \end{enumerate}
  Beachte die \emph{tote Zone}.
  Contest: ,,CQ Contest, this is …`` (3~mal) ,,CQ Test de …`` (1-3~mal)
}

\card{02}{Was ist das gebräuchliche Minimum einer Amateurfunkverbindung?}{
  \begin{itemize}
    \item Rufzeichen
    \item Rapport (RS bzw. RST)
    \item Vorname
    \item Standort (QTH)
    \item (optional) Stationsbeschreibung
  \end{itemize}
}

\card{03a}{Welche Bedeutung haben die Q-Gruppen im allgemeinen?
  \begin{center}
    QRM \quad QSO \quad QSY \quad QSL \quad QRP \quad QTR
  \end{center}
}{\begin{description}
  \item[QRM] ich werde gestört (Fremdstörungen),
  \item[QSO] ich habe Verbindung mit \dots
  \item[QSY] wechseln Sie auf die Frequenz \dots kHz
  \item[QSL] ich werde eine Empfangsbestätigung (QSL-Karte) geben
  \item[QRP] vermindern Sie die Sendeleistung
  \item[QTR] es ist \dots Uhr GMT (UTC)
 \end{description}
}

\card{03b}{Welche Bedeutung haben die Q-Gruppen im allgemeinen?
  \begin{center}
    QRS \quad QRX \quad QRO \quad QRV \quad QSP \quad QRG
  \end{center}
}{\begin{description}
  \item[QRS] geben Sie langsamer
  \item[QRX] ich werde Sie um \dots~Uhr auf \dots~kHz wieder rufen
  \item[QRO] erhöhen Sie Ihre Sendeleistung
  \item[QRV] ich bin betriebsbereit
  \item[QSP] ich werde an \dots\ weiterübermitteln,
  \item[QRG] ihre genaue Frequenz ist \dots~kHz
\end{description}}

\card{03c}{Welche Bedeutung haben die Q-Gruppen im allgemeinen?
  \begin{center}
    QRT \quad QRU \quad QRN \quad QRB \quad QTH \quad QSB
  \end{center}
}{\begin{description}\itemsep0pt
  \item[QRT] stellen Sie die Aussendung(en) ein
  \item[QRU] ich habe nichts für Sie vorliegen
  \item[QRN] ich habe atmosphärische Störungen (1 = keine, 5 = sehr stark),
  \item[QRB] die Entfernung zwischen unseren beiden Stationen ist \dots\ km
  \item[QTH] mein Standort ist \dots
  \item[QSB] Ihre Zeichen weisen Fading auf (= die Empfangsfeldstärke schwankt).
 \end{description}
}

\question{04}{Sie wollen, dass Ihre Gegenstation die Sendeleistung vermindert. Welche Q-Gruppe verwenden Sie?}{}
\question{05}{Was bedeuten die Hinweise \\ ,,5 UP`` bzw. ,,10 DOWN``?}{}
\question{06}{Sie wollen in einen bestehenden Funkverkehr einsteigen. Wie führen Sie das durch?}{}
\question{07}{Welche betrieblichen Auswirkungen haben die besonderen Ausbreitungsbedingungen auf Kurzwelle?}{}
\question{08}{Welche betriebliche Auswirkung hat die Bodenwellen-Ausbreitung?}{}
\question{09}{Welche betriebliche Auswirkung hat die Raumwellen-Ausbreitung, in welchem Frequenzbereich ist sie von Bedeutung?}{}
\question{10}{Welche betriebliche Bedeutung hat die kritische Frequenz?}{}
\question{11}{Welche betriebliche Bedeutung haben die Begriffe ,,MUF`` und ,,LUF``?}{}
\question{12}{Was versteht man unter Fading auf Kurzwelle, wodurch entsteht Fading und wie reagieren Sie, um den Funkverkehr aufrecht zu erhalten?}{}
\question{13}{Ausbreitung von Funkwellen -- Ausbreitungsmerkmale in den verschiedenen Amateurfunk Frequenzbereichen?}{}
\question{14}{Welchen Einfluß hat die Ionosphäre auf die Ausbreitung von Funkwellen über 30 MHz?}{}
\question{15}{Erklären Sie die Begriffe Fresnelzone, Geländeschnitt}{}
\question{16}{Was ist die tote Zone? Was ist ein Skip?}{}
\question{17}{Wovon hängt die maximal erzielbare Reichweite auf Kurzwelle ab?}
\question{18}{Was verstehen Sie unter kurzem Weg? Was unter langem Weg?}
\question{19}{Was verstehen Sie unter dem Dämmerungseffekt?}
\question{20}{Was verstehen Sie unter der ,,Grey-Line``, welche Besonderheiten in der Funkausbreitung können auftreten?}
\question{21}{Beschreiben Sie den Aufbau der lonosphäre und welche betriebliche Konsequenzen ergeben sich daraus?}
\question{22}{Wie verhalten sich die Ionosphärenschichten im Tagesverlauf bzw. im Jahresverlauf?}
\question{23}{Welchen Einfluss hat die geographische Breite auf die Kurzwellenausbreitung?}
\question{24}{Was versteht man unter Sonnenaktivität, unter der Sonnenfleckenrelativzahl, unter dem ,,Solar-Flux``? Welchen Einfluss hat sie auf die Kurzwellenausbreitung?}
\question{25}{Welchen Zyklen unterliegen die Ausbreitungsbedingungen auf Kurzwelle?}
\question{26}{Beschreiben Sie das charakteristische Ausbreitungsverhalten in den dem Amateurfunkdienst zugewiesenen Frequenzbändern unter 30 MHz?}
\question{27}{Was versteht man unter einem Mögel-Dellinger-Effekt und welche betriebliche Auswirkungen hat er?}
\question{28}{Welche Auswirkungen haben Polarlicht-Erscheinungen auf die Kurzwellenausbreitung?}
\question{29}{Welche Faktoren können den Funkbetrieb auf Kurzwelle beeinflussen?}
\question{30}{Wie wirkt sich die Tageszeit auf die Ausbreitung in den Kurzwellenbändern bis 40m aus? (160m/80m-/40m-Band)}
\question{31}{Was verstehen Sie unter ,,Sporadic E-Verbindungen``?}
\question{32}{Was verstehen Sie unter ,,Short-Skips``?}
\question{33}{Was verstehen Sie unter einem Notverkehr, wie wird er angekündigt?}
\question{34}{Sie empfangen einen Notruf – woran erkennen Sie diesen und wie haben Sie sich zu verhalten?}
\question{35}{Auf welchen Bändern könnten Sie einen Notruf empfangen?}
\question{36}{Welche Sendearten sind im Kurzwellenbereich zulässig?}
\question{37}{Müssen Sie ein Funktagebuch führen und welche Angaben muss es enthalten?}
\question{38}{Was verstehen Sie im Telegraphiebetrieb unter ,,BK-Verkehr``?}
\question{39}{Was verstehen Sie unter UTC (GMT) -- Zusammenhang zu Lokalzeit, Sommerzeit}
\question{40}{Nennen Sie die konkreten Frequenzbereiche, die dem Amateurfunkdienst in den jeweiligen Frequenzbändern zugewiesen sind (5 Beispiele)}
\question{41}{Wie arbeiten Sie mit ausländischen Amateurfunkstationen zusammen, die einen anderen/erweiterten Bandbereich benutzen? (Beispiele: 40m, 80m)?}
\question{42}{Was bedeuten die folgenden Abkürzungen: BK, CQ, CW, DE, K?}
\question{42}{Was bedeuten die folgenden Abkürzungen: PSE, RST, R, N, UR?}
\question{42}{Was bedeuten die folgenden Abkürzungen: FB, DX, RPT, HW, CL?}
% BK engl. break (Aufforderung zur Unterbrechung)
% CQ an alle (Funkstellen)
% CW engl. continuous wave / Telegraphie
% DE von
% K kommen
% PSE engl. please / bitte
% RST Rapport (R = engl. Readability / Lesbarkeit; S = engl. Signalstrengh / Lautstärke; T = engl. Tonequality / Signalqualität, nur für CW)
% R engl. roger / verstanden
% N engl. no / nein
% UR engl your / dein, deine
% FB engl. faible / gut
% DX Weitverbindung
% RPT engl. Repeat / wiederholen
% HW engl. how? / wie?
% CL engl. close / für ,,ich schließe die Funkstelle``
\question{43}{Wie wirkt sich Polarisationsfading auf den Kurzwellenbetrieb aus?}
\question{44}{Was versteht man unter Schwund im Kurzwellenbereich und wie reagieren Sie, um den Funkverkehr aufrecht zu erhalten?}
\question{45}{Welche Maßnahmen ergreifen Sie, wenn Sie darauf aufmerksam gemacht werden, dass Ihre Aussendung ,,splattert``?}
\question{46}{Was ist ein ,,Pile-Up`` -- wie verhalten Sie sich richtig?}
\question{47}{Was verstehen Sie unter den Begriffen {\footnotesize\texttt MAYDAY - SECURITEE - SILENCE MAYDAY - MAYDAY RELAY?}}
\question{48}{Welche Mess- und Kontrollgeräte sind bei einer Amateurfunkstelle vorgeschrieben?}
\question{49}{Was ist bei der Abstimmung des Leistungsverstärkers einer Amateurfunkstelle zu beachten?}
\question{50}{Wie wird ein Funkrufzeichen allgemein bzw. ein Amateurfunkrufzeichen aufgebaut – nach welcher Vorschrift?}
\question{51}{Buchstabieren Sie folgende Worte bzw. den folgenden Text nach dem internationalen Buchstabieralphabet: \dots}
\question{52}{Was ist beim Betrieb an den Bandgrenzen zu beachten?}
\question{53}{Nennen Sie Beispiele österreichischer Amateurfunkrufzeichen mit Zusätzen (zB: am, mm, /1).}
\question{54}{Nennen Sie die Landeskenner von fünf Nachbarländern und von fünf weiteren Ländern.}
\question{55}{Was bedeuten die Ziffern im österreichischen Amateurfunkrufzeichen, welche Rufzeichenzusätze sind zulässig?}
\question{56}{Welche Bestimmungen sind beim Betrieb im 160m-Band zu beachten?}
\question{57}{Welche Betriebsverfahren werden bei Scatter-Verbindungen verwendet?}
\question{58}{Welche Betriebsverfahren werden bei Meteorscatter-Verbindungen angewendet?}
\question{59}{Erklären Sie die Betriebsabwicklung bei Relaisbetrieb.}
\question{60}{Was versteht man unter ,,EME - Verbindungen``? Welches Betriebsverfahren wird angewendet?}
\question{61}{Was verstehen Sie unter Packet Radio? Welches Betriebsverfahren wird angewendet?}
\question{62}{Was verstehen Sie unter den Begriffen Mailbox, Digipeater, Netzknoten und welche betriebliche Besonderheiten sind zu beachten?}
\question{63}{Erklären Sie die Begriffe Relaisfunkstelle, Transponder, Bakensender und welche betrieblichen Besonderheiten sind zu beachten?}
\question{64}{Erklären Sie die Betriebsabwicklung bei ATV-Betrieb.}
\question{65}{Was ist bei Überreichweitenbedingungen zu beachten?}
\question{66}{Welchen Einfluss hat die Wahl des Standortes für UKW-Ausbreitung?}
\question{67}{Erklären Sie das Betriebsverfahren SSTV.}
\question{68}{Nennen Sie Einflüsse, die die Lesbarkeit einer Funkverbindung verschlechtern.}
\question{69}{Wie beurteilen Sie die Aussendung Ihrer Gegenstelle und wie wird diese Beurteilung der Gegenstelle mitgeteilt?}
\question{70}{Wie teilen Sie der Gegenstation Ihren Standort mit?}
\question{71}{Was ist ein ,,Contest``? Wie verhalten Sie sich richtig?}
\question{72}{Wie gehen Sie bei der Planung einer Amateurfunkverbindung zu einem bestimmten Ort vor?}
\question{73}{Was ist hinsichtlich der Herstellung oder Veränderung von Amateurfunkgeräten zu beachten?}
\question{74}{Beschreiben Sie das typische Ausbreitungsverhalten in den Frequenzbändern 6m--2m und 70cm.}

\def\chap{Betrieb und Fertigkeiten \class{3}}

\question{01}{Frequenzbereich des 70cm-Amateurfunkbandes / 2m Bandes?}
\question{02}{Wie eröffnen Sie einen Sprechfunkverkehr?}
\question{03}{Wie sind Amateurfunkrufzeichen aufgebaut?}
\question{04}{Welche Zusätze zu einem Amateurfunkrufzeichen sind zulässig?}
\question{05}{Nennen Sie mindestens 5 Landeskenner der umliegenden Länder.}
\question{06}{Wie beurteilen Sie das Signal Ihrer Gegenstation?}
\question{07}{Was versteht man unter ,,S-Stufe(n)``?}
\question{08}{Was versteht man unter Not- und Katastrophenfunkverkehr, wie wird er gekennzeichnet?}
\question{09}{Wie nahe dürfen Sie beim Sendebetrieb an die Bandgrenze herangehen?}
\question{10}{Welche Sendearten sind mit der Bewilligungsklasse~3 zulässig und mit welcher maximalen Sendeleistung?}
\question{11}{Was versteht man unter einem Amateurfunkrelais, wozu dient es?}
\question{12}{Wie wickeln Sie einen Betrieb über ein Amateurfunkrelais ab?}
\question{13}{Buchstabieren Sie Ihren Vor- und Zunamen nach dem internationalen Buchstabieralphabet.}
\question{14}{Wie verhalten Sie sich beim Empfang von Signalen mit ,,Doppler - Shift``?}
\question{15}{Was versteht man unter ,,Frequenzablage`` bei Relaisbetrieb?}
\question{16}{Nennen Sie drei anormale Ausbreitungsmöglichkeiten im 70 cm-Band oder 2m Band.}
\question{17}{Welche Betriebsverfahren werden im Satellitenfunkverkehr angewendet?}
\question{18}{Was verstehen Sie unter ,,Scatter-Verbindung``?}
\question{19}{Was verstehen Sie unter ,,EME-Verbindung``?}
\question{20}{Was verstehen Sie unter ,,Meteor-Scatter``?}
\question{21}{Was verstehen Sie unter ,,Tropo-Scatter``?}
\question{22}{Was verstehen Sie unter Überreichweiten, was unter dem Funkhorizont?}
\question{23}{Wodurch werden starke Überreichweiten im 70 cm-Band verursacht?}
\question{24}{Wie verhalten Sie sich bei Überreichweitenbedingungen, wenn Sie im Relaisbetrieb arbeiten?}
\question{25}{Wie können Sie sich über die herrschenden Ausbreitungsbedingungen informieren?}
\question{26}{Welche Faktoren beeinflussen die erzielbare Reichweite im 2m-Band?}
\question{27}{Erklären Sie die Bedeutung der auch im Sprechfunk verwendeten Q-Gruppen: QSO - QSY - QRL.}
\question{28}{Erklären Sie die Bedeutung der auch im Sprechfunk verwendeten Q-Gruppen: QRM - QRB - QSB.}
\question{29}{Erklären Sie die Bedeutung der auch im Sprechfunk verwendeten Q-Gruppen: QRT - QSL.}
\question{30}{Erklären Sie die Bedeutung der im Sprechfunk verwendeten Abkürzungen \\ 73- 55- 88- CL.}
\question{31}{Was versteht man unter der Betriebsart ,,Packet-Radio``, welche Betriebsverfahren werden dabei angewendet?}
\question{32}{Welche Faktoren beeinflussen die erzielbare Reichweite im 70cm-Band?}
\question{33}{Was verstehen Sie unter ,,Split-Betrieb``?}
\question{34}{Welche Verfahren werden bei ATV-Betrieb im 70 cm-Band angewendet und welche Besonderheiten sind dabei zu beachten?}
\question{35}{Wie gehen Sie bei der Planung einer Amateurfunkverbindung zu einem bestimmten Ort vor?}
\question{36}{Wie teilen Sie der Gegenstation den Standort ihrer Amateurfunkstelle mit?}
\question{37}{Was ist hinsichtlich der Herstellung oder Veränderung von Geräten für den Amateurfunkverkehr im 2m oder 70 cm-Band zu beachten?}
\question{38}{Sie haben einen abstimmbaren Leistungsverstärker - wie stimmen Sie ihn ab?}

\def\chap{Technische Grundlagen \class{1}}

\card{01}{Ohmsches und Kirchhoff’sches Gesetz}{\small
Ohmsches Gesetz  gibt den Zusammenhang zwischen einem Widerstand  (R) der anliegenden 
Spannung (U) und dem durch den Widerst. fließenden Strom (I) wieder.
\[ U = I \cdot R  \qquad  I = U / R \qquad R = U / I \]
\begin{description}
  \item[1. Kirchhoffsches Gesetz] Parallelschaltung von Widerst., Gesamtstrom = Summe der Teilströme.
  \item[2. Kirchhoffsches Gesetz] Widerst. in Reihe geschaltet, Gesamtspannung = Summe der Teilspannungen.
\end{description}}

\card{02}{Begriff Leiter, Halbleiter, Nichtleiter}{
  \small
  \begin{description}
    \item[Leiter]
      Materialien, die den elektr. Strom sehr gut leiten. Alle Metalle, Kohle und Säuren.
      Beste Leitfähigkeit: Silber, Kupfer, Aluminium, Gold, Messing.
    \item[Halbleiter] Materialien, die Leitfähigkeit
      aufgrund physikalischer oder elektrischer Einflüsse ändern
      (Silizium, Germanium).
    \item[Nichtleiter] Isolatoren leiten schlecht bis gar nicht.
      Keramik, Kunststoff, trockenes Holz.
      Gute Isol.: Glas, Keramik, Teflon, Glasfaser Harz, Gummi.
  \end{description}
}


\card{03}{Kondensator, Begriff Kapazität, Einheiten - Verhalten bei Gleich- und Wechselspannung}{
  \small
  \begin{description}
    \item[Kondensator]
      Ladungsspeicher; besteht aus zwei elektr. leitenden Materialien, durch Isolator getrennt.
      Bei \emph{Gleichspannung} lädt er sich auf und kann später die Ladung an einen Verbraucher abgeben.
      Bei \emph{Wechselspannung} durch die laufende Umladung wird er zu einem Stromfluss im Leitungskreis,
      der mit steigender Frequenz zunimmt.
    \item[Einheit] Farad~(F) für Kapazität  \hspace{20pt} \textbf{Kürzel} C
    \item[Kleinere Einheiten] Milli- ($10^3$) bis Picofarad ($10^{12}$)
  \end{description}
}

\card{04}{Spule, Begriff Induktivität, Einheiten - Verhalten bei Gleich- und Wechselspannung}{
  \footnotesize
  \begin{description}
    \item[Spule] eine oder mehrere Windungen eines Leiters auf einen magnetischen Kern (Induktivität)
    \item[Gleichspannung] baut in der Spule ein Magnetfeld auf
    \item[Wechselspannung] durch den Richtungswechseln des Stromes kommt es zu Richtungswechseln des Magnetfeldes (Selbstinduktion) der dem verursachenden Strom entgegen wirkt.
    Mit steigender Frequenz nimmt Widerstand zu; als induktiver Blindwiderstand~(XL) bezeichnet.
    \item[Einheit] Henry~(H)   \hspace{50pt} \textbf{Formel} (L)
    \item[Kleinere Einheiten] Millihenry, Mikrohenry, PicoH 0,001~H = 1~mH = 1000 microH
  \end{description}
}

\card{05}{Wärmeverhalten von elektrischen Bauelementen}{
  Alle Metalle und die meisten guten Leiter erhöhen mit steigender Temperatur ihren Widerstand.
  \centerline{PTC $\Rightarrow$ positive temperatur coefficient}
  Die meisten Halbleiter verringern mit steigender Temperatur ihren Widerstand.
  \centerline{NTC $\Rightarrow$ negative temperatur coefficient}
  \small
  \begin{description}
    \item[Kenngrößen] gibt an um wie viel Ohm sich der Widerstand ändert, wenn die Temperatur um 1 Grad erhöht wird
    \item[Einheit] Ohm/Grad
  \end{description}
}

\card{06}{Stromquellen (Kenngrössen)}{
  \footnotesize
  \begin{description}
    \item[Gleichstrom Primärbatterien] Durch chemischen Prozess wird elektrische Spannung zwischen zwei Polen erzeugt. Strom kann entnommen werden (Entladung).
    \item[Sekundärbatterien] Akkus vorher aufladen, dann Strom entnehmen.
    \item[Beispiele]
      Bleiakku, Nickel-Cadmium-Akku, Nickel-Metallhybrid-Akku, Lithium-Ionen-Akku,
      Solarzelle, Piezo-Elemente
    \item[Kenngröße] Spannung, Strombelastbarkeit, Kapazität (Fassungsvermögen) in Ah
  \end{description}
  Die 220~V Steckdose liefert Wechselstrom mit 50~Hz.
}

\card{07}{Sinus- und nicht-sinusförmige Signale}{
  \footnotesize
  \begin{description}
    \item[Sinusförmige Signale] haben zeitlichen Verlauf der exakt einer mathemat. Sinusfunktion entspricht und sind frei von Oberwellen (zB Spannung des Wechselstromnetzes).
    \item[Nicht sinusförmige Signale]
      Wechselspannungen mit beliebigem Kurvenverlauf.
      Dreieck-, Rechteck-, Trapez-, Sägezahn-, Rauschsignale: Kombination aus aus mehreren Sinussignalen.
    \item[Kenngrößen]
      \begin{description}
        \item[bei Gleichspannung] Spannung (Amplitude)
        \item[bei Wechselspannung] 3 Kenngrößen: Kurvenform, Scheitelspannung (V), Frequenz (Hz) / Polaritätswechsel/sec
      \end{description}
  \end{description}
}

\card{08}{Was verstehen Sie unter dem Begriff Skin-Effekt?}{
  \small
  Bei zunehmenden Frequenzen wird Stromfluss im Leiter immer mehr zum Rand gedrängt.
  Strom fließt praktisch nur an der Außenhaut.
  Dadurch steigt der Widerstand an, was zu Leistungsverlust führt, nicht bei Gleichstrom.
  Dicke HF Leiter auch als Rohre ausgeführt.
  \begin{description}
    \item[Abhilfe] viele dünne Adern vergrößern die Oberfläche. Dickere Drähte und Versilbern der Leiter
    \item[Größenordnung] Eindringtiefe des Stroms
      9,38~mm bei 50~Hz, 70~$\mu$m bei 1~MHz, 7~$\mu$m bei 100~MHz
  \end{description}
}

\card{09}{Gleich- und Wechselspannung - Kenngrößen}{
  \scriptsize
  \begin{description}\itemsep0pt
    \item[Gleichspannung] Spannung ist konstant, die Polarität verändert sich nicht. \textbf{Kürzel} DC (direct current) \textbf{Kenngrößen} Spannung, Strombelastbarkeit der Quelle, Kapazität in Ah
    \item[Wechselspannung]\itemsep0pt
      Spannung und Polarität ändern sich laufend ($\rightarrow$ Frequenz); der zeitliche Verlauf kann als Kurve dargestellt werden. 
      \begin{description}
        \item[Kürzel] AC (alternating current)
        \item[Kenngröße] Spannung, Amplitude, Frequenz, Kurvenform, Strombelastbarkeit der Quelle
        \item[Formelzeichen] $f = \frac1T$
        \item[Einheit] Hertz (Hz, kHz, MHz)
      \end{description}
  \end{description}
}


\card{10}{Was verstehen Sie unter dem Begriff Permeabilität?}{
  Wird ein Material in eine Spule eingebracht, erhöht dies die Induktivität der Spule.
  Permeabilität ist jene Materialkonstante, die angibt um wie viel höher die Induktivität
  gegenüber Vakuum ist.
  \begin{description}
    \item[Formelzeichen] $\mu$
    \item[Beispiele]
      \begin{tabular}{rlrl}
        Luft       & 1      & Eisen      & 5000 \\
        Aluminimum & 250    & Mu Metall  & 100 000 \\
        Nickel     & 600    &            & \\
      \end{tabular}
  \end{description}
}

\card{11}{Serien- und Parallelschaltung von $R$, $L$, $C$}{
  \begin{minipage}{0.5\textwidth}
    \centering
    Serienschaltung \\ von $R$ und $L$
    \begin{center}
      $ R_{\text{ges}} = R_1 + R_2 $ \\
      $ L_{\text{ges}} = L_1 + L_2 $
    \end{center}
    Parallelschaltung \\ von $R$ und $L$
    \begin{center}
      $ {R_{\text{ges}}} = \frac{R_1 \cdot R_2}{R_1 + R_2} $ \\
      $ {L_{\text{ges}}} = \frac{L_1 \cdot L_2}{L_1 + L_2} $
    \end{center}
  \end{minipage}
  \begin{minipage}{0.49\textwidth}
    \centering
    Parallelschaltung von $C$
    \begin{center}
      $ C_{\text{ges}} = C_1 + C_2 $
    \end{center}
    Serienschaltung von $C$
    \begin{center}
      $ {C_{\text{ges}}} = \frac{C_1 \cdot C_2}{C_1 + C_2} $
    \end{center}
  \end{minipage}
}


\card{12}{Was verstehen Sie unter dem Begriff Dielektrikum?}{
  Isolierende Schicht zwischen den Platten eines Kondensators. z.B. Keramik, Kunststoff; Teflon
  {\small
    \begin{description}
      \item[Kenngößen]
        Dielektritätskonstante, Materialkonstante die angibt um wie viel höher die Kapazität gegenüber Vakuum ist, wenn dieses Material zwischen den Kondensatorplatten angeordnet wird.
      \item[Beispiele] Luft 1, Papier 1--4, Teflon 2, Wasser 80, destilliertes Wasser isoliert
      \item[Eigenschaften] Hohe Dielektritätskonstante, hohe Spannungsfestigkeit, geringe Dicke
    \end{description}
  }
}

\card{13}{Wirk-, Blind- und Scheinleistung bei Wechselstrom.}{
  \begin{description}
    \item[Wirkleistung] nur ohmsche Widerstand (Verbraucher) vorhanden.
    \item[Blindleistung] nur kapazitive oder induktive Verbraucher vorhanden.
    \item[Scheinleistung] ohmsche und (kapazitive oder induktive) Verbraucher vorhanden.
  \end{description}
  {\small
    \textbf{Achtung!}
      Wirk- und Blindleistung können nicht addiert werden,
      da Wirk- und Blindströme nicht gleichphasig sind.
  }
}


\card{14}{Begriff elektrischer Widerstand (Schein-, Wirk- und Blindwiderstand), Leitwert}{
  \small
  \begin{description}
    \item[Ohmscher Widerstand] bei Gleichstrom nur Ohmscher Widerstand,
      keine Phasenverschiebung (,,Wirkwiderstand``),
      Leitwert ist Kehrwert des Ohmschen Widerstands: $G = \frac1{R}$. Einheit Siemens ($S$).
    \item[Blindwiderstand]
      Phasenverschiebung von Strom ($+90^\circ$) und Spannung ($-90^\circ$) bei $C$ und $L$.
      ,,Reaktanz``. Einheit Ohm.
    \item[Scheinwiderstand]
      Phasenverschiebung von 0--$90^\circ$. RC- und RL-Kombinationen. ,,Impedanz``. Einheit Ohm.
  \end{description}
}

\card{15}{Berechnen Sie den induktiven Blindwiderstand einer Spule mit $30~\mu H$ bei $7$ MHz (Werte sind variabel)}{
  \centering
  siehe Skriptum, Seite 39, Frage~T15
}

\card{16}{Berechnen Sie den kapazitiven Blindwiderstand eines Kondensators von 500 pF bei 10 MHz (Werte sind variabel)}{
  \centering
  siehe Skriptum, Seite 38, Frage~T16
}

\card{17}{Der Transformator - Prinzip und Anwendung}{
  \footnotesize
  Gemeinsamer Eisenkern mit 2 Wicklungen (Spulen) fließt Wechselströme in Spule (Primärspeicher).
  Dabei induziert das erzeugte wechselnde Magnetfeld in der 2. Spule (Sekundärspule)
  eine Wechselspannung. Die Wechselspannungen sind proportional zu den Windungszahlen
  = Übersetzungsverhältnis.
  \begin{description}
    \item[Anwendung] Stromversorgungs-, NF- und HF-Technik
    \item[Übertrager] anderes Wort für Transformator
    \item[Kenndaten]
      Primär- / Sekundärspannung, Windungszahlen, Übersetzungsverhältnis,
      maximal übertragbare Leistung, Impedanz
  \end{description}
}

\question{18}{Der Resonanzschwingkreis - Kenngrößen}
\question{19}{Der Resonanzschwingkreis - Anwendungen in der Funktechnik}
\question{20}{Berechnen Sie die Resonanzfrequenz eines Schwingkreises mit folgenden Werten: L = 15 H, C = 30 pF (Werte sind variabel)}
\question{21}{Filter – Arten, Aufbau, Verwendung und Wirkungsweise}
\question{22}{Was sind Halbleiter?}
\question{23}{Die Diode - Aufbau, Wirkungsweise und Anwendung}
\question{24}{Der Transistor - Aufbau, Wirkungsweise und Anwendung}
\question{25}{Die Elektronenröhre - Aufbau, Wirkungsweise und Anwendung}
\question{26}{Arten von Gleichrichterschaltungen - Wirkungsweise}
\question{27}{Stabilisatorschaltungen}
\question{28}{Hochspannungsnetzteil - Aufbau, Dimensionierung und Schutzmaßnahmen}
\question{29}{Welche Arten von digitalen Bauteilen kennen Sie? - Wirkungsweise}
\question{30}{Was sind elektronische Gatter? - Wirkungsweise}
\question{31}{Messung von Spannung und Strom am Beispiel eines vorgegebenen Stromkreises}
\question{32}{Erklären Sie die prinzipielle Wirkungsweise eines Griddipmeters, Anwendung und Funktion}
\question{33}{Erklären Sie die Funktionsweise eines HF-Wattmeters}
\question{34}{Erklären Sie die Funktionsweise eines Oszillografen (Oszilloskop)}
\question{35}{Erklären Sie die Funktionsweise eines Spektrumanalysators}
\question{36}{Begriff Demodulation}
\question{37}{Zeichnen Sie das Blockschaltbild eines Überlagerungsempfängers}
\question{38}{Was verstehen Sie unter Spiegelfrequenz und Zwischenfrequenz?}
\question{39}{Erklären Sie die Kenngrößen eines Empfängers - Empfindlichkeit, intermodulationsfreier Bereich, Eigenrauschen}
\question{40}{Erklären Sie den Begriff des Rauschens. - Auswirkungen auf den Empfang.}
\question{41}{Mischer in Empfängern - Funktionsweise und mögliche technische Probleme}
\question{42}{Nichtlineare Verzerrungen - Ursachen und Auswirkungen}
\question{43}{Empfängerstörstrahlung - Ursachen und Auswirkungen}
\question{44}{Mikrofonarten - Wirkungsweise}
\question{45}{Prinzip, Arten und Kenngrößen der Einseitenbandmodulation}
\question{46}{Prinzip, Arten und Kenngrößen der Pulsmodulation}
\question{47}{Erklären Sie die wichtigsten Anwendungen der digitalen Modulationsverfahren}
\question{48}{Erklären Sie die Begriffe CRC und FEC}
\question{49}{Prinzip und Kenngrößen der Frequenzmodulation}
\question{50}{Prinzip und Kenngrößen der Amplitudenmodulation}
\question{51}{Erklären Sie den Begriff Modulation (analoge und digitale Verfahren)}
\question{52}{Oszillatoren - Grundprinzip, Arten}
\question{53}{Erklären Sie den Begriff VCO}
\question{54}{Erklären Sie den Begriff PLL}
\question{55}{Erklären Sie den Begriff DSP}
\question{56}{Erklären Sie die Begriffe sampling, anti aliasing filter, ADC/DAC}
\question{57}{Merkmale, Komponenten, Baugruppen eines Senders}
\question{58}{Zweck von Puffer- und Vervielfacherstufen, Aufbau}
\question{59}{Aufbau einer Senderendstufe, Leistungsauskopplung}
\question{60}{Anpassung eines Senderausganges an eine symmetrische oder asymmetrische Antennenspeiseleitung}
\question{61}{Der Antennentuner, Wirkungsweise, 2 typische Beispiele}
\question{62}{Antennenzuleitungen - Aufbau, Kenngrößen}
\question{63}{Erklären Sie den Begriff Balun. Aufbau, Verwendung und Wirkungsweise}
\question{64}{Der Dipol - Aufbau, Kenngrößen und Eigenschaften}
\question{65}{Die Vertikalantenne - Aufbau, Kenngrößen und Eigenschaften}
\question{66}{Gekoppelte Antennen - Aufbau, Kenngrößen und Eigenschaften}
\question{67}{Strahlungsdiagramm einer Antenne}
\question{68}{Die Yagi-Antenne - Aufbau, Kenngrößen und Eigenschaften}
\question{69}{Breitbandantennen - Aufbau, Kenngrößen und Eigenschaften}
\question{70}{Die Parabolantenne - Aufbau, Kenngrößen und Eigenschaften}
\question{71}{Erklären Sie den Begriff Wellenwiderstand}
\question{72}{Stehwellen und Wanderwellen, Ursachen und Auswirkungen}
\question{73}{Strahlungsfeld einer Antenne, Gefahren}
\question{74}{Aufbau und Kenngrößen eines Koaxialkabels}
\question{75}{Erklären Sie den Begriff Dezibel am Beispiel der Anwendung in der Antennentechnik}
\question{76}{Was versteht man unter Richtantennen, Anwendungsmöglichkeiten}
\question{77}{Welche Kenngrößen von Antennen kennen Sie und wie können sie gemessen werden?}
\question{78}{Dimensionieren Sie einen Halbwellendipol für f = 3.6 MHz ; V = 0.97 (Werte sind variabel)}
\question{79}{Bestimmen Sie die effektive Strahlungsleistung bei folgenden Gegebenheiten: Senderleistung: 200 Watt; Dämpfung der Antennenleitung: 6 dB/100m; Kabellänge : 50 m; Gewinn: 10 dB (Werte sind variabel)}
\question{80}{Bestimmen Sie die effektive Strahlungsleistung bei folgenden Gegebenheiten: Senderleistung 100 Watt; Dämpfung der Antennenleitung 12 dB/100m; Kabellänge 25 m; Rundstrahlantenne mit Gesamtwirkungsgrad von 50 \% (Werte sind variabel)}
\question{81}{Langdrahtantennen - Aufbau, Kenngrößen und Eigenschaften}
\question{82}{Zweck von Radials / Erdnetz bei Vertikalantennen - Dimensionierung}
\question{83}{Blitzschutz für Antennenanlagen}
\question{84}{Sicherheitsabstände bei Antennen}
\question{85}{Erklären Sie den Begriff ,,elektromagnetisches Feld``. Kenngrößen?}
\question{86}{Begriff elektrisches und magnetisches Feld; Abschirmmaßnahmen für das elektrische bzw. das magnetische Feld?}
\question{87}{Erklären Sie den Begriff ,,EMV`` und dessen Bedeutung im Amateurfunk}
\question{88}{Erklären Sie den Begriff ,,EMVU`` und dessen Bedeutung im Amateurfunk}
\question{89}{Erklären Sie den Begriff ,,Trap``, Aufbau und Wirkungsweise}
\question{90}{Was versteht man unter einem Hohlraumresonator, Anwendung.}
\question{91}{Funkentstörmaßnahmen im Bereich Stromversorgung der Amateurfunkstelle}
\question{92}{Funkentstörmaßnahmen bei Beeinflussung durch hochfrequente Ströme und Felder}
\question{93}{Was sind Tastklicks, wie werden sie vermieden?}
\question{94}{Erklären Sie die Begriffe: ,,Unerwünschte Aussendungen``, ,,Ausserbandaussendungen``, ,,Nebenaussendungen`` (spurious emissions)}
\question{95}{Erklären Sie den Begriff: ,,Splatter`` - Ursachen und Auswirkungen}
\question{96}{Erklären sie den Begriff ,,schädliche Störungen``}
\question{97}{Prinzipieller Aufbau einer Relaisfunkstelle und einer Bakenfunkstelle}
\question{98}{Definieren Sie den Begriff ,,Senderleistung``}
\question{99}{Definieren Sie den Begriff ,,Spitzenleistung``}
\question{100}{Definieren Sie den Begriff ,,belegte Bandbreite``}
\question{101}{Definieren Sie den Begriff ,,Interferenz in elektronischen Anlagen``; beschreiben Sie Ursachen und Gegenmassnahmen}
\question{102}{Erklären Sie die Begriffe ,,Blocking``, ,,Intermodulation``}
\question{103}{Welche Gefahren bestehen für Personen durch den elektrischen Strom?}
\question{104}{Was ist beim Betrieb von Hochspannung führenden Geräten zu beachten?}
\question{105}{Definieren Sie die Gefahren durch Gewitter für die Funkstation und das Bedienpersonal, beschreiben Sie Vorbeugemassnahmen}

\def\chap{Technische Grundlagen \class{3,4}}

\card{01}{In welchem Zusammenhang stehen die Größen Strom – Spannung - Widerstand in einem Stromkreis?}{
  Damit Strom fließen kann, müssen zwischen zwei Polen eine Spannung und eine leitende Verbindung vorhanden sein. Je höher die Spannung, umso mehr Strom fließt. Der Widerstand behindert die elektrische Ladung. Mehr Widerstand bedeutet bei gleicher Spannung, dass weniger Strom fließt.
  \begin{description} \itemsep1pt
    \item[Maßzahl] Ohm
    \item[Symbol] $R$
    \item[Formel] $R = \frac{U}{I}$
  \end{description}
}
\card{02}{Was versteht man unter einem Kurzschluß - wie entsteht er?}{
  Wenn der Widerstand eines Verbrauchers $0$ ist, kann so viel Strom fließen, dass die Leitungen oder die Stromquellen Schaden nehmen. Sicherungen trennen bei einem Kurzschluss den Stromkreis von der Stromquelle.
}
\card{03}{Nennen Sie Stromquellen}{
  \small
  \begin{description}
    \item[Primärbatterien] Spannung zwischen Polen entsteht durch einen chemischen Prozess. Strom kann entnommen werden. Entladung ist nicht umkehrbar.
    \item[Sekundärbatterien] Spannung zwischen Polen entsteht durch einen chemischen Prozess. Strom kann entnommen werden. Entladung ist umkehrbar (Ladevorgang).
    \item[230V Steckdose] liefert 50 Hz Wechselstrom
    \item[Kenngrößen]
      \begin{itemize*}
        \item Spannung
        \item Strombelastbarkeit
        \item Kapazität in Ah
      \end{itemize*}
  \end{description}
}
\card{04}{Kenngrößen einer Gleichstromquelle. Kenngrößen einer Wechselstromquelle - Gefahrengrenze?}{
  \small
  \begin{description}
    \item[Gleichstrom] Die Spannung ist konstant, Polarität verändert sich nicht.

      \begin{itemize*}
        \item Spannung
        \item Strombelastbarkeit der Quelle
        \item Kapazität in Ah (Batterie, Akkus)
      \end{itemize*}
    \item[Wechselstrom] Spannung und Polarität ändern sich laufend, Kurvendarstellung 

      \begin{itemize*}
        \item Spannung (Amplitude)
        \item Frequenz
        \item Kurvenform (Signalform)
        \item Strombelastbarkeit der Quelle
      \end{itemize*}
  \end{description}
  Die Gefahrengrenze liegt bei 25~V, Lebensgefahr besteht bei 40~V.
}
\card{06}{Nennen Sie die wichtigsten Eigenschaften von Ohm'schen Widerständen, Induktivitäten und Kapazitäten.}{
  \footnotesize
  \begin{description}
    \item[Widerstand] Hemmung entgegen Stromfluss. Abhängig von Material und Maßen des Leiters. Widerstand steigt mit Länge und abnehmendem Durchmesser des Leiters.
    \item[Spule] Einheit Henry $H$. Bei Gleichspannung ein ohmscher Widerstand. Bei Wechselspannung auch ein induktiver Blindwiderstand. Höhere Frequenz führt zu größerem Blindwiderstand
    \item[Ladungsspeicher] zwei gegenüberstehenden Metallplatten. Einheit Farad $F$. Nur bei Wechselspannung fließt Strom. Höhere Frequenz bedeutet kleinerer kapazitiver Blindwiderstand
  \end{description}
}
\card{07}{Was verstehen Sie unter dem Begriff ,,Fehlanpassung``?}{Eine Fehlanpassung liegt vor, wenn die Anpassungsbedingungen bei Strom-, Spannungs- und Leistungsanpassung nicht erfüllt sind.}
\card{08}{Was verstehen Sie unter dem Begriff ,,Transformation``?}{Transformation ist der allgemeine Begriff für ,,Wandlung`` (zB. Spannungstransformation, Impedanztransformation). Auf- oder Abwärtstransformation von Wechselspannungen in der Stromversorgungs-, Niederfrequenz- und Hochfrequenztechnik.}
\card{09}{Prinzipieller Aufbau eines Kommunikationssystems. Erläutern Sie die Wirkungsweise von Mikrophon und Lautsprecher bzw. Kopfhörer.}{
  \small
  \begin{itemize*}
    \item Signal- Eingabegerät (Mikrophon)
    \item Sender
    \item Antennenanpassgerät
    \item Antenne
    \item Empfänger
    \item Signal Ausgabegerät (Kopfhörer)
  \end{itemize*}
  \vspace{10pt}

  Ein Mikrophon ist ein Schallwandler, der Schall in elektrische Spannungsänderungen als Signal umwandelt. Ein Wandler - gekoppelt mit einer Membran - generiert Tonfrequenz-Wechselspannung oder eine pulsierende Gleichspannung. Ein Lautsprecher ist ein Wandler der elektrische Signale in Schall (Ton) umwandelt. Tonerzeugung in für Menschen hörbaren Frequenzbereichen.
}
\card{11}{Prinzipieller Aufbau eines Senders}{
  \small
  \begin{minipage}{0.4\textwidth}
  \begin{itemize}
    \item Oszillator (CO oder VFO)
    \item Modulator
    \item Pufferstufe
  \end{itemize}
  \end{minipage}
  \begin{minipage}{0.5\textwidth}
  \begin{itemize}
    \item Frequenzvervielfacher
    \item Treiber
    \item Endstufe
  \end{itemize}
  \end{minipage}
  \vspace{10pt}

  Moderne Sender arbeiten nach dem Überlagerungsprinzip, allerdings verläuft der Signalweg in umgekehrte Richtung. Viele Baugruppen sind für das Senden und Empfangen nutzbar, deshalb ist dieses Konzept in Sendeempfängern (,,Transceiver``) verbreitet.
}
\card{12}{Funktionsprinzip des Oszillators}{
  Ein Oszillator erzeugt ein Wechselspanungssignal gewünschter Frequenz und Kurvenform.
  Jeder Oszillator ist ein Verstärker, bei dem ein Teil des Ausgangssignals wieder an den Eingang zurückgeführt wird. Dadurch kommt es zur Selbsterregung (Rückkopplung). Befindet sich im Rückkopplungsweg ein frequenzbestimmtes Bauteil (Filter), meist ein Schwingkreis (oder Quarz), so kann Selbsterregung nur auf dessen Resonanzfrequenz stattfinden.
}
\card{13}{Prinzipieller Aufbau eines Empfängers}{
  In einem Empfänger wird das NF-Modulationssignal aus dem modulierten HF-Signal zurückgewonnen.
  Die einfachsten Bauweisen bestehen aus einem Filter, HF-Verstärker, Demodulator, NF-Verstärker.
  Demodulator bezeichnet eine Baugruppe, die der Wiedergewinnung des Modulationssignals aus dem HF-Signal dient. Je nach Modulationsart ist der Demodulator unterschiedlich aufgebaut.
}
\card{14}{Prinzip des Überlagerungsempfängers. Was verstehen Sie unter dem Begriff Zwischenfrequenz?}{
  Bandfilter; Verstärkung im HF-Verstärker; Signale werden im Mischer mit Signal eines VFO gemischt; Filter wird ZF herausgefiltert und zu ZF Verstärker; Produktdetektor erfolgt Mischung mit Signal des BFO; Aus Mischprodukt wird NF-Signal verarbeitet. Über NF Verstärker und NF Endstufe zu Lautsprecher.
  Mischung von zwei HF-Signalen, entstehen 2 neue Signale (Summe oder Differenz). Ein Mischprodukt kann gefiltert werden und weiter verarbeitet- Zwischenfrequenz} % TODO
\card{16}{Was verstehen Sie unter dem Begriff Modulation?}{
  Modulation ist ein zentraler Begriff jeder technischen Form von Nachrichtenübertragung. Man muss zwischen dem Träger, der dauernd ausgesandt wird (zB. elektromagnetische Strahlung), und dem eigentlichen Signal, das mittels des Trägers übertragen werden soll, unterscheiden. Modulation bezeichnet den Vorgang, bei dem einen hochfrequenten Träger ein NF Signal aufgeprägt wird. Es gibt analoge und digitale Verfahren der Modulation.}
\card{17}{Kenngrößen der Amplitudenmodulation}{
  \begin{description}
    \item[Modulationsgrad] \hfill{} \\ (NF-Amplitude / HF-Amplitude) $\cdot$ 100 (\%)
    \item[Bandbreite]
      2~fm wobei fm die maximale zu übertragende Frequenz des Modulationssignales ist.
      Im Amateurfunk wird die Amplitudenmodulation auf den Kurzwellenbändern benützt.
  \end{description}
}
\card{18}{Kenngrößen der Frequenzmodulation}{
  \begin{description}
    \item[Frequenzhub] die maximale Ablenkung der Trägerfrequenz von der Grundfrequenz in kHz,
      im Amateurfunk: 5 kHz
    \item[Modulationsindex]
      Frequenzhub (kHz) / Modulationsfrequenz (kHz)
      Im Amateurfunk wird die Frequenzmodulation auf den 2~m und 70~cm Bändern benützt.
      Der Frequenzhub beträgt in der Regel 5~kHz. Die Modulationsfrequenz beträgt 3~kHz.
      Modulationsindex von $\frac53 = 1,7$
  \end{description}
}
\card{19}{Definieren Sie den Begriff ,,belegte Bandbreite``. Arten und Vorteile der Einseitenbandmodulation?}{
  \small
  Jene Frequenzbandbreite, bei der die unterhalb und oberhalb ihrer Frequenzgrenzen ausgesendeten mittleren Leistungen 0,5~\% der gesamten mittleren Leistung einer gegebenen Aussendung betragen.
  Der Vorteil der Einseitenbandmodulation liegt in der weit günstigeren Leistungausbeute und der halben Bandbreite. Beides ergibt eine geringere Störanfälligkeit der Signalübertragung.

  Methoden:
  \begin{itemize}
    \item Filtermethode
    \item Phasenmethode
  \end{itemize}
}
\card{21}{Begriff Dezibel (Werte fragen: zB 3~dB, 6~dB, 10~dB, 30~dB Leistungssteigerung)}{
  {\small
    Dezibel ist ein logarithmisches Maß für das Verhältnis von zwei gleichartigen Leistungsgrößen
    $P_1$ und $P_2$ bzw. Spannungsgrößen $U_1$ und $U_2$.}

  {\scriptsize
    \begin{minipage}{0.45\textwidth}
      \vspace{5pt} Leistungsverhältnisse in dB\vspace{-8pt}
      \begin{description}\itemsep0pt
        \item[3~dB] 2fach
        \item[6~dB] vierfach
        \item[10~dB] zehnfach
        \item[13~dB] 20-fach
        \item[20~dB] 100-fach
        \item[-3~dB] halb
        \item[-10~dB] ein Zehntel
      \end{description}
      \end{minipage}
      \begin{minipage}{0.5\textwidth}
      Spannungsverhältnisse in dB:
      \begin{description}
        \item[6~dB] doppelte Spannung
        \item[12~dB] vierfache Spannung
        \item[20~dB] zehnfache Spannung
        \item[-6~dB] halbe Spannung
      \end{description}
    \end{minipage}
  }
}
\card{22}{Was ist eine Diode - Wirkungsweise, Verwendung?}{
  Eine Diode ist ein Halbleiterbauelement mit einem P-N Übergang. Die P-Schicht bildet die Anode, die N-Schicht die Kathode. Die Anwendung erfolgt als Gleichrichter, da Strom nur in einer Richtung fließen kann.
  \begin{description}
    \item[Durchlassrichtung] +Pol der Stromquelle an der Anode
    \item[Sperrrichtung] +Pol der Stromquelle an der Kathode (durch Ring gekennzeichnet)
  \end{description}
}
\card{23}{Was ist ein Transistor - Wirkungsweise, Verwendung?}{
  \small
  Ist ein Halbleiterbauelement, aus zwei N-Leitern, und dünnen Schicht eines P-Leiters, Emitter-Basis-Kollektor. Zwischen Basis, Emitter und Kollektor bilden sich zwei Sperrschichten. Weil Basis schwach dotiert ist, können Elektronen bei fließendem Basisstrom auch die B-K Sperrschicht überwinden und über Kollektor abfließen. Transistor verhält sich wie elektrisch gesteuerter Widerstand zwischen E und K.

  \begin{itemize}\itemsep-1pt
    \item NF/HF Verstärker
    \item Schalter
    \item Oszillatoren
  \end{itemize}
}
\card{24}{Was versteht man unter ,,AGC`` und ,,AFC``? Erklären Sie die Empfängerkenngrößen - Empfindlichkeit, Eigenrauschen, Empfangsmischprodukte}{
  \footnotesize
  \begin{description}
    \item[AGC] Lautstärke des NF-Signals eines Empfängers konstant gehalten. Notwendig, da Amplituden von Antenne kommende Signale Bereich von 120~dB übersteigen können.
    \item[AFC] Aus FM Demodulator Nachstimmspannung gewonnen, zur Nachstimmung der Oszillator-Frequenz,Schwankungen Empfangsfrequenz ausgeglichen.
  \end{description}
  \begin{enumerate}
    \item kleinster Signalpegel Empfangen werden kann
    \item Rauschquellen aller Bauteile, kein Eingangssignal
    \item Empfangsfrequenz gemischt- 2 Mischprodukte entstehen
  \end{enumerate}
}
\card{26}{Was versteht man unter dem S/N - Verhältnis?}{
  Das Zahlenverhältnis von Signalpegel zu Rauschpegel. S/N wird in dB angegeben und auch zur Messung der Grenzempfindlichkeit von Empfängern benützt.
  Ein S/N von 3~dB bedeutet, dass die Signalamplitude 1,4~mal größer als die Rauschamplitude ist.
}
\card{27}{Erklären Sie die Begriffe ,,digital`` und ,,analog``.}{
  Ein analoges Signal kann zwischen den Spitzenwerten jeden beliebigen Zwischenwert annehmen. Die Verarbeitung setzt Linearität voraus. Lautsprecher und Kopfhörer benötigen analoge Signale.

  Digitale Signale weisen nur zwei (binäre; 0 oder 1) Spannungszustände auf und keine Zwischenwerte. Zur Verarbeitung ist Linearität nicht erforderlich. Nichtlinearität ist sogar von Vorteil. Beispiel Lichtschalter: ,,An`` oder ,,Aus``}
\card{28}{Was versteht man unter der Ausgangsleistung, was unter der Verlustleistung?}{
  Die Ausgangsleistung ist jene Leistung, die ein Sender an eine definierte Schnittstelle abgibt (Sendeausgangsbuchse, meist 50~Ohm). Durch den nicht 100~\%igen Wirkungsgrad eines Senders muss der Sender bei einer vorgegebenen Ausgangsleistung mehr Energie zugeführt werden, als er abgeben kann. Die Differenz zwischen zugeführter und abgegebener Leistung (Ausgangsleistung) wird als Verlustleistung bezeichnet.
}
\card{29}{Was versteht man unter der Strahlungsleistung? (Beispiel vorgeben, zB. Sender mit 10 W Ausgangsleistung; Antennenkabel mit 3 dB Dämpfung; Antenne mit 10 dB Gewinn)}{
  \small
  Die \emph{effektive Strahlungsleistung} ergibt sich aus der in eine Sendeantenne eingespeisten Leistung, vermehrt um den Antennengewinn in dB. ERP bezieht sich auf einen Halbwellendipol. Bezieht man den Antennengewinn auf den Isotropstrahler ($\Rightarrow$ dBi), spricht man von EIRP (Watt):
  \[ \text{EIRP} = \text{ERP} \cdot 1,64 \]
  \[ \text{ERP} = 10~W- 3~dB + 10~dB = 10 \cdot 0,5 \cdot 10 = 50~W \]
  Die Strahlungsleistung beträgt $\text{ERP} = 50~W$
  \[ \text{EIRP} = 50 \cdot 1,64 = 82~W \]
}
\card{30}{Begriff Speiseleitung (Antennenzuleitung) - Kenngrößen?}{
  \footnotesize
  \begin{description}
    \item[Symmetrische Speiseleitung]
      Zweidrahtleitungen (Paralleldrahtleitung).
      2 Leiter mit isolierendem Abstandshalter.
    \item[Unsymmetrische Speiseleitung]
      Koaxialkabel. Konzentrische Anordnung Innenleiter, Dielektrikum, Außenleitergeflecht,
      Außenisolation
  \end{description}
  \begin{description}
    \item[Hohlleiter] Rechteckige / runde Rohre ohne Innenleiter (Verwendung im GHz-Bereich).
    \item[Elektrische Kenngrößen]
      \begin{itemize*}
        \item Impedanz
        \item Dämpfung
        \item Verkürzungsfaktor
        \item Belastbarkeit
      \end{itemize*}
    \item[Mechanische Kenngröße]
      \begin{itemize*}
        \item Durchmesser
        \item Gewicht
        \item Zugfestigkeit
      \end{itemize*}
  \end{description}
}
\card{31}{Auswirkung(en) des Stehwellenverhältnisses (SWR)?}{
  Bei Fehlanpassung wird ein Teil der Leistung am fehlangepassten fernen Ende reflektiert, läuft zurück und wird am nahen Ende teilweise reflektiert. Die Überlagerung von hin- und rücklaufenden Wellen führt zu Stehwellen. Es kommt zur Überlastung der Endstufe und zu einem zusätzlichen Leistungsverlust auf der fehlangepassten Leitung. Die Reflektionsverluste bei hohem SWR sind Verluste auf realen Leitungen.
}
\card{32}{Kenngrößen einer Antenne am Beispiel des Dipols}{
  \begin{itemize}
    \item Wellenwiderstand im Speisepunkt: ca. 50~Ohm, Speisung mit Koaxialkabel und Balun
    \item Strahlungsdiagramm: hat die Form einer Acht, d.h. Strahlungsmaxima quer zur Antennenachse, axiale Minima
    \item Gewinn: 2,15~dB in Hauptstrahlrichtung
    \item Im Amateurfunk werden häufig gestreckte und abgewinkelte Dipole verwendet
  \end{itemize}
}
\card{33}{Vertikalantenne - Eigenschaften}{
  Vertikalantennen sind senkrecht zur Erdoberfläche angeordnete Antennen, deren Strahlung vertikal polarisiert ist. Im Resonanzfall zeigen Viertelwellenstrahler einen Fußpunktwiderstand von etwa 30~Ohm. Das horizontale Strahlungsdiagramm zeigt die Charakteristik eines Rundstrahlers, die vertikale Charakteristik ist stark von den umgebenden Untergrundeigenschaften abhängig. Werden als Mobilantennen verwendet.}
\card{34}{Die Yagi-Antenne - Aufbau, Eigenschaften, Kenngrößen}{
  Form der Richtantenne im KW/UKW Bereich.
  Resonanter Halbwellendipol wird durch zwei oder mehrere Elemente ähnlicher Länge ergänzt. Längeres Element als Reflektor, kürzere als Direktor bezeichnet. Neben Reflektoren kann man beliebig viele Direktoren verwenden. Yagi-Antenne zeigt eine einseitige Richtwirkung, Bündelung Richtung kürzeren Elemente. Mehr Direktoren - größere Richtwirkung:

  \begin{itemize*}
    \item Frequenz 
    \item Impedanz 
    \item Gewinn (dB) 
    \item Strahlungsdiagramm 
    \item Vor/Rückverhältnis
  \end{itemize*}
}
\card{35}{Dipolkombinationen (Zeilen, Spalten)}{
  Einfacher Dipol-Achter Charakteristik. Kombination von Dipolen untereinander (Spalten) oder nebeneinander (Zeilen) kann Antennencharakteristik verändern - Gewinn steigt. Kombiniert man Spalten und Zeilen zu Antennenfläche - erfolgt Strahlungseinzug nicht nur einer Ebene, sondern räumlich entsteht Diagramm einer ,,Doppelzigarre``. Diagrammform und Gewinn vom Abstand Dipole untereinander, Verhältnis der Ströme und Phasenwinkel zwischen Strömen abhängig.
}
\card{36}{Die Parabolantenne - Aufbau, Eigenschaften, Kenngrößen}{
  \small
  Im UKW/UHF Bereich verwendet.
  Hinter Strahler Parabolspiegel aus Metall angebracht. Durchmesser des Spiegels muss gegenüber Wellenlänge groß sein. Strahler im Brennpunkt des Spiegels angebracht. Oft Strahler selbst eine Richtantenne die auf den Spiegel zeigt. Parabolantenne zeigt ausgeprägte Richtwirkung. Strahlungskeule nur Winkelgrad, Ausrichtung muss sehr präzise sein.

  \vspace{5pt}
  \begin{minipage}{0.5\textwidth}
    \begin{itemize}
      \item Frequenz 
      \item Gewinn 
      \item Strahlungsdiagramm 
    \end{itemize}
  \end{minipage}
  \begin{minipage}{0.49\textwidth}
    \begin{itemize}
      \item Öffnungswinkel 
      \item {\footnotesize Rück/Seitendämpfung}
    \end{itemize}
  \end{minipage}
}
\card{37}{Mobilantennen - Aufbau, Eigenschaften, Kenngrößen, Montageort}{
  \small
  Verbreitet sind Viertelwellenstrahler, die aus einem Element bestehen. Zum Dipol wird die fehlende Hälfte durch Gegengewicht, zB. Fahrzeugkarosserie, ersetzt. UKW-Bereich Verlängerung nicht nötig. Im KW-Bereich induktiv verlängerte Antennen. Resonanzfall zeigen sie Fußpunktwiderstand ca. 30~Ohm. Horizontale Strahlungsdiagramm-Charakteristik Rundstrahlers, vertikale Charakteristik-Untergrundeigenschaften abhängig.

  \begin{minipage}{0.45\textwidth}
    \begin{itemize}\itemsep0pt
      \item Frequenz
      \item Gewinn
      \item Bauhöhe
    \end{itemize}
  \end{minipage}
  \begin{minipage}{0.5\textwidth}
    \begin{itemize}\itemsep0pt
      \item Gegengewicht
      \item Bandbreite
    \end{itemize}
  \end{minipage}
}
\card{38}{Grundausrüstung einer Amateurfunkstelle für Sprechfunk (Komponenten)}{
  \footnotesize
  \begin{minipage}{0.49\textwidth}
    \begin{itemize}
      \item Mikrofon
      \item PC mit Soundkarte (wahlweise zur Logbuchführung)
      \item Leistungsverstärker (wahlweise im Rahmen der Vorschriften)
      \item Antennentuner (wahlweise nach technischen Erfordernissen, vornehmlich auf Kurzwelle)
    \end{itemize}
  \end{minipage}
  \begin{minipage}{0.5\textwidth}
    \begin{itemize}
      \item Sender / Empfänger
      \item Sendeantenne / Empfangsantenne
      \item Lautsprecher / Kopfhörer
      \item Mess- und Kontrollgeräte, Blitzschutz (nach Maßgabe der geltenden Vorschriften)
    \end{itemize}
  \end{minipage}
}
\card{39}{Grundausrüstung einer Amateurfunkstelle für Packet Radio}{
  \begin{itemize}\itemsep1pt
    \item PC mit Soundkarte
    \item Modem / Controller
    \item Sender / Empfänger
    \item Leistungsverstärker (wahlweise im Rahmen der Vorschriften)
    \item Sendeantenne / Empfangsantenne
    \item Mess- und Kontrollgeräte, Blitzschutz (nach Maßgabe der geltenden Vorschriften)
  \end{itemize}
}
\card{40}{Grundausrüstung einer Amateurfunkstelle für ATV-Betrieb}{
  \begin{itemize}\itemsep1pt
    \item TV Kamera
    \item Sender / Empfänger
    \item Leistungsverstärker (wahlweise im Rahmen der Vorschriften)
    \item Sendeantenne / Empfangsantenne
    \item TV Monitor
    \item Mess- und Kontrollgeräte, Blitzschutz (nach Maßgabe der geltenden Vorschriften)
  \end{itemize}
}
\card{41}{Was versteht man unter Betriebserde; was unter Blitzschutzerde?}{
  Die \emph{Betriebserde} dient der Schutzmaßnahme (FI-Schalter, Nullung) und darf nicht für die Blitzableitung verwendet werden.

  Die \emph{Blitzschutzerde} stellt eine Schutzmaßnahme gegen Blitzeinwirkungen dar. Diese ist regelmäßig auf Funktionstüchtigkeit zu überprüfen. Neben den äußeren Blitzschutz des Gebäudes und der Antennenanlage sind die Antennenzuleitungen bei Beendigung des Funkbetriebes zu erden, daher mit dem Gebäudeblitzschutz zu verbinden.
}
\card{42}{Was versteht man unter BCI, TVI?}{
  \begin{description}
    \item[BCI] Störungen des Rundfunkempfanges durch eine andere Funkstelle. BCI wird durch Einstrahlung in die Empfangsantennenanlage, die Antennenzuleitung oder direkte Einstrahlung in den Rundfunkempfänger verursacht.
    \item[TVI] Störungen des Fernsehempfanges. Auch hier erfolgt die Einstrahlung in die Antennenanlage, die Zuleitungen oder direkt in den Fernsehempfänger. Besonders FS-Verstärkeranlagen und Hausverteiler sind gegen Einstrahlung anfällig.
  \end{description}
}
\card{43}{Maßnahmen gegen BCI, TVI?}{
  Gegen BCI und TVI richten sich die notwendigen Maßnahmen nach der Ursache der Störung.
  Grundsätzlich ist die Amateurfunkstelle so zu errichten und zu betreiben, dass Störungen anderer Funkdienste vermieden werden.
  Dies wird durch eine entsprechend ober- und nebenwellenfreies Sendesignal und der Einhaltung der zulässigen Sendeleistung sichergestellt.
}
\card{44}{Was versteht man unter dem ``SQUELCH`` - wozu dient er?}{
  Unter \emph{Squelch} versteht man eine Rauschsperre bei FM-Empfängern, wenn kein HF-Signal empfangen wird. Der NF-Verstärker wird ,,stumm`` geschaltet, wenn das Eingangssignal unter einer gewissen Schwelle (einstellbar am Gerät) liegt.
}
\card{45}{Wie bestimmt man die Resonanzfrequenz einer Antenne?}{
  Die \emph{Resonanzfrequenz} einer Antenne wird mit dem \emph{Griddipmeter} bestimmt.
  Dabei nähert man sich dem zu untersuchenden Schwingkreis mit der Koppelspule des Messgerätes an und durch Verändern der Oszillatorfrequenz des Griddipmeters wird diesem bei Resonanz mit dem Prüfling Energie entzogen. Das kann an einem Messinstrument (Rückgang des Gitterstroms) abgelesen werden. Somit kann die Frequenz festgestellt werden.
}
\card{46}{Was ist ein SWR-Meter, wo und wie wird es eingesetzt?}{
  Unter einem \emph{SWR-Meter} versteht man ein Messgerät zur Messung von Stehwellen. Das SWR wird in die Antennenzuleitung unmittelbar nach dem Antennenausgang eingeschliffen. Mit Hilfe des SWR-Meters kann festgestellt werde, ob auf der Antennenleitung stehende Wellen auftreten, daher der Antennenfußpunktwiderstand nicht mit dem Wellenwiderstand des Antennekabels übereinstimmt.
  Das SWR-Meter wird zur Abstimmung eines Antennenanpassgerätes benötigt.
}
\card{47}{Was versteht man unter einem ``Antennen-Tuner``?}{
  Der Antennentuner sitzt idealerweise an der Antennenschnittstelle und dient der Transformation der Kabelimpedanz auf die Impedanz des Antennenspeisepunktes.
}
\card{48}{Was versteht man unter ``Dopplershift``?}{
  Auf Grund der großen orbitalen Geschwindigkeit eines Satelliten ändern sich die Uplink und Downloadfrequenzen für die Bodenstation während seines Überflugs. Dieses Phänomen wird als \emph{Dopplershift} (auf Basis des Doppler-Effekts) bezeichnet.}
\card{49}{Komponenten einer Amateurfunkstation für Satellitenfunk}{
  \footnotesize
  \begin{minipage}{0.43\textwidth}
    \begin{itemize}\itemsep1pt
      \item Mikrofon
      \item Sende-/Empfangsantenne
      \item Lautsprecher
      \item Leistungsverstärker (im Rahmen der Vorschriften)
    \end{itemize}
  \end{minipage}
  \begin{minipage}{0.56\textwidth}
    \begin{itemize}
      \item Sender/Empfänger
      \item PC mit Soundkarte (Bahndatenberechnung und Steuerung der Frequenz)
      \item Mess- und Kontrollgeräte, Blitzschutz (nach geltenden Vorschriften)
    \end{itemize}
  \end{minipage}
  Für Satellitenfunk werden eine nachführbare Richtantennenanlage und ein Antennenvorverstärker benötigt, der unmittelbar an Antennenanlage montiert werden soll und bei Sendebetrieb zu schützen ist.
}
\card{50}{Abstrahlung und Ausbreitung elektromagnetischer Wellen, Feldstärke?}{
  \small
  HF-Schwingungen breiten sich in Leitern als Leitungswellen aus. Öffnet man den Leiter, beginnt er elektromagnetische Wellen abzustrahlen. Diese Leitungswellen gehen in Freiraumwellen über. Das auftretende Feld ist ein elektromagnetisches Feld. Dieses Feld wird beschrieben durch:
  \begin{itemize}
    \item elektrischen Feldanteil
    \item Frequenz des Wechselfeldes (in Hz)
    \item die elektromagnetische Feldstärke (in V/m)
    \item die Polarisation des elektrischen Feldvektors (als Feldgrößen)
  \end{itemize}
}
\card{51}{Was versteht man unter Freiraumausbreitung?}{
  Unter der \emph{Freiraumausbreitung} versteht man die Ausbreitung des elektromagnetischen Feldes im materiefreien Raum (Vakuum). Bei Freiraumausbreitung nimmt die Feldstärke mit wachsender Entfernung nur auf Grund der Entfernung ab (Entfernungsdämpfung). Freiraumbedingungen herrschen praktisch im Weltraum und noch mit sehr guter Näherung innerhalb des optischen Horizontes, wenn sonst keine störenden Effekte auftreten (Niederschlag, Reflexionen).
}
\card{52}{Welche Einflüsse haben Hindernisse auf die UKW-Ausbreitung?}{
  Ausbreitung über 100~MHz erfolgt quasi optisch. Unter der Annahme einer Standardatmosphäre, die eine Ablenkung der Funkstrahlen zum Boden bewirkt, ergibt sich für einen Standort eine max. Reichweite, die man als Funkhorizont bezeichnet. Je höher der Standort, desto größer die Reichweite. Durch Reflektion kann es zu einem Funkschatten kommen, der eine Funkverbindung unmöglich macht. Neben der Lage spielt also auch die Hindernisfreiheit eine wichtige Rolle.
}
\card{53}{Definieren Sie den Begriff ,,Schädliche Störung``?}{
  Ist eine Störung, welche die Abwicklung des Funkverkehrs bei einem anderen Funkdienst, Navigationsfunkdienst, Sicherheitsfunkdienst gefährdet oder den Verkehr bei einem Funkdienst, der in Übereinstimmung mit den für den Funkverkehr geltenden Vorschriften wahrgenommen wird, beeinträchtigt, behindert oder wiederholt unterbricht. Auch Amateurfunk kann von schädlichen Störungen betroffen sein.
}
\card{54}{Definieren Sie den Begriff ,,Senderleistung``?}{
  Die Sendeleistung ist die der Antennenspeiseleitung zugeführte Leistung. Messgröße ist Watt. Gemäß Amateurfunkverordnung.
}
\card{55}{Definieren Sie den Begriff ,,Spitzenleistung``?}{
  Die Spitzenleistung ist eine Effektivleistung, die ein Sender während einer Periode der Hochfrequenzschwingung während der höchsten Spitze der Modulationshüllkurve unverzerrt der Antennenspeiseleitung zuführt. Ident mit dem Begriff \emph{PEP (peak envelope power)}
  \[ \text{PEP} = (0,707 \cdot \text{Uss}/2)^2 / R_0 \]
}
\card{56}{Definieren Sie den Begriff ,,unerwünschte Aussendung``?}{
  Die der Antennenspeiseleitung am Ausgang des Sende-Empfängers zugeführten Störsignale auf jeder anderen Frequenz als der Trägerfrequenz samt den Seitenbändern, die sich aus dem Modulationsprozess ergeben. Gemäß Amateurfunkverordnung.
}

\end{document}
